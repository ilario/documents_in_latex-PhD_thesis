	%
	% If you are looking for the PDF file, you can compile the .tex file or donwload the PDF from here:
	% 
	% 
	% Istruzioni di compilazione (per produrre il pdf): avere un texlive funzionante e i pacchetti che vengono richiesti qui sotto; lanciare i comandi
	% pdflatex gelmetti-phd_thesis
	% biber gelmetti-phd_thesis
	% makeglossaries gelmetti-phd_thesis
	% texindy gelmetti-phd_thesis.idx
	% pdflatex gelmetti-phd_thesis
	% pdflatex gelmetti-phd_thesis

	
	\documentclass[b5paper, 12pt, openright]{book} % URV suggests B5
	\usepackage[utf8]{inputenc}
	%\usepackage[T1]{fontenc}% for using Bera Mono in Matlab listings
	\usepackage[spanish, es-noquoting, italian, english]{babel}
	% es-noquoting is needed for using < and > in siunitx https://github.com/josephwright/siunitx/issues/355#issuecomment-452731339
	
	\usepackage[product-units = single, list-units = single, range-units = single]{siunitx} % Typesetting quantities in a consistent way.
	% product-units controls if 1 x 1 m or 1 m x 1 m
	
	\DeclareSIUnit\sq{\ensuremath\Box} % for the sheet resistance units https://tex.stackexchange.com/questions/71220/si-units-for-sheet-resistance-using-siunitx
	\DeclareSIUnit\Molar{M}
	\DeclareSIUnit\rpm{rpm}
	\DeclareSIUnit\eq{eq}
	\DeclareSIUnit\eV{eV}
	\DeclareSIUnit\ppm{ppm}
	\DeclareSIUnit\unit{unit}
	\DeclareSIUnit\year{yr}
	
	\usepackage{upgreek} % for avoiding the warning "Package chemmacros Warning: You haven't loaded any package for upright Greek"
	\usepackage{chemmacros} 		% for \ch{} and \iupac{}
	%\chemsetup{formula={chemformula}}
	%\chemsetup[chemformula]{font-shape=sf}
	%\chemsetup[chemformula]{format=\sffamily}
		
	\usepackage{verbatim} % for using curly braces inside \iupac of chemmacros
	
	\usepackage[backend=biber, labelnumber=true, url=true, isbn=false, giveninits=true, sorting=none, maxnames=30, style=numeric-comp]{biblatex}%
	
	% from https://tex.stackexchange.com/questions/278890/biblatex-suppressing-urldate-does-not-work-clearfield
	\AtEveryBibitem{\ifentrytype{article}{\clearfield{url}\clearfield{urlyear}}{}}
	%\AtEveryBibitem{\ifentrytype{article}{\togglefalse{bbx:url}}{}}
	\newcommand{\authoryear}[1]{\citeauthor*{#1}~(\citeyear{#1}, \cite{#1})}
	
	\DeclareBibliographyCategory{mypapers}
	\addtocategory{mypapers}{Moia2019,Gelmetti2019,Gelmetti2017}
	
	%\usepackage{chemnum}			% Package used to manage index of compounds and schemes
	\usepackage{graphicx}
	\usepackage[textfont={small}]{subcaption} %the subfigure and subfig packages are deprecated
	\usepackage{ragged2e} % needed by ragged option of sidecap?
	
	
	%\usepackage{chngcntr} % needed for setting option not resetting footnote counter at chapter end
	%\usepackage{pgfplots} % graphs
	%\usepackage{booktabs}
	%\usepackage{lscape} %for big landscape figures, rotate the page
	%\usepackage{afterpage} %for making landscape float
	\usepackage[section]{placeins} %keeps floats `in their place', preventing them from floating past a "\FloatBarrier" command into another section.
	%\usepackage{float}
	\usepackage{eso-pic} %background image on titlepage
	\usepackage{transparent} %transparency for background image on titlepage
	
	\usepackage[format=plain, indention=0.5cm, labelfont={small,bf}, textfont=small]{caption}%personalize figure and table captions
	\newcommand{\mycaption}[2][]{\caption[#1]{\textbf{#1} #2}}
	


	\usepackage[ragged, centerbody, rightcaption]{sidecap}%wide does not work neither setting marginparwidth in geometry, I have no idea why. centerbody is undocumented, I'm not sure it works.
	\captionsetup[SCfigure]{indention=0cm}
	
	\usepackage{hyperref}

	\usepackage{xcolor} % for colors in hyperref links
	\hypersetup{
		colorlinks,
		linkcolor={red!10!black},
		citecolor={blue!50!black},
		urlcolor={blue!80!black},
%			bookmarks=true,         % show bookmarks bar?
%			unicode=false,          % non-Latin characters in Acrobat’s bookmarks
%			pdftoolbar=true,        % show Acrobat’s toolbar?
%			pdfmenubar=true,        % show Acrobat’s menu?
%			pdffitwindow=false,     % window fit to page when opened
%			pdfstartview={FitH},    % fits the width of the page to the window
%			pdftitle={My title},    % title
%			pdfauthor={Author},     % author
%			pdfsubject={Subject},   % subject of the document
%			pdfcreator={Creator},   % creator of the document
%			pdfproducer={Producer}, % producer of the document
			pdfkeywords={perovskite, solar cells, photovoltaics, photophysics, modelling, characterization, drift-diffusion, fabrication, hole transporting materials, recombination, simulation, ion migration, halide perovskite, hybrid perovskite, perovskite solar cells, impedance spectroscopy, solar energy, optical transients, electrical transients, data acquisition, data analysis, open circuit voltage, spin coating, hole transport layer, charge transfer, semiconductor, time resolved spectroscopy} % list of keywords
%			pdfnewwindow=true,      % links in new PDF window
%			colorlinks=false,       % false: boxed links; true: colored links
%			linkcolor=red,          % color of internal links (change box color with linkbordercolor)
%			citecolor=green,        % color of links to bibliography
%			filecolor=magenta,      % color of file links
%			urlcolor=cyan           % color of external links
	}
	
	%%%%%%%%%%%
	%%%%%%%%%%% GLOSSARIES
	%%%%%%%%%%%
	\usepackage[xindy, nonumberlist, acronym, nomain]{glossaries} %table of acronyms
	\makeglossaries % has to be after glossaries and after hyperref
	\setacronymstyle{long-sp-short}

%	\usepackage{relsize}%needed by textsmaller in glossaries
%	\renewcommand*{\acronymfont}[1]{\textsmaller{#1}}
	
	% for the table of symbols, this soultion could be used: https://tex.stackexchange.com/questions/348640/how-to-effectively-use-list-of-symbols-for-a-thesis
	
	%\usepackage{xspace} % for adding a conditional space in macros, dangerous as explained here: https://tex.stackexchange.com/questions/25820/no-space-following-macro-without-argument/25825#25825
	
	%%%%%%%%%%%
	%%%%%%%%%%% MULTIROW
	%%%%%%%%%%%
	\usepackage{multirow}% for tables
	%\usepackage[bottom,multiple]{footmisc} %bottom insert additional whitespace between the main text and the footnote rule rather than over-stretching the inter-paragraph glue AND to prevent the floats going below footnotes. This has to be loaded before fancyhdr
	
	%%%%%%%%%%%
	%%%%%%%%%%% FANCYHDR
	%%%%%%%%%%%
	\usepackage{fancyhdr}%page headers
	\usepackage[hmarginratio=3:2, vmarginratio=1:1, b5paper, textwidth=13cm, textheight=20cm]{geometry}% marginparwidth inserted only for wide sidecaps. 
	
	%%%%%%%%%%%
	%%%%%%%%%%% MAKEIDX
	%%%%%%%%%%%
	\usepackage{makeidx} % for alphabetical index, using xindy or makeindex
	\makeindex
	\usepackage[strict=true,autostyle=true, english = british]{csquotes} % for automatic direction of quotes`
	\MakeOuterQuote{"}
	
	%%%%%%%%%%%
	%%%%%%%%%%% AMSSYMB
	%%%%%%%%%%%
	\usepackage{amssymb}% for the \square math symbol

% this replaces the usage of \bm (bold in math), taken from https://tex.stackexchange.com/a/124311/173645
\makeatletter
\g@addto@macro\bfseries{\boldmath}
\makeatother
	
	\usepackage[super]{nth} % for superscript 1st 2nd, to be used with \nth{1}
	
	\usepackage{epigraph}% for inserting stupid phrases in random places
	\setlength{\epigraphrule}{0pt}% eliminate the rule below the epigraphs
	
	
	\usepackage{listings}
	\usepackage[numbered,framed]{matlab-prettifier}% for Matlab listings highlighting
	\lstset{style=Matlab-editor, basicstyle=\footnotesize}% basicstyle has to be placed after style
	
	\usepackage{cleveref} % for automatic Figure or figure and for ranges
	%\crefdefaultlabelformat{#2\textbf{#1}#3} % bold numbers of references https://tex.stackexchange.com/a/255187/173645
	\creflabelformat{figure}{#2\textcolor{blue}{#1}#3} % bold numbers just for figures references
	
	\usepackage[spacing=true, stretch=40, shrink=40]{microtype} % automagically reduces overfull lines and improves justification

	\usepackage{titlesec}
	% standatard definitions from section 9.2 in titlesec manual
%\titleformat{\chapter}[display]{\normalfont\huge\bfseries}{\chaptertitlename\ \thechapter}{20pt}{\Huge}
%\titleformat{\section}{\normalfont\Large\bfseries}{\thesection}{1em}{}
%\titleformat{\subsection}{\normalfont\large\bfseries}{\thesubsection}{1em}{}
%\titleformat{\subsubsection}{\normalfont\normalsize\bfseries}{\thesubsubsection}{1em}{}
%\titleformat{\paragraph}[runin]{\normalfont\normalsize\bfseries}{\theparagraph}{1em}{}
%\titleformat{\subparagraph}[runin]{\normalfont\normalsize\bfseries}{\thesubparagraph}{1em}{}

% add negative spacing
\titleformat{\section}{\normalfont\Large\bfseries}{\hspace*{-0.8cm}\thesection}{1em}{}
\titleformat{\subsection}{\normalfont\large\bfseries}{\hspace*{-0.8cm}\thesubsection}{1em}{}
\titleformat{\paragraph}[runin]{\normalfont\normalsize\bfseries}{\theparagraph}{1em}{\hspace*{-0.8cm}}
	%\newcommand{\myp}[1]{\paragraph{\hspace*{-0.5cm}#1}}
	
	\usepackage{etoolbox}
	% \sectionmark is broken for the page of first appearance
	% solution from https://tex.stackexchange.com/a/241299/173645
	% this sets the title for the ToC as the same of the section title, but the title for the header as a shorter alternative
	% usage: \mysection[header content]{index and title content}
	\usepackage{ifthen} % provides \ifthenelse test  
	\newcommand{\mysection}[2][]{
		\ifthenelse{\equal{#1}{}}%
			{\section{#2}}%
			{%
				\let\orisectionmark\sectionmark% solution from https://tex.stackexchange.com/a/241299/173645
				\renewcommand\sectionmark[1]{}%
				\section[#2]{#2\orisectionmark{#1}}%
				\orisectionmark{#1}%
				\let\sectionmark\orisectionmark%
			}%
	}
	
	\usepackage{needspace}
%	\preto\section{\vfil\penalty-10\vfilneg} % weakened \filbreak which would be \vfil\penalty-200\vfilneg % to avoid newpages to appear just below epitaphs, that usually are just after section names
%	\preto\section{\vskip 0pt plus 1fil\penalty-200\vskip 0pt plus -1fil}
	\preto\section{\needspace{7\baselineskip}}
	
	\usepackage{breqn} % for automatic line breaking in equations with dmath environment
	
%	\usepackage{commath} % for upright d in derivatives

	\usepackage{physics} % redefines plenty of things: \Im \Re \sin() \exp()... adds \dd for derivatives, \pdv for partial derivatives, \eval for evaluating the derivative in a point (does not work properly in dmath from breqn)

	\renewcommand{\sfdefault}{cmbr}%for lighter sans serif
	\captionsetup[SCfigure]{format=plain}
	%\pgfplotsset{compat=newest}
	%\counterwithout{footnote}{chapter} % not resetting footnote counter at chapter end
	\newcommand{\HNMR}{$^1$\ch{H}-\textsmaller{NMR}}
	\newcommand{\CNMR}{$^{13}$\ch{C}-\textsmaller{NMR}}
	%\pgfplotsset{tick label style={font=\small},label style={font=\small}}
	%\let\oldch\ch %move the ch command, needed for redefinition
	%\renewcommand{\ch}[1]{\textsf{\oldch{#1}}} %add sans serif to \ch
	
	
	% \cmpd*[cmpd-label = 1S]{ig2-1}
	\newcommand{\tallcell}[2][c]{\begin{tabular}[#1]{@{}c@{}}#2\end{tabular}}
	%\pgfplotsset{minor grid style={densely dotted,lightgray}}
	\newcommand{\acr}[1]{\gls{#1}\index{\glsdesc{#1}}} % for having the entry also in index
	\newcommand{\Acr}[1]{\Gls{#1}\index{\glsdesc{#1}}}
	
	
	\newcommand{\me}{\mathrm{e}} % for having upright neper number symbol, from https://tex.stackexchange.com/a/19491/173645
	\newcommand{\voc}{V_\mathrm{OC}} 
	\newcommand{\jsc}{J_\mathrm{SC}}
	
%	\newcommand{\PbItwoS}{\ch{PbI2}\index{lead diiodide} }
%	\newcommand{\PbBrtwoS}{\ch{PbBr2}\index{lead dibromide} }
%	\newcommand{\PbCltwoS}{\ch{PbCl2}\index{lead dichloride} }
%	\newcommand{\CsIS}{\ch{CsI}\index{cesium iodide} }
%	\newcommand{\TiOtwoS}{\ch{TiO2}\index{titanium dioxide} }
%	\newcommand{\dTiOtwoS}{\ch{d-TiO2}\index{titanium dioxide} }
%	\newcommand{\mpTiOtwoS}{\ch{mp-TiO2}\index{titanium dioxide} }
%	
%	\newcommand{\spiroS}{\gls{spiro}\index{\gls{spiro}} }
%	\newcommand{\SpiroS}{\Gls{spiro}\index{\gls{spiro}} }
%	\newcommand{\taeS}[1]{\gls{tae#1}\index{\gls{tae#1}} } % this allows to use \tae1, with a number in the command name. Maybe it's a bad idea, see: https://texfaq.org/FAQ-linmacnames
%	\newcommand{\TaeS}[1]{\Gls{tae#1}\index{\gls{tae#1}} }
%	
%	\newcommand{\PbItwo}{\ch{PbI2}\index{lead diiodide}}
%	\newcommand{\PbBrtwo}{\ch{PbBr2}\index{lead dibromide}}
%	\newcommand{\PbCltwo}{\ch{PbCl2}\index{lead dichloride}}
%	\newcommand{\CsI}{\ch{CsI}\index{cesium iodide}}
%	\newcommand{\TiOtwo}{\ch{TiO2}\index{titanium dioxide}}
%	\newcommand{\dTiOtwo}{\ch{d-TiO2}\index{titanium dioxide}}
%	\newcommand{\mpTiOtwo}{\ch{mp-TiO2}\index{titanium dioxide}}
%	
%	\newcommand{\spiro}{\gls{spiro}\index{\gls{spiro}}}
%	\newcommand{\Spiro}{\Gls{spiro}\index{\gls{spiro}}}
%	\newcommand{\tae}[1]{\gls{tae#1}\index{\gls{tae#1}}} % this allows to use \tae1, with a number in the command name. Maybe it's a bad idea, see: https://texfaq.org/FAQ-linmacnames
%	\newcommand{\Tae}[1]{\Gls{tae#1}\index{\gls{tae#1}}}
	
	\bibliography{biblio}
	
	\renewbibmacro*{name:andothers}{% Based on name:andothers from biblatex.def for "et al." in italic
	\ifboolexpr{
	test {\ifnumequal{\value{listcount}}{\value{liststop}}}
	and
	test \ifmorenames
	}
	{\ifnumgreater{\value{liststop}}{1}
	{\finalandcomma}
	{}%
	\andothersdelim\bibstring[\emph]{andothers}}
	{}}
	

	\renewcommand{\floatpagefraction}{.8}% less likely to have images by their own in a page https://tex.stackexchange.com/questions/68516/avoid-that-figure-gets-its-own-page
	
	
% usually subscripts should be typed in upright font, like with \mathrm{}, unless they're indexes
% this allows me to use V_|OC| for upright subscripts
% https://tex.stackexchange.com/a/156963/173645
\makeatletter
\begingroup
\catcode`\_=\active
\protected\gdef_{\@ifnextchar|\subtextup\sb}
\endgroup
\def\subtextup|#1|{\sb{\textup{#1}}}
\AtBeginDocument{\catcode`\_=12 \mathcode`\_=32768}
\makeatother


	
	
	\hyphenation{tetra-hydro-furan di-chlo-ro-methane enantio-purity oxy-methyl-ene di-chroic aceto-nitrile iso-propyl alk-oxy-amine poly-thio-phene di-chloride inter-chain photo-lu-mine-scence supra-molecular photo-gene-rated hetero-junction poly-thio-phenic di-block thio-phenic poly-thio-phene bi-layer macro-initiator poly-vinyl-pyridine homo-polymer path-length intra-molecular inter-molecular hydro-chloric poly-dispersity poly-disperse chlo-ro-ben-ze-ne acetyl-acetone di-iso-prop-oxide acetyl-acet-on-ate aceto-nitrile}
	
	\title{Advanced characterization and modelling of charge transfer in perovskite solar cells.}
	\author{Ilario Gelmetti}
	\date{2018}
\begin{document}
% I cite here the papers which should be numbered first	
\nocite{Gelmetti2017,Moia2019,Gelmetti2019}

\pagestyle{plain}

\selectlanguage{english}
%\apptocmd{\@epitext}{\penalty=500}{}{}
%\apptocmd{\epigraph}{\nopagebreak}{}{}
%\AtEndEnvironment{\epigraphflush}{\nopagebreak}

\frontmatter

{\let\cleardoublepage\clearpage % i don't want openright in frontmatter

\begin{titlepage}\begin{center}


		\AddToShipoutPicture*{
			\put(0,0){
				\parbox[b][\paperheight]{\paperwidth}{%
					\vfill
					\centering
					\transparent{0.04}%stampato risulta randomicamente più scuro o più chiato, on paper this image results much more visible (or less, randomly) than on screen
					{\includegraphics[width=0.9\paperwidth]{contents_img/emblem-urv.pdf}}%
					\vfill
				}
			}
		}

		\makebox[\textwidth]{\centering{\includegraphics[width=.27\textwidth]{contents_img/logo-urv.pdf}}}\\%
		\bigskip
		\large{\textsc{Universitat Rovira i Virgili\\ Institute of Chemical Research of Catalonia}}\\
		\rule{5cm}{1pt}\\
		{
		\bigskip
		{\textsc{Doctoral Thesis} in \textsc{Electronic Engineering}}}\\
		\bigskip\bigskip\vfill

		\huge{\textsc{Advanced characterization and modelling of charge transfer in perovskite solar cells}}\\
		\bigskip\bigskip\vfill
		\footnotesize{Author:}\\
		\large{Ilario Gelmetti}\\
		\makebox[.2\textwidth]{\rule{0pt}{.02\textheight}}\\
	\end{center}
	\begin{small}
		\makebox{\parbox[b]{.8\textwidth}{



				\flushleft{\textbf{Supervised by:}}
				\flushright{\begin{tabular}{r c}
						{\large Prof. Dr. Emilio Jose Palomares Gil} & \makebox[.4\textwidth]{\dotfill} \\
					\end{tabular}}
				\bigskip

				\flushleft{\textbf{Author:}}
				\flushright{\begin{tabular}{r c}
						{\large Ilario Gelmetti} & \makebox[.4\textwidth]{\dotfill} \\
					\end{tabular}}

			}}
		\bigskip\bigskip\bigskip\bigskip
		\begin{center}
			\rule{3cm}{1pt}\\
			Session: 2018\\
		\end{center}
	\end{small}
\end{titlepage}

\addtocounter{page}{-1}

\newpage

\addcontentsline{toc}{chapter}{Statement of Authorship}

\makeatletter
\noindent I, \@author, declare that this thesis titled, \enquote{\@title} and the work presented in it are my own. I confirm that:
\makeatother

\begin{itemize}
	\item This work was done wholly or mainly while in candidature for a doctoral degree at this University.
	\item Where any part of this thesis has previously been submitted for a degree or any other qualification at this University or any other institution, this has been clearly stated.
	\item Where I have consulted the published work of others, this is always clearly attributed.
	\item Where I have quoted from the work of others, the source is always given. With the exception of such quotations, this thesis is entirely my own work.
	\item I have acknowledged all main sources of help.
	\item Where the thesis is based on work done by myself jointly with others, I have made clear exactly what was done by others and what I have contributed myself.
\end{itemize}

\bigskip\bigskip

\noindent Signed:\\
\rule[0.5em]{25em}{0.5pt} % This prints a line for the signature

\bigskip

\noindent Date:\\
\rule[0.5em]{25em}{0.5pt} % This prints a line to write the date

%%%%%%%%%%%%
% ACRONYMS %
%%%%%%%%%%%%

% techniques
\newacronym{pl}{PL}{photoluminescence}
\newacronym{uvvis}{UV--vis}{ultraviolet--visible}
\newacronym{nmr}{NMR}{nuclear magnetic resonance}
\newacronym{xrd}{XRD}{X--ray diffraction}
\newacronym{sem}{SEM}{scanning electron microscope}
\newacronym{esem}{ESEM}{environmental scanning electron microscopy}
\newacronym{edx}{EDX}{energy-dispersive X-ray analysis}
\newacronym{tpv}{TPV}{transient photovoltage}
\newacronym{tpc}{TPC}{transient photocurrent}
\newacronym{ce}{CE}{charge extraction}
\newacronym{dc}{DC}{differential capacitance}
\newacronym{mppt}{MPPT}{maximum power point tracking}
\newacronym{tcspc}{TCSPC}{time-correlated single photon counting}
\newacronym{kpfm}{KPFM}{Kelvin probe force microscopy}
\newacronym{afm}{AFM}{atomic force microscopy}
\newacronym{ac-afm}{AC-AFM}{alternating current atomic force microscopy}
\newacronym{trpl}{TRPL}{time-resolved photoluminescence}

% physical properties
\newacronym{homo}{HOMO}{highest-energy occupied molecular orbital}
\newacronym{lumo}{LUMO}{lowest-energy unoccupied molecular orbital}
\newacronym{ff}{FF}{fill factor}
\newacronym{pce}{PCE}{power conversion efficiency}
\newacronym{voc}{V\textsubscript{OC}}{open circuit voltage}
\newacronym{jsc}{J\textsubscript{SC}}{short circuit current}
\newacronym{eqe}{EQE}{external quantum efficiency}

% other
\newacronym{rt}{RT}{room temperature}
\newacronym{htm}{HTM}{hole transporting material}
\newacronym{etm}{ETM}{electron transporting material}
\newacronym{led}{LED}{light emitting diode}
\newacronym{osc}{OSC}{organic solar cells}
\newacronym{dssc}{DSSC}{dye sensitized solar cells}
\newglossaryentry{rctime}{name={RC time}, description={characteristic time of a circuit composed of a resistance in parallel to a capacitance}}

% long non-acronym names and descriptions

% molecules and materials
\newglossaryentry{pcbm60}{name={\ch{PC60BM}}, description={\iupac{[6,6]-phenyl C\textsubscript{61} butyric acid methyl ester}}}
\newglossaryentry{pcbm70}{name={\ch{PC70BM}}, description={\iupac{[6,6]-phenyl C\textsubscript{71} butyric acid methyl ester}}}
\newglossaryentry{ptfe}{name={\ch{PTFE}}, description={\iupac{poly|tetra|fluoro|ethyl|ene}}}
\newglossaryentry{fto}{name={\ch{FTO}}, description={fluorine tin oxide}}
\newglossaryentry{ito}{name={\ch{ITO}}, description={indium tin oxide}}
\newglossaryentry{pedotpss}{name={\ch{PEDOT:PSS}}, description={\iupac{poly\-(3,4-ethyl|ene|di|oxy|thio|phene) poly|(styr|ene|sulfon|ate)}}}
\newglossaryentry{mai}{name={\ch{MAI}}, description={methylammonium iodide}}
\newglossaryentry{mabr}{name={\ch{MABr}}, description={methylammonium bromide}}
\newglossaryentry{fai}{name={\ch{FAI}}, description={formamidinium iodide}}
\newglossaryentry{dmso}{name={\ch{DMSO}}, description={dimethyl sulfoxide}}
\newglossaryentry{dmf}{name={\ch{DMF}}, description={\iupac{N,N-di|methyl|form|amide}}}
\newglossaryentry{pes}{name={\ch{PES}}, description={\iupac{poly|ether|sulf|one}}}
\newglossaryentry{mapi}{name={\ch{MAPbI3}}, description={\iupac{methyl|ammonium lead iodide}}}
\newglossaryentry{mapicl}{name={\ch{MAPbI_{3-x}Cl_x}}, description={\iupac{methyl|ammonium lead iodide chloride}}}
\newglossaryentry{famapbibr}{name={\ch{FAMAPbIBr}}, description={\iupac{formam|idinium methyl|ammonium lead iodide bromide}}}
\newglossaryentry{csfamapbibr}{name={\ch{CsFAMAPbIBr}}, description={\iupac{cesium formam|idinium methyl|ammonium lead iodide bromide}}}
\newglossaryentry{csfamapbibr_complete}{name={\ch{Cs_{0.06}FAMA_{0.2}Pb_{1.3}I_{3.2}Br_{0.6}}}, description={\iupac{cesium formam|idinium methyl|ammonium lead iodide bromide}}}
\newglossaryentry{litfsi}{name={\ch{LiTFSI}}, description={\iupac{bis|(tri|fluoro|methane)|sulfon|imide lithium salt}}}
\newglossaryentry{spiro}{name={\iupac{spiro-OMeTAD}},description={\iupac{N2,N2,N2',N2',N7,N7,N7',N7'-octa|kis|(4-meth|oxy|phenyl)|-9,9'-spiro|bi|[9H-fluor|ene]|-2,2',7,7'-tetr|amine}}}
\newglossaryentry{tae1}{name={TAE-1},description={\iupac{tetra|\{4-|[N,N-|(4,4'-di|meth|oxy|di|phen|yl|amino)]|phen|yl\}|eth|ene}}}
\newglossaryentry{tae3}{name={TAE-3},description={\iupac{3,3',6,6'-tetra|kis|[N,N-bis|(4-meth|oxy|phen|yl)|amino]|-9,9'-bi|fluor|en|yl|id|ene}}}
\newglossaryentry{tae4}{name={TAE-4},description={\iupac{3,3',6,6'-tetra|kis|(3,6-di|meth|oxy-|9H-carba|zol-9-yl)|-9,9'-bi|fluor|en|yl|id|ene}}}
\newglossaryentry{ptaa}{name={PTAA}, description={\iupac{poly[bis(4-phen|yl)(2,4,6-tri|methyl|phen|yl)amine}}}
\newglossaryentry{loess}{name={LOESS},description={locally estimated scatterplot smoothing}}

% symbols
\newglossaryentry{symb:v}{name={$v$}, description={ideality of pseudo first order life-time \textit{versus} light bias}}
\newglossaryentry{symb:m}{name={$m$}, description={ideality of charge \textit{versus} light bias}}
\newglossaryentry{symb:mu}{name={$\mu$}, description={internal chemical potential}, sort={mu}}
\newglossaryentry{symb:barmu}{name={$\bar\mu$}, description={electrons electrochemical potential}, sort={mubar}}
\newglossaryentry{symb:V}{name={\ensuremath{V}}, description={voltage}}
\newglossaryentry{symb:P}{name={\ensuremath{P}}, description={electrical power}}
\newglossaryentry{symb:J}{name={\ensuremath{J}}, description={current or current density}}
\newglossaryentry{symb:phi}{name={\ensuremath{\phi}}, description={illumination intensity}, sort={phi}}
\newglossaryentry{symb:nid}{name={\ensuremath{n_{\mathrm{id}}}}, description={ideality factor}}
\newglossaryentry{symb:kB}{name={\ensuremath{k_{\mathrm{B}}}}, description={Boltzmann constant}}
\newglossaryentry{symb:T}{name={\ensuremath{T}}, description={temperature}}
\newglossaryentry{symb:Jph}{name={\ensuremath{J_{\mathrm{ph}}}}, description={photogenerated current}}
\newglossaryentry{symb:J0}{name={\ensuremath{J_0}}, description={dark diode saturation current}}
\newglossaryentry{symb:alpha}{name={\ensuremath{\alpha}}, description={power law exponent in current \textit{versus} illumination intensity relationship}, sort={alpha}}
\newglossaryentry{symb:q}{name={\ensuremath{q}}, description={elementary charge}}
\newglossaryentry{symb:n}{name={\ensuremath{n}}, description={electrons density}}
\newglossaryentry{symb:p}{name={\ensuremath{p}}, description={electron holes density}}
\newglossaryentry{symb:ni}{name={\ensuremath{n_{\mathrm{i}}}}, description={intrinsic carrier density}}
\newglossaryentry{symb:epsilonzero}{name={\ensuremath{\epsilon_0}}, description={electric constant}, sort={epsilonzero}}
\newglossaryentry{symb:epsilonr}{name={\ensuremath{\epsilon_{\mathrm{r}}}}, description={relative permittivity}, sort={epsilonr}}
\newglossaryentry{symb:Eg}{name={\ensuremath{E_{\mathrm{g}}}}, description={band gap}}
\newglossaryentry{symb:A}{name={\ensuremath{A}}, description={area}}
\newglossaryentry{symb:d}{name={\ensuremath{d}}, description={distance, thickness}}
\newglossaryentry{symb:Cg}{name={\ensuremath{C_{\mathrm{g}}}}, description={geometric capacitance}}
\newglossaryentry{symb:Cion}{name={\ensuremath{C_{\mathrm{ion}}}}, description={ionic capacitance}}
\newglossaryentry{symb:g}{name={\ensuremath{g}}, description={charges photo-generation rate}}
\newglossaryentry{symb:U}{name={\ensuremath{U}}, description={charges recombination rate}}
\newglossaryentry{symb:t}{name={\ensuremath{t}}, description={time}}
\newglossaryentry{symb:tau}{name={\ensuremath{\tau}}, description={recombination life-time}, sort={tau}}
\newglossaryentry{symb:Phi}{name={\ensuremath{\Phi}}, description={reaction order}, sort={Phi}}
\newglossaryentry{symb:C}{name={\ensuremath{C}}, description={capacitance}}
\newglossaryentry{symb:Z}{name={\ensuremath{Z}}, description={complex impedance}}
\newglossaryentry{symb:omega}{name={\ensuremath{\omega}}, description={angular frequency}, sort={omega}}
\newglossaryentry{symb:ux}{name={\ensuremath{u_x}}, description={placeholder for useless constants}}
%\newglossaryentry{symb:}{name={\ensuremath{}}, description={}, sort={}}
%\newglossaryentry{symb:}{name={\ensuremath{}}, description={}, sort={}}
%\newglossaryentry{symb:}{name={\ensuremath{}}, description={}, sort={}}
%\newglossaryentry{symb:}{name={\ensuremath{}}, description={}, sort={}}
%\newglossaryentry{symb:}{name={\ensuremath{}}, description={}, sort={}}
%\newglossaryentry{symb:}{name={\ensuremath{}}, description={}, sort={}}
%\newglossaryentry{symb:}{name={\ensuremath{}}, description={}, sort={}}
%\newglossaryentry{symb:}{name={\ensuremath{}}, description={}, sort={}}

\tableofcontents
\listoffigures
\listoftables
\lstlistoflistings

\glsaddall
\printglossary[type=\acronymtype, title={Glossary, Acronyms, and Symbols}]

}%here ends openany and starts openright

\cleardoublepage \vspace*{12cm}
\begin{flushright}
	\large {\it Dedicata a So$\phi$a.}
\end{flushright}

\newgeometry{top=1in, bottom=1in, outer=1in, inner=1in}

\chapter[Summary of the Thesis]{\centering Summary of the Thesis}
%	In \cref{ch:intro}, an overview on the solar cells main concepts and working mechanisms are presented.
Perovskite solar cells are inserted in the provided framework and the research state of the art is briefly described. 


	\EnableQuotes


\chapter[Motivation and Aims]{Motivation and Aims}\label{ch:aims}
	%\addcontentsline{toc}{chapter}{Motivation and Aims}
%	\input{contents_tex/aims.tex}


	\restoregeometry

	\fancypagestyle{plain}{% chapter pages use style plain
		\renewcommand{\headrulewidth}{0pt} % no top rule
		\fancyhf{}% clear all fields
		\fancyfoot[L]{\footnotesize{Ilario Gelmetti}}% left footer
		\fancyfoot[C]{\footnotesize{\textit{Photophysics of perovskite}}}
		\fancyfoot[R]{\thepage}% right footer
	}

	\mainmatter
	\pagestyle{headings}

\chapter{Introduction}\label{ch:intro}
	\graphicspath{ {./contents_img/intro/} }
	
\section{Why Photovoltaic}

	\paragraph{Abundance}
	The total solar irradiance hitting the outer Earth atmosphere is around \SI{1361}{\watt\per\square\metre},\cite{Kopp2011} which, considering our planet cross section area, makes \SI{1.6e17}{\watt}.
	Nature conveys this energy in plenty of ways, including the generation of every renewable and most of non-renewable energy sources.
	Solar energy is effectively the primary energy source for our planet's ecosystem, so it is the most interesting source of energy for human usage.

	JUST A SMALL PART OF THIS INCOMING ENERGY IS GETTING USED


	\paragraph{Availability}
	The ubiquitous availability of solar power can be the leverage for an economical power levelling across different regions of the planet.
	Indeed, the abundance of solar irradiation (represented in \cref{fig:world_map-PVOUT} by the ratio between the photovoltaic electric energy that can be obtained over the nominal power of an installed solar panel) is quite high in most of the regions where life quality is seriously affected by economical situation (represented in \cref{fig:world_map-HDI} by the Human Development Index).

	\begin{figure}%[!hbtp]%
		\makebox[\textwidth][c]{
			\parbox{1.1\textwidth}{
				\centering
				\begin{subfigure}[b]{1\textwidth}
					\includegraphics[width=0.9\textwidth]{world_map-HDI/world_map-HDI.pdf}
					\subcaption{Human development index by region}\label{fig:world_map-HDI}
				\end{subfigure}

				\begin{subfigure}[b]{1\textwidth}
					\includegraphics[width=0.9\textwidth]{world_map-PVOUT/world_map-PVOUT.pdf}
					\subcaption{Daily photovoltaic electricity potential}\label{fig:world_map-PVOUT}
				\end{subfigure}

				\begin{subfigure}[b]{1\textwidth}
					\includegraphics[width=0.9\textwidth]{world_map-electrical_vs_land/world_map-electrical_vs_land.pdf}
					\subcaption{Yearly electricity consumption over region land area}\label{fig:world_map-electrical_vs_land}
				\end{subfigure}
				\mycaption[Geographical distribution of development, sunlight and consumption.]{Data in (\textbf{a}) represent the human development index:
					"a summary measure of average achievement in key dimensions of human development: a long and healthy life, being knowledgeable and have a decent standard of living"
					from UNDP\cite{UNDP2018} (missing data in grey); data in (\textbf{b}) represents the photovoltaic electricity potential: considering a photovoltaic module installed in a region, is the ratio between the average daily produced energy, in kWh, and the nominal power, or nameplate capacity, of the installed module;\cite{Solargis2018} data in (\textbf{c}) represents the yearly electricity consumption\cite{CIAa} (comparing total electricity generated annually plus imports and minus exports) on a country scale divided by country land surface,\cite{CIA} expressed in kilowatt-hour per year and per square kilometre.}\label{fig:world_map}
			}}
	\end{figure}

	\paragraph{Resilience}
	The usage of photovoltaic energy source is compatible with a distributed and decentralized network model.
	Such a power network can be much more resilient than the fossil fuels based system, with the only single-point-of-failure being the climate variability.
	%	is the electric energy production method that best matches a distributed and decentralized network model, 
	Combined with accumulation (needed for night time usage) and with other energy sources, it can be the pivot of an extremely reliable electric energy provisioning system.

	\paragraph{More energy is not enough} It is intuitive that increasing the energy production is a high-price solution to the growing energetic demand.
	Indeed, in some regions a decrease in the electricity consumption has to be included for a long-term solution.
	For example, a study\cite{Margolis2016} reports that if every rooftop (not considering utility-scale solar facilities) in United States of America was covered with solar panel, just the 39~\% of its nowadays national consumption would be covered.
	In \cref{fig:world_map-electrical_vs_land} we can see how electricity consumption density is greatly inhomogeneous and comparing with the photovoltaic potential map in \cref{fig:world_map-PVOUT}, it is evident that such a problem is shared with many other poorly insulated but energy eager regions.
	Both an increase in machinery's efficiency and a change in life-style can be part of the solution, following the example and thinking at life-style, in USA the \textsl{per capita} energy usage is more than twice the European average, and five times the Latin America average.\cite{IEA}

	\paragraph{And more research is needed} Every source of electrical energy we can think of works \textsl{via} the conversion of motive power (a flow of steam, wind, or water, waves, tides...) to electricity, which relays on the well established electric generator.
	Except photovoltaic energy.
	This very simple difference already hints for the huge conceptual and technological step required by photovoltaics as compared to other electrical energy sources.

	% transducer
	\mysection[Relevant Physics]{Relevant Physics for Stacked Semiconductor Thin Films}

	\subsection{Energy Levels and Occupancy}

		\paragraph{Valence and conduction bands}

		\paragraph{Boltzmann and Fermi-Dirac statistics}

		\paragraph{Fermi and quasi-Fermi levels}

	\subsection{Electrostatics}

		\paragraph{Poisson equation}

		\paragraph{Space charge limited layers, Debye length and electric field screening}\label{intro-space_charge}
		\cite{WikipediaDebye2019}

	\subsection{Electrodynamics}

		\paragraph{Charge diffusion}

		\paragraph{Charge drift}

		\paragraph{Displacement current -- definition}\label{intro_displacement_current} The displacement current $J_D$ appears in the fourth macroscopic Maxwell's equation as $\partial D / \partial t$ for describing the contributions to magnetizing field not originated by a current of free charges $J_f$: $\nabla \times H = J_f + \frac{\partial D}{\partial t}$.
		The displacement electric field $D$ includes contributions from the electric field intensity $E$ and the polarization $P$: $D=\epsilon_0 E + P$ which can also be written in terms of relative permittivity $\epsilon_r$ as: $D= \epsilon_0 \epsilon_r E$.
		So its time derivative defining the displacement current is: $\frac{\partial D}{\partial t} = \epsilon_0 (\epsilon_r\frac{\partial E}{\partial t} + E\frac{\partial \epsilon_r}{\partial t})$.
		For the scope of this thesis we're interested in the first term and we're going to ignore the second one considering the relative permittivity as a material dependent constant.


		\begin{figure}
			\makebox[\textwidth][c]{
				\parbox{1.1\textwidth}{
					\centering
					\begin{subfigure}[t]{0.5\textwidth}
						\includegraphics[width=1\textwidth]{displacement_current/first.pdf}
						\subcaption{Initial condition}\label{fig:displacement_current-initial}
					\end{subfigure}
					\bigskip

					\begin{subfigure}[t]{0.5\textwidth}
						\includegraphics[width=1\textwidth]{displacement_current/second.pdf}
						\subcaption{Free charges crossing interfaces}\label{fig:displacement_current-electronic}
					\end{subfigure}
					\qquad
					\begin{subfigure}[t]{0.5\textwidth}
						\includegraphics[width=1\textwidth]{displacement_current/third.pdf}
						\subcaption{Ionic charges accumulating at interfaces}\label{fig:displacement_current-ionic}
					\end{subfigure}

					\mycaption[Representation of current and displacement current.]{(\textbf{a}) a mixed ionic-electronic conductor material layered between two layers of electronic conductor material.
						(\textbf{b}) the electronic charge can cross the material's interfaces.
						(\textbf{c}) ionic charge cannot be transferred to the electrodes, still a displacement current due to the ionic migration is generated and can be measured in the amperometer.}\label{fig:displacement_current}
				}
			}
		\end{figure}


		\paragraph{Displacement current -- interpretation} So the displacement current accounts for the charge movements not identifiable as current and for their effect on the surrounding circuitry.
		An example is represented in \cref{fig:displacement_current-ionic}: the amperometer can measure a current even if no charge crossed the perovskite/electrode interface, this is a displacement current caused by the creation of a dipole due to the ionic accumulation at the interfaces.
		One can imagine the resulting current as needed for maintaining the zero potential difference between the two contacts.
		It is not, as one could erroneously and instinctively think, the effect of an electric field generated in the contacts by the moving charge in the perovskite layer (with related Coulomb force and image charges) as out of the two two-dimensional planes of opposed charges no electric field is present.
		This can be understood thinking that the electric field intensity generated by a large plate of charges has a constant magnitude at distances much smaller than the plate dimensions (which is always true in our solar cells considering the thickness \SI{\approx 1}{\um} to area \SI{3x3}{\square\mm} ratio, except at electrode edges).
		The concept is represented in \cref{fig:dipole_plane}.
		In \cpageref{displacement_current_ionic} we'll use the displacement current concept for simulating the current caused in the electrodes not by a net flux of charges through a device section but by the rearrangement of charges which not necessarily leave the device, like the ionic migration in perovskite material.

		\begin{figure}
			\makebox[\textwidth][c]{
				\parbox{1.1\textwidth}{
					\centering
					\begin{subfigure}[t]{0.3\textwidth}
						\includegraphics[height=0.3\textheight]{dipole_plane/single.pdf}
						\subcaption{Infinite plane of charges}\label{fig:dipole_plane-single}
					\end{subfigure}
					\qquad
					\begin{subfigure}[t]{0.3\textwidth}
						\includegraphics[height=0.3\textheight]{dipole_plane/double.pdf}
						\subcaption{Two infinite planes of opposed charges}\label{fig:dipole_plane-double}
					\end{subfigure}

					\mycaption[Electric field by two planes of opposite charges.]{(\textbf{a}) an infinite plane of charged particles is represented, the electric field is constant on the two sides, with a discontinuity in the plane.
						(\textbf{b}) adding a second plane with the same concentration of charges of the opposite kind, the electric field out of the sandwich cancels out.
						What is present outside of the sandwich is just an electrostatic potential difference.}\label{fig:dipole_plane}
				}
			}
		\end{figure}

		\paragraph{Continuity equations} The continuity equations have a very intuitive meaning: the concentration of a specie in a specific volume can increase due to generation $G$, can decrease due to recombination $U$ or can change due to a speed change of a charge flow passing by.
		This last contribution can be understood with the following example: a traffic light stops the cars flow and causes the increase of their concentration.
		For electron concentration $n$:
		\begin{equation}
			\frac{\partial n_x}{\partial t} = \frac{1}{q}\nabla_x \cdot J_n + G_{n,x} - U_{n,x}
		\end{equation}



	\subsection{Charge Recombination}

		\paragraph{Primary geminate recombination} \label{intro_geminate} This kind of recombination happens with the annihilation of a photo-generated exciton prior to the free charges separation.
		This is independent from the illumination intensity, so it can be considered as having a reaction order of zero.
		Lead halide perovskite have quite a high static permittivity \cite{Moia2019} and a small electron effective mass (high free carriers mobility) \cite{Herz2017}, this means that, at room temperature, the exciton binding energy is even smaller than $k_|B|T$ \cite{Miyata2015,Galkowski2016,Tvingstedt2015} and the direct generation of free charges occurs.
		So this kind of recombination is negligible in perovskite solar cells CITATION NEEDED.
		%Non-geminate recombination refers to the annihilation of an electron and an hole happening after their complete separation, as opposed to geminate recombination where the recombination happens just after the charges separation but before their distancing.

		\paragraph{Radiative recombination}
		This is an unavoidable form of recombination, which finds its origin in the detailed balance principle describing that, at steady state, all the absorbed thermal photons, coming from the surroundings of the cell, have to be re-emitted as black body radiation.
		From this parallelism it can be understood that this recombination is larger for materials with greater absorptivity \cite{Nelson2003} like the ones used for thin film solar cells \cite{Tvingstedt2015}.
		An indirect bandgap disfavours the radiative recombination: in these materials the recombination involves a large momentum variation, while the photon emitted due to the recombination can take just a small momentum and the rest of it should be released as an additional phonon (lattice vibrations).
		Radiative recombination involves the collision of two opposite free charges, so it can be considered as having a reaction order of 2 when electrons and holes concentrations are similar, $n \approx p$ (in intrinsic semiconductors out of the depletion layer or, in intrinsic perovskites, once ionic profile stabilized cancelling the electric field), while in case of uneven concentrations (in doped semiconductors or inside depletion layers) the limiting reagent is the minority carrier (electrons in p-type and holes in n-type materials) and the reaction order is 1.
		The expression used for modelling is: $U_|rad| = k_|rad| (np-n_|i|^2)$, where $n_|i|$ is the intrinsic carrier density as obtained from Boltzmann distribution for an intrinsic semiconductor, it is just introduced in the equation in order to account for the thermal generation.
		This type of recombination is present in perovskite solar cells, indeed some of the most efficient perovskite solar cells can be used as \gls{led} taking advantage of their radiative recombination \cite{Bi2016}.

		\paragraph{SRH trap mediated recombination}
		Also known as Shockley-Read-Hall recombination \cite{Shockley1952}, this recombination happens in two steps: a free charge from the respective band decays into an empty localized state with energy in between the valence and the conduction bands, a trap, after this event, a free charge of the opposite sign reaches the trapped charge location and recombine with it.
		As aforementioned, in an indirect band gap material the radiative recombination is disfavoured, but the presence of a trap state can divide in two easier steps the charge recombination process.
		%		In an indirect band gap material (where the momentum difference between the starting and final state is large) band-to-band bimolecular radiative recombination is disfavoured, but the presence of a localized trap state can catalyse the transition 
		Usually both the momentum difference and the chemical energy is released \textsl{via} phonons, so no radiative emission is involved.
		This recombination type can have reaction order of 1 or 2 depending on how many of these steps constitutes a bottleneck \cite{Calado2018b}.
		In case of mid-gap traps (with energy in the middle of the bandgap), these states will be rather saturated and the process of trapping a free charge will be slow as an empty trap has to be hit, then the actual recombination is fast, so the reaction order is 1.
		In case of shallow traps, the first step could or could not be a bottleneck, this time depending on the availability of free charges to trap: in doped semiconductors this could be a limiting factor, and the reaction order would be again 1; in intrinsic semiconductors there will be plenty of free charges of both types, and the reaction order can get close to 2.
		The expression used for modelling is \cite{Shockley1952}:
		\begin{equation}\label{eq:srh}
			U_{SRH} = k_{SRH} \frac{np-n_|i|^2}{\tau_n(p+p_t)+ \tau_p(n+n_t)}
		\end{equation}
		where $n_t$ and $p_t$ AAAAAAAAAAAAAAAAAAAAAAAAAAAAAAAAAAAAAAAA
		For lead halide hybrid perovskite materials, this kind of recombination is or is not important depending on the presence of mid-gap states, which in turn depends on the material composition and stoichiometry CITATION NEEDED.

		\paragraph{Surface recombination}\label{intro_surface_recombination}
		Also known as interfacial recombination, refers to the annihilation of one kind of free charge on a material with a charge of the opposite sign located on another material, happening at the materials' interface.
		At the interface between a, for example, p-doped and a close-to-intrinsic material as we consider the perovskite, the surface recombination will involve the majority carrier, holes, in the doped material and the opposite charge in the intrinsic semiconductor.
		As the doped semiconductor has abundance of majority carriers, the limiting factor will be the concentration of "minority" carriers on the other side of the interface, which will depend on its mobility and on the electric field in the intrinsic.
		So the reaction order should be 1 for this case.
		The presence of mid-gap states in the doped selective contact can mediate the recombination in a fashion similar to the aforementioned Shockley-Read-Hall mechanism.
		In efficient perovskite solar cells, this is the most important recombination pathway CITATION NEEDED.

		\paragraph{Auger recombination}
		When a free charge gets in contact with another of the same kind, it is possible that one of these decays and the other absorbs the just released energy as kinetic energy.
		Additionally to the two free charges of the same kind, it also requires a free charge of the opposite kind for the decay to be possible, so it can be considered as having a reaction order of three.
		This recombination is important in materials with high free carriers densities or at high photo-generation conditions.
		These high charge densities are unlikely to happen in perovskite solar cells at normal working conditions.

	\subsection{Charge Generation}

		\paragraph{Light absorption}

		\paragraph{Light absorption in presence of electric field}\label{intro_electroabsorbance}

		\paragraph{Excitons and free charges generation}

	\subsection{PhotoVoltaic Effect}

		\paragraph{Charge collection}

		\paragraph{Electromotive force}

		%https://en.wikipedia.org/wiki/Electromotive_force#Solar_cell

\section{Perovskite Solar Cells}
	\epigraph{\textit{"That's yogurt science!"}}

	Perovskite solar cells is a relatively new player in the photovoltaics world, existing just since 2009 (started by the research groups Miyasaka \cite{Kojima2009}, Park \cite{Im2011a,Kim2012b}, and Snaith \cite{Lee2012}).
	This type of solar cells evolved from the \gls{dssc} tradition, where an absorber dye is in contact with a titania \gls{etm} and an organic \gls{htm}.
	Here the absorber dye is substituted by a hybrid organic-inorganic lead halide perovskite semiconductor and both the \gls{etm} and the \gls{htm} can be composed of various, either organic or inorganic, materials.
	Thanks to the enormous research effort, these kind of devices reached a record stabilized efficiency of 20.9~\% \cite{Green2019} with reports of even higher non-stabilized efficiencies (23.7~\% \cite{Green2019,Jiang2017}); all this in relatively short time as represented in \cref{fig:nrel_chart}.




	\begin{SCfigure}
		\centering
		\includegraphics[width=0.5\textwidth]{nrel_chart/pv-efficiencies-2019-01-03.pdf}
		\mycaption[NREL chart of "Best Research-Cell Efficiencies"]{Just the perovskite solar cells (not stabilized) results have been reported.}\label{fig:nrel_chart}
	\end{SCfigure}

	\paragraph{Composition}

	Cesium introduction \cite{Bi2016,Saliba2016}



	\paragraph{Perovskite Absorber Synthesis}
	The preparation of this hybrid semiconductor \textsl{via} spin coating and annealing at low temperature is extremely easy and convenient for small-scale research purposes.
	%	From my point of view, 
	Nevertheless, this fabrication process is affected by a low reproducibility, likely due to the sensibility of the drying and crystallization step \cite{Pockett2015}.
	This factor clearly slows down the research in perovskite solar cells.
	In literature various reports of completely different results obtained from supposedly identical synthesis can be found \cite{Pockett2015,Gottesman2014}.
	This issue is being addressed recently with the publication of more reliable and complete fabrication procedures \cite{Saliba2018}.

	\paragraph{Recombination mechanisms}\label{intro_prv_recombination}
	The non-radiative recombination inside the perovskite material is surprisingly low for a low temperature processed material.
	This is demonstrated by the large diffusion length in isolated crystals \cite{Wehrenfennig2014,Wehrenfennig2014a,Stranks2013,Xing2013,Shi2015a,Eperon2014}: for the \gls{csfamapbibr} it is reported being \SI{\approx140}{\nm} for the electrons and \SI{\approx1.9}{\um} for holes \cite{Liu2017}.
	This is confirmed by the fact that once a thin layer (\SI{\approx500}{\nm}) of the material is layered between an \gls{etm} and an \gls{htm} the photoluminescence lifetime is dramatically reduced \cite{Jimenez-Lopez2017,Eperon2014} indicating that the free charges can diffuse at least over the layer thickness.
	Even if not always the case \cite{Valadez-Villalobos2019,Tress2018}, it is often reported that at open circuit conditions, the predominant recombination pathway in perovskite solar cells is identifiable as the surface recombination at the perovskite/selective contacts interfaces \cite{Calado2018b,Stolterfoht2018a,Stolterfoht2018,Gelmetti2019,Shao2016,Correa-Baena2017,Hou2016}.
	%	Considering the perovskite material, traps are more likely to form on the surface of crystal domains, where the bulk symmetry is broken and carrier-phonon coupling, which can ease indirect transitions is more likely due to the easier deformation of the broken structure . 
	Nevertheless, also the morphology and composition of perovskite material can have an impact on the \gls{voc}: non-passivated perovskite surface states which can act as traps \cite{Zheng2017} thanks to the easier deformation of the broken crystal cell, favouring the electron-phonon coupling needed for indirect transitions \cite{Wu2015}; inhomogeneous perovskite layer can lead to pinholes \cite{Lee2015,Montcada2017,Qiu2016}; accessible grain boundaries due to non-compact perovskite layer causes the presence of more recombination centres \cite{Shao2016a}; the presence of secondary phases with in-gap energies can be reduced modifying the perovskite composition \cite{Bi2016}.

	\paragraph{Open circuit voltage}
	The open circuit voltage of a perovskite solar cell depends on many contributions.
	As we will see in \cite{ch:characterization}, the aforementioned charge recombination rate is one of these.
	This recombination is more important in a perovskite solar cell rather than in \gls{osc} as the built-in field is shielded by the ionic accumulation and the minority carriers concentration at the perovskite\-/contacts interfaces is higher.
	A contact material having mid-gap states will favour the surface recombination, but also its additives (\textsl{e.g.} dopants, dispersants\dots) can introduce such trap states \cite{Correa-Baena2017}.
	Additionally, a different material employed as \gls{etm} and \gls{htm} will result in a different built-in voltage (difference in contacts' Fermi levels energies), which is often regarded as a limit for the the open circuit voltage \cite{Gelmetti2019,Wu2016}.
	The presence of ionic accumulation in perovskite solar cells have been reported to reduce the influence of the \gls{voc} from the built-in voltage \cite{Belisle2016}.
	Even a different molecular ordering in the selective contacts can influence the \gls{voc} through the variation of the \gls{dos} width, and by consequence the energetics of the band edges \cite{Shao2016}.
	Rather than a hard limit, the built-in voltage represents the applied voltage that saturates the depletion layers in the contacts, which implies a large charge density at the materials interface and an important surface recombination CITATION NEEDED.
	Indeed, it has been shown that suppressing the surface recombination can allow a device to give a voltage higher than the built-in voltage. CITATION NEEDED
	A large absorber band gap will have the excited charges thermally relaxing to a more energetic band, allowing the solar cell to achieve a higher \gls{voc}.
	The perovskite band gap can be easily tuned changing its stoichiometry, for example partially replacing iodine with bromine \cite{McMeekin2016,Noh2013a,Wheeler2017} or using different organic cations \cite{Eperon2014}.
	For perovskite solar cells, the record reported \textit{bandgap\hyp{}voltage offset} are already much smaller than the voltage losses reported for \gls{osc} \cite{Tvingstedt2015}.
	If we consider the possibility of having the injection of hot carriers (free charges generated by a photon with energy larger than the band gap) into the contacts before their thermal relaxation to the band edge, also a \gls{voc} larger than the absorber band gap is possible CITATION NEEDED and could, at least theoretically, make possible to break the Shockley\hyp{}Queisser limit \cite{WikipediaSQlimit}.

	\paragraph{Charge extraction and charge blocking}
	Selective contacts are in charge of selectively extract one kind of charge, blocking the other.
	Clearly, the relative position of the perovskite and selective contact energy level is key both for extraction \cite{CorreaBaena2015} and for blockage CITE SOMETHING.

	\paragraph{Ionic migration}
	Ionic conductivity in halide perovskite materials has been pointed out more than 35 years \cite{Mizusaki1983,Yamada1995,Yamada1998}.
	More recently, the presence of mobile ionic species has been confirmed also for hybrid lead halide perovskites, using \gls{kpfm} \cite{Birkhold2018},  XXXXXXXXXXXXXXXXXX
	%	The presence of and mobile ionic species have been demonstrated in various kind of hybrid lead halide perovskite materials.
	In complete perovskite solar cells, it has been observed even for devices not showing any current-voltage hysteretic behaviour \cite{Calado2016,Jacobs2018,Bryant2015}.
	For \gls{mapi} perovskite, the majoritarian mobile species has been identified as iodine vacancy XXXXXXXXXXXXXXXX \cite{Yang2015e,Senocrate2017} while a significant contribution from methylammonium ions migration has been excluded \cite{Senocrate2018,Senocrate2017}.
	These charged species can migrate inside the material either by diffusion or when drifted by an electric field \cite{Xiao2015}.

	\paragraph{Field free absorber}
	Thanks to the abundance of the ionic specie CITATION, the electric field inside the absorber is completely shielded \cite{Tress2015}.

	diffusive transport vs drift transport

	characterization complexity reduction due to more homogeneous carriers profile \cite{Kirchartz2012}

	\paragraph{Diffusion as the main transport mechanism}(due to the built-in voltage being the workfunction difference of the selective contacts)(drift is not the main transport mechanism in stabilized perovskite solar cells due to the electric screening effect of the mobile ionic species)


	\paragraph{Hysteresis}
	Hysteretic behaviour in current-voltage sweeps, while not often observed in other kind of solar cells, it is frequent in perovskite solar cells.
	Still, many reports of hysteresis-free perovskite solar cells can be found, especially for top cathode or for high performances bottom cathode devices.
	As we saw above, the presence of mobile ionic species is widely accepted for most of the hybrid lead perovskite materials and this has been indicated as one of the causes of the hysteresis \cite{Unger2014,Xiao2015}.
	Considering that the recombination and collection of free charges is affected by the ions-modulated electric field \cite{Pockett2017}, this is enough to explain the presence of hysteresis \cite{Tress2015,Calado2016} \textsl{via} the modulation of interfacial energy barriers \cite{Moia2019}.
	Additionally, adsorption or chemical reaction of perovskite ionic defects with the selective contact surface (\textsl{e.g.}\ with titanium oxide \cite{Yu2016,Beilsten-Edmands2015,Carrillo2016} or tin oxide CITATION NEEDED) could be important and can explain why \cite{Moia2019}, bottom cathode cells with inorganic contacts usually present hysteresis while top cathode with all-organic contacts usually does not.

	\paragraph{Intrinsic or doped semiconductor}


	\paragraph{Characterization}
	The characterization techniques for perovskite solar cells and the relative interpretation evolved from the techniques and theory built around \gls{osc} and \gls{dssc} \cite{Barnes2013}.
	This pre-existing framework has been widely used in literature for perovskite solar cells by various research groups \cite{ORegan2015b,Shao2016,Gelmetti2019,Kiermasch2018,Carnie2015}.
	Unfortunately, perovskite solar cells are different enough from \gls{osc} and \gls{dssc} to doom the utility of most of these observations.
	Moreover, while an acceptable agreement between different characterization techniques was often observed in \gls{osc} \cite{Clarke2015,Maurano2011,Foertig2012} and \gls{dssc} \cite{Barnes2013}, in perovskite solar cells we often observe hard-to-explain discrepancies \cite{Kiermasch2018}.
	In the past few years, the theoretical framework has been expanded and should finally enable the perovskite community to re-interpret and re-design the solar cells characterization techniques.
	In \cref{ch:characterization} the reader can find some of the accepted concepts and some novel proposals about the interpretation of the classical characterization techniques when used on perovskite solar cells.

	\paragraph{Stability}
	The stability of the perovskite solar cells is the number one blocker to commercialization.
	Its keystone has still to be identified.
	An interesting investigation is being carried on by D.\ R.\ Ceratti and D.\ Cahen (unpublished) regarding the release of a proton by methylammonium cation when in contact with non-acidic materials, and thus leaving the perovskite crystal structure.
	This could explain the reports about the beneficial addition of acids in perovskite precursors solution, mainly by Snaith group \cite{Noel2017,Zhang2015a,Nayak2016}.


	https://www.ossila.com/pages/perovskite-solar-cell-degradation-causes


	Formation of HI in contact with water which reacts with silver 10.1039/C5TA00358J 10.1002/admi.201500195
	Reaction with other contact materials 10.1021/acsnano.5b03687

	Influence of perovskite material on Voc \cite{Wheeler2017,Eperon2014,Noh2013a}

	Influence of contacts on Voc \cite{CorreaBaena2015} WU2016A

\section{Background and Related Work}\label{sec:background}

	\subsection{Perovskite Solar Cells}

	\subsection{Hole Transporting Materials}

	\subsection{PhotoPhysical Studies and Techniques}

	\subsection{Modelling and Relevant Physics}

\section{Motivation and Aims}\label{sec:aims}
\epigraph{\textit{"AAAAAAAAAAAAAAAAAAAAAAAA"}}

AAAAAAAAAAAAAAAAAAAAAAAA


\chapter{Experimental Methods}\label{ch:methods}
\chaptermark{Experimental Methods}
	\graphicspath{ {./contents_img/methods/} }
	\section{Top Cathode Perovskite Solar Cells One Step Fabrication}

\section{Top Cathode Perovskite Solar Cells Two Step Fabrication}

\section{Bottom Cathode Perovskite Solar Cells Fabrication}

\section{Solar Cells Handling and Routinary Characterization}

	\subsection{Current-Voltage Curve}

\section{Advanced Characterization}

	\subsection{Transient PhotoVoltage}
	
	\subsection{Charge Extraction}
	
	\subsection{Transient PhotoCurrent}
	
	\subsection{Differential Capacitance}
	
\section{Data Analysis and Handling}





\chapter{Characterization Techniques: Description and Interpretation}\label{ch:interpretation}
\chaptermark{Characterization Techniques}
	\graphicspath{ {./contents_img/characterization/} }
	\epigraph{\textit{""}}

\paragraph{Abstract} Characterization of perovskite solar cells is a non-trivial subject, the techniques researchers successfully employed for \gls{osc} and \gls{dssc} needs to be re-validated for this new kind of solar cells. The presence of ionic migration in the absorber can be a game-changer for which special care has to be taken.

\section{Conventions and General Remarks}

	All the characterization on complete devices was performed keeping them in an air-tight holder filled with nitrogen. The electrical connection from the cell electrode to the external end of the holder was obtained using gold tips connected via a printed circuit board to a coaxial cable.

	\subsection{Sign Convention and Parameters Definitions}

		\paragraph{Fermi level} The electrons electrochemical potential, also known as Fermi level, is defined as the energy required for adding an electron in a specified position. Its value depends on the electrostatic potential $V$ in that position and on the internal chemical potential $\mu$ which in our case depends mainly on the concentration of electrons (not related to their electric charge, similarly to the density of a gas). As the Fermi level is going to be used mainly for comparisons, its zero is not going to be defined thesis-wide, instead it will be defined to a convenient reference just where needed.

		\paragraph{Cathode and anode} Considering a solar cell device at steady state under illumination and in open circuit conditions, its cathode is defined as the contact where the electrons electrochemical potential $\bar\mu$  is the highest. By consequence the other contact is the anode. The naming of the two contacts holds to the one defined in illuminated, open circuit conditions even in conditions where the contacts' electrochemical potential is in the reversed order.

		\paragraph{Voltage} The voltage $\Delta V$ is always used as a relative value, defined subtracting the electrons electrochemical potential of the cathode from the anode's one. So in the aforementioned solar cell example, the voltage is positive. The unit is the Volt.

		\paragraph{Electrical power} The electrical power $P$ is defined as positive when the device absorbs electrical energy (incoming, passive element) and negative when it generates energy (outgoing, active element). It can be expressed in extensive form with power (Watt) unit or in intensive form "electrical power density" with power over active area unit (Watt over square centimetre).

		\paragraph{Current} The current $J=P/V$ is measured through an external circuit and the sign is a consequence of the voltage and electrical power definition: A current ("conventional current", flow of positively charged particles) being released from the device's anode and being received from the device's cathode is defined as negative. This can be thought as: Inside the device, somehow, a positive charge was moved from the high Fermi level contact to the low Fermi level contact, increasing its electrochemical energy, the opposite to what would happen in a resistor, whose current is always positive. In a solar cell device, the current and the electrical power can be either positive or negative depending on the illumination and voltage conditions. It can be expressed in extensive form with current (Amperes) unit or in intensive form "current density" with current (Amperes) per active area (square centimetre) unit.

		\begin{SCfigure}
			\centering
			\includegraphics[width=0.5\textwidth]{iv_params/IV-revIVs.pdf}
			\mycaption[Parameters extraction from current-voltage sweeps.]{A typical current-voltage sweep is represented. MPP stands for maximum power point, \gls{jsc} stands for \glsdesc{jsc}, \gls{voc} stands for \glsdesc{voc}. The ration between the small and the large rectangles areas is the \glsdesc{ff} (\gls{ff}).}\label{fig:iv_params}
		\end{SCfigure}

		\paragraph{\Glsdesc{voc}} \Gls{voc} parameter is defined as the voltage $\Delta V$ at which the current is zero while the solar cell device is illuminated at 1~sun conditions and in steady state (positive by its own definition).

		\paragraph{\Glsdesc{jsc}} \Gls{jsc} parameter is defined as the unsigned value of the current density flowing in an external circuit short circuiting (zero resistance) the solar cell device's contacts while illuminated at 1~sun and in steady state. It is usually reported in current (milli Amperes) over active area (square centimetre) unit.

		\paragraph{Maximum power density} The maximum power density is defined as the unsigned minimum of electrical power density which can be obtained by $P(V) = J(V) \cdot V$. It is usually reported using power (Watt) over active area (square centimetre) unit.

		\paragraph{\Glsdesc{pce}} \Gls{pce} parameter is defined as the maximum power density over the illuminating power density, which at 1~sun AM 1.5G is defined to \SI{100}{\mW\per\square\cm}. It is usually reported as a percentage.

		\paragraph{\Glsdesc{ff}} \Gls{ff} parameter is defined as the ratio between \gls{pce} and the product of \gls{voc} and \gls{jsc}. This parameter does not have a physical meaning, but it represents how much the series and shunt resistances affect the device efficiency.

		\paragraph{Forward and reverse bias} Forward bias is a device condition where the voltage is positive, reverse bias is the case where the voltage is negative.

		\paragraph{Forward and reverse scan} In current-voltage sweeps, a scan where the voltage is increasing over time is a forward scan, while a voltage variation in the opposite direction constitutes a reverse scan.

		\paragraph{Ideality factor} An ideality factor $n_{id}$ different from 1 describes deviations from the ideal photo-diode. The adapted $J(V,\phi)$ equation becomes\cite{Calado2018b}:
		\begin{equation} \label{eq:photodiode}
			J = J_{SC}(\phi) - J_0\left(\exp\left(\frac{qV}{n_{id}k_BT}\right)-1\right)
		\end{equation}
		where $J_0$ is the diode saturation current (the current flowing in dark when applying a reverse bias), $q$ is the elementary charge, $k_B$ is the Boltzmann constant, and $T$ is the temperature.  Clearly the reported equation just offers a simplified model. For example, it can be improved adding the contribution from the series resistance $R_s$ and would become

		$$J = J_{ph}(\phi) - J_0\left(\exp\left(\frac{q(V+JR_s)}{n_{id}k_BT}\right)-1\right)$$

		where now $J_{ph}$ is the total photo-generated current. The function is now an implicit one, requiring numerical solving even for obtaining $J_{SC}$.

		\paragraph{Top and bottom of devices} The point of view of the manufacturer is used for defining the physical top and bottom of a device: the bottom is the glass substrate and the top is the last deposited layer. This is opposite with the usage of a solar cell in the real world and with most of the solar simulators (but not all of them, for example Paios from Fluxim has an illumination from below, more convenient for contacting the electrodes without a samples holder \cite{Fluxim}).

	\subsection{Usage of Shadowing Mask}

		\begin{SCfigure}
			\centering
			\includegraphics[width=0.5\textwidth]{shadowing_mask/shadowing_mask.pdf}
			\mycaption[Illuminated area after a shadowing mask.]{This schema is just for explaining the concept described in the text, its dimensions are not realistic.}\label{fig:shadowing_mask}
		\end{SCfigure}

		In literature is generally suggested to use a shadowing mask when measuring the solar cell devices in order to better define the illuminated area (as in the broken \cite{Brinser2017} form from Nature\cite{NatureResearch2017}). %, the file is a dynamic XFA form and cannot be opened by most PDF readers (and they rightly do not, as it was deprecated in ISO 32000-2:2017), Adobe products or Master PDF Editor is required for reading).
		In our case the active area is just \SI{0.09}{\square\cm} so the mask aperture should be extremely small and its exact positioning troublesome. Additionally, the fact that the illumination reaches the mask from a wide angle (the illuminating source dimension, which is not just the lamp as the illumination passes through spread lenses, is not small compared to the lamp-cell distance) allows the light to spread through the substrate glass (\SI{2.2}{\mm} for \gls{fto} substrates, other groups use even thicker glass substrates) reaching a significantly larger area on the active layer at the other side of the glass, as represented in \cref{fig:shadowing_mask}. In our solar simulator a linear widening of 8~\% over \SI{2}{\mm} was estimated, this makes an illuminated area 16~\% larger than the mask aperture. Even if the total incident power is still determined from the mask aperture, the illumination intensity is not 1~sun any more, compromising the validity of a measurement done with a shadowing mask.


	\subsection{Stability During the Measurement and Small/Large Perturbations}

		Most of the reported hybrid lead halide perovskite materials can show rather impressive changes in their structure on long time scales, for example due to ionic migration \cite{Calado2016}, degradation \cite{OKane2019}, and self-healing \cite{Ceratti2018}.
		This have to be taken into account for all the measurements techniques output which either takes too long time to be measured or employs large perturbations.

		\paragraph{Long lasting measurement} An example of the first case is the impedance spectroscopy where during the long lasting measurement various phenomena can occur, like: a slow current evolution due to perovskite well known hysteretic behaviour prior to stabilization; a degradation process changing the current; the heating of the device changing its properties. This slow current evolution can easily be misinterpreted for capacitive current \cite{Jacobs2018} and introduce artefacts like loops in the Nyquist plots CITE IMPEDANCE PAPER.

		\paragraph{Large perturbations} Large perturbations regime means that the independent variable being perturbed is changed by an amount large enough to cause the quantities under study to not follow the approximation given by the first term of the series expansion. Let's take some example.

		\paragraph{Large perturbations -- \gls{tpv}} For example, if too intense, the light pulse in \acr{tpv} could change the voltage by a less-than-linear amount. In this case, the light pulse is not only probing the recombination, but it is adding some, so a large perturbation has to be avoided.

		\paragraph{Large perturbations -- impedance} Another example: a too wide sinusoidal voltage oscillation amplitude in impedance measurements can cause a non-sinusoidal current output. This is not a problem for the measurement itself, as the lock-in amplifiers are perfectly able to extract the amplitude and the phase of the signal first harmonic, ignoring the higher harmonics caused by the too large perturbation. But artefacts could arise and cause misinterpretations, as explained in \cpageref{impedance-large_perturbations}.

		\paragraph{Large perturbations -- \gls{trpl}} Last example: in the \glsdesc{trpl} a laser pulse illuminates the otherwise unilluminated absorber layer. This pulse induces a the migration of the ionic defects to a new profile depending on the pulse intensity \cite{Levine2018}, by a small extent due to its short duration. The fact that the relaxation time of the ionic migration is usually much larger than the laser repetition rate (\si{\ms} to seconds for the ions \cite{Jacobs2018} and \si{\ms} to \si{\us} for typical \gls{trpl} lasers \cite{EdinburghInstruments}) implies that the ionic profile variation slowly "builds up" pulse after pulse. As the ionic profile affects the free charges concentration, and this in turn rules the radiative recombination, the measurements of \gls{trpl} have to be done with extreme care. An example of hysteretic behaviour observed with \gls{trpl} can be found in \authoryear{Motti2016}.


\section{Current-Voltage Sweeps}

	After calibrating the light intensity in the solar simulator (see \cpageref{solarsimulator}), the devices were exposed to the illumination at open circuit for some seconds in order to have a stabilized open circuit voltage. Then usually the curves were measured with the auto-measure function of the PyPV software (see \cpageref{automeasure}) which measures the reverse scan and then the forward scan.

	\paragraph{Parameters Extraction from Sweeps}
	For the devices studied in this thesis, the reported values of \gls{voc}, \gls{jsc}, \gls{pce} and \gls{ff} are extracted from a forward or reverse current-voltage sweep. This is in accordion to the tradition of solar cells reporting but for hysteretic devices, like perovskite solar cells, a static measurement should be preferred. %Checking the aforementioned definitions, one can easily notice how this is not the correct way as these parameters need to be measured in steady-state conditions. This is comes from the tradition of solar cell reporting, which was established in pre-hysteresis times and is no longer a valid approximation. Apologizing for the lack of coherence I hope I'll be able to implement the static measurements of \gls{voc} and \gls{jsc} in the PyPV measurement routines (described in \cpageref{automeasure}).

	\paragraph{Parameters Extraction from Sweeps -- \gls{pce}} Regarding the \gls{pce}, and by consequence the \gls{ff}, a proper measurement is made difficult due to the cell evolution over time (hysteresis). The voltage associated to the maximum power point is drifting and its localization affects its evolution. A proper \gls{mppt} system needs to be bought or developed, see \cpageref{software_mppt} for thoughts about possible implementations.

	\paragraph{Auto-scale}\label{autoscale} In literature one can easily find current-voltage curves with discontinuities or "kinks" \cite{Li2016,Snaith2014,Zhang2015} like the one reported in \cref{fig:autoscale}. Some even lucubrate about the origin of these in perovskite solar cells. Indeed this is likely just caused by the auto-scale feature of the Keithley equipment, disabling this, the discontinuities disappears.

	\begin{SCfigure}%[!hbtp]%
		\centering
		\includegraphics[width=0.5\textwidth]{autoscale/ig1-3-1-int4.pdf}
		\mycaption[Kinks in JV sweep due to autoscale.]{A current-voltage sweep of an hysteretic perovskite solar cell with Keithley autoscale active. Both the forward (dashed) and the reverse (solid line) present small discontinuities around \SI{1}{\mA} and \SI{0.1}{\mA}.}\label{fig:autoscale}
	\end{SCfigure}

	\paragraph{Scan speed} The used sweep speed is \SI{500}{\mV\per\s}, which was arbitrarily chosen for avoiding bumps leading to currents higher than \gls{jsc}, like the one in \cref{fig:iv_ugly}. %having an aesthetically good looking current-voltage curve
	Our arbitrary choice allowed us to make fair comparisons between devices, but the absolute values should be considered as approximations.
	\begin{SCfigure}%[!hbtp]%
		\centering
		\includegraphics[width=0.5\textwidth]{iv_ugly/ig47-b32-int2-4.pdf}
		\mycaption[Hysteretic current-voltage scan.]{At the employed scan speed, the hysteresis phenomena causes the reverse (solid line) scan to reach currents higher than the \gls{jsc}.}\label{fig:iv_ugly}
	\end{SCfigure}
	We wanted to underline that due to hysteresis phenomena, no scan speed, direction, or precondition is the correct one.  % and they all heavily affect the resulting \gls{pce}.
	Rather, a static measurement or a \acr{mppt} should be used for obtaining a accurate and realistic result.
	This comment regards also the so-called "hysteresis-free" perovskite solar cells, which can also have hysteretic phenomena\cite{Jacobs2018}.

	\paragraph{Noise} The noise often observed in current-voltage sweeps at high scan speeds in this thesis is mainly caused by oscillations in the solar simulator illumination intensity, as an example see \cref{fig:iv_params}. For reducing the noise impact on the \gls{jsc} and \gls{voc} parameters extraction, these values were extracted via a parabolic fitting. %Indeed, using a white \gls{led} as illumination source, this noise is not present but the spectral mismatch affects the meaningfulness of the measurement.

	\paragraph{Stabilized or dynamic current-voltage sweeps} One very appealing alternative to current-voltage sweeps are the so-called "stabilized current-voltage sweeps", where at each voltage point a fixed stabilization time is waited and the stabilised current is reported \cite{Unger2014, Christoforo2015}.
	An improvement of this technique is named "dynamic current-voltage sweeps", here the stabilization step is of variable duration, until the current-time derivative falls below a threshold (e.g. \SI{0.2}{\%\per\minute}) \cite{Dunbar2017,Dunbar2017a}.
	In this thesis, these techniques have not been used.

	\paragraph{Shunt and series resistances} \label{resistances} In our group the shunt and series resistances are evaluated by the current versus voltage derivative of a dark current-voltage sweep respectively at zero and at high-enough voltage. This methodology is inherited from organic solar cells and, as is easy to foresee, is unreliable on hysteretic devices: in the case of perovskite solar cells a measurement of the stabilised current at a few points have better chances to produce a useful result. The measurement of the current at a voltage close to zero is enough for estimating the shunt resistance while two points at high voltages are needed for estimating the series resistance.

\section{V\textsubscript{OC} and J\textsubscript{SC} Dependence on Light Intensity}
	The solar simulator illumination intensity $\phi$ is reduced via neutral density filters with transmittance of 0.05, 0.12, 0.25, 0.51, 0.81 and 1 (no filter). The values of $V_{OC}(\phi)$ and $J_{SC}(\phi)$ can be obtained from static measurements or from current-voltage sweeps. The static measurement of $V_{OC}(\phi)$ at high light intensities is troublesome as it can easily damage the device. In this thesis, the used method is specified case by case. %this measurement is reported both from static measurement of \gls{voc} and from \gls{voc} obtained from current-voltage sweeps; the used method is specified case by case. %The devices are kept at this reduced illumination $\phi$ and at open circuit or short circuit conditions until steady state for measuring respectively 

	\subsection{\Gls{jsc} versus $\phi$}
		The \gls{jsc} dependency on the light intensity $\phi$ is close to linear and can be fitted with a power law:
		\begin{equation} \label{eq:jsc-phi}
			J_{SC} \propto \phi^\alpha
		\end{equation}
		giving $\alpha$ values usually from 0.95 to 1.

		\paragraph{Interpretation} %The \gls{jsc} dependency on the light intensity $\phi$ is close to linear and can be fitted with a power law $J_{SC} \propto \phi^\alpha$ as described in \cpageref{methods_jsc_intensity}.
		An $\alpha$ value lower than 1 indicates the presence of non-geminate recombination (for geminate recombination see \cpageref{intro_geminate}) at short circuit\cite{Credgington2011}, indicating that not all the photo-generated charges get extracted, neither at short circuit conditions.

	\subsection{\Gls{voc} versus $\phi$ and the Ideality Factor $m$}
		Setting $J=0$, which corresponds to open circuit conditions, in \cref{eq:photodiode} (without considering the series resistance correction) we can obtain a relation between \gls{jsc} and \gls{voc}:
		%\begin{equation} \label{eq:photodiode-zerocurrent}
		$$J_{SC}(\phi) = J_0\left(\exp\left(\frac{qV_{OC}(\phi)}{n_{id}k_BT}\right)-1\right)$$
		%\end{equation}
		This equation can already be used for obtaining the $n_{id}$ and $J_0$ values fitting the \gls{jsc} and \gls{voc} measured at different light intensities $\phi$ (also varying the temperature could be used for fitting the values, but it was not done during this thesis).

		Solving for \gls{voc} we obtain:
		%\begin{equation} \label{eq:photodiode-zerocurrent-voc}
		$$V_{OC} = \frac{n_{id}k_BT}{q}\cdot\ln\left(\frac{J_{SC}}{J_0} + 1\right)$$
		%\end{equation}

		Considering that the saturation current $J_0$ (current in dark under reverse bias) is much smaller than $J_{SC}$ for the light intensities we usually employ (down to \SI{0.05}{suns}), we can approximate to:
		%\begin{equation} \label{eq:photodiode-zerocurrent-voc_approx}
		$$V_{OC} \approx u_1 + \frac{n_{id}k_BT}{q}\cdot\ln(J_{SC})$$
		%\end{equation}

		where $u_1$ is a useless constant. Then if the $\alpha$ value is close enough to~1, we can use \cref{eq:jsc-phi} and further approximate for plotting against light intensity $\phi$:
\begin{equation}\label{eq:voc_vs_phi}
V_{OC}(\phi) \approx u_2 + \frac{n_{id}k_BT}{q}\cdot\ln(\phi)
\end{equation}

		This is the equation we commonly employ for fitting and obtaining ideality factors \cite{Nelson2003}. In some cases this latest expression is used as ideality factor definition as it conveniently uses a zero current measurement (\gls{voc}) so that series resistance can be completely ignored \cite{Kirchartz2012}. %In literature one can found reports of ideality factor determined via \gls{voc} versus \gls{jsc} dependency. The results of the latter method may differ from the aforementioned one due to series resistance and the non-linear dependency of the \gls{jsc} from the light intensity ($\alpha < 1$). 
		The so-obtained ideality factor $n_{id}$ is usually from 1 to 2. A voltage dependent ideality factor can also be measured from a current-voltage sweeps in dark but has not been evaluated in this thesis, as it would be affected by hysteresis. A critical analysis of these methods and the proposal of a new "Transient Suns-\gls{voc}" method (employing a pre-biassing for flattening the ionic profile) can be found in \authoryear{Calado2018b}. % Anyway in perovskite solar cells exhibiting hysteresis, a current-voltage sweep would not give fair ideality factors and the current-voltage points should be acquired reaching steady state conditions for each point. Both these methods for deriving the ideality factor have been implemented in the DrIFtFUSION simulation as described in \cpageref{dd_ideality}.

		\paragraph{Interpretation} %\The \gls{voc} dependency on the light intensity $\phi$ can be fitted with a natural logarithmic dependence obtaining the ideality factor $m$. 
		\authoryear{Pockett2015} measured ideality factors of planar perovskite solar cells via stabilized \gls{voc} obtaining, for some cases, values as high as 5. Also in organic \cite{Kirchartz2011,Kirchartz2012} and silicon solar cells \cite{Breitenstein2006} ideality factors greater than 2 has been observed and explained. According to \authoryear{Calado2018b} and \authoryear{Kirchartz2012}, the ideality factor, once obtained in the correct way, is 1 when studying most of the recombination types and 2 for mid-gap trap mediated recombination in regions where the electrons and holes concentrations are similar, $n \approx p$.

\section{Charge Extraction (CE)}

	\paragraph{Concept} The charge extraction experiment has been designed\cite{Duffy2000} to quantify the free charges available in a device. After stabilization of a device at a light intensity and an applied voltage (in our case always at \gls{voc}), the illumination is switched off and the electrodes of the device are short circuited. The transient current flowing through the short circuiting circuit can be measured and integrated to estimate the available charge.
	
	\paragraph{Limitations -- from \gls{osc}}
	
	\paragraph{Limitations -- specific of perovskite} \label{ce_limitations_perovskite}When measuring the \gls{ce} of a perovsktie solar cell, additionally to the aforementioned limitations, one should also consider that the ionic profile update (from $V=V_{OC}$ to $V=0$) causes a displacement current, as described in \cpageref{intro_displacement_current}. A simulation with DrIFtFUSION is reported for an homojunction device in \cref{fig:ce_single_dd}, it can be seen that with ionic mobile defects in \cref{fig:ce_single_dd-ions_zoom} a very weak but long lasting current appears and gives a relevant contribution to the integrated charge. This will happen on very large time scales and it will not affect the short measurements used for the free charges estimation, so it is rarely reported\cite{ORegan2015b}. The charge measured in the external circuit due to the ionic displacement current could be underestimated as the free charges rearrangements can also occur through the perovskite layer rather than through the external circuit.
	
\begin{figure}
	\makebox[\textwidth][c]{
		\parbox{1.1\textwidth}{
			\centering
			\begin{subfigure}[t]{0.5\textwidth}
				\includegraphics[width=1\textwidth]{ce_single_dd/ce_single_dd-noions.pdf}
				\subcaption{Without mobile ions}\label{fig:ce_single_dd-noions}
			\end{subfigure}
			\bigskip
			
			\begin{subfigure}[t]{0.5\textwidth}
				\includegraphics[width=1\textwidth]{ce_single_dd/ce_single_dd-ions.pdf}
				\subcaption{With mobile ions}\label{fig:ce_single_dd-ions}
			\end{subfigure}
			\qquad			
			\begin{subfigure}[t]{0.5\textwidth}
				\includegraphics[width=1\textwidth]{ce_single_dd/ce_single_dd-ions_zoom.pdf}
				\subcaption{With mobile ions, magnified}\label{fig:ce_single_dd-ions_zoom}
			\end{subfigure}
		}
	}
	\mycaption[Simulation of a CE without or with mobile ions.]{The current versus time profile of a \acr{ce} simulated experiment is shown on the left axis just for the 1~sun illumination intensity. On the right axis the cumulative integration of the extracted charge. Clearly, the extraction is unrealistically quick as the resistance included in a real \acr{ce} experiment was not included in the simulation. In (\textbf{a}) the measurement of a device without mobile ions is shown, we can observe just one current peak contributing to the integrated charge; in (\textbf{b}) the presence of the mobile ions introduces a long-times contribution to the extracted charge, the current causing this is very weak and long lasting, it can be better observed in the magnification in (\textbf{c}).}\label{fig:ce_single_dd}
\end{figure}
	
	%\paragraph{Interpretation of the single measurement}
	%From some preliminary and unpublished simulations of \acr{ce} show that a short living exponential decay can be accounted for free charges and a long living and weak exponential decay is caused by a displacement current due to ionic profile updating to the new voltage. The slow decay is not usually measured and seldom reported\cite{ORegan2015b}.
	
	\paragraph{Procedure} The device is kept under 1~sun equivalent illumination by a white \gls{led} at open circuit conditions until stabilization is reached. 1~sun equivalent illumination is defined as the illumination at which the a silicon photodiode gives the same \gls{jsc} as under calibrated 1~sun from the solar simulator. The \gls{led}-solar spectral mismatch affects slightly the measurement, but in no case a \gls{pce} is reported from any \gls{led}-illuminated experiment. After stabilization the illumination is switched off and, at the exactly same moment, the device is short circuited through a small and known resistance of \SI{50}{\ohm}.
	This is repeated decreasing the light intensity from 1~sun down to dark (in dark no signal should be observed, indeed some residual charge can usually be seen, the reason of this could be ionic profile updating or an insufficient darkness) and a single decay is measured for each illumination point, over approximatively 30 illumination points.
	The equipment includes two transistors (in a home made circuit by Dr. Javier Pérez Hernández) connected to a pulse generator providing a square pulse long at least as the measurement window. From my experience, I recommend to use a short dark period in order to save time for the following stabilization step.
	The measurement is carried out with an oscilloscope in parallel to the known small resistance. In the first microseconds, most of the free charge flows through the resistor generating a voltage drop across it which is measured by the oscilloscope. This potential drop can be converted to current using the Ohm's law, which, integrated over time, gives the amount of extracted charge.

	\paragraph{Noise reduction}\label{r_ce_noise} Most of the observed short-time noise (\SI{< 5E-7}{\s}) observable in \cref{fig:ce_noise-normal} is related to the opening and closing of the transistor switches included in the home-made circuit. The characteristic frequencies of the observed noise are not small compared to the measurement window, so its time-integral does not necessarily sum to zero.
	In order to reduce its impact, various approaches have been tested and here described.
	The noise can be ignored in some cases, but it's a problem if the charge versus light bias (\gls{voc} generated at a given illumination intensity) profile, reported on the right hand column of \cref{fig:ce_noise}, has to be studied in detail, as in \cref{ch:tae}.
	In the mentioned case, just the exponential part of its linear plus exponential behaviour, reported in the bottom of each right hand figure, was of interest and it is evident that the employed noise-reduction method influences it heavily.
	
	\paragraph{Noise reduction -- subtraction of dark noise} Annoyingly, the noise profile is characteristic of the cell and of the circuitry, so a simple average over many decays does not help in cancelling it. Based on this consideration, I tried to subtract a pure noise profile as obtained from a dark measurement (without any light bias). The operation was made more difficult by the slight variations of the noise profile with the light bias. A data-to-morphed-noise fit was implemented where the $f(t)$ noise profile was transformed with:$t'= A + B \cdot t + C \cdot t^2$ and $f'(t') = D \cdot f(t') + E \cdot t' \cdot f(t') + F \cdot e^{-t'/G}$ where A, B, C, D, E, F, and G are restrained fit variables. Then F was set to zero and the resulting profile was subtracted from the data and the result integrated. As can be seen in \cref{fig:ce_noise-subtractDark}, this technique is working for most of the cases but it can fail if the noise profile changes in a more complex fashion.
	
	
	\paragraph{Noise reduction -- integration of a fitting} Finally the decays were fitted with a bi-exponential formula (sum of two exponential) or, if the bi-exponential fitting was not converging, by a simple exponential and the integral of this fit was used. In both cases a robust fitting routine was employed \cite{Maechler2018}.
	
	\begin{figure}
		\makebox[\textwidth][c]{
			\parbox{1.1\textwidth}{
				\centering
%				\begin{subfigure}[t]{0.55\textwidth}
%					\includegraphics[width=1\textwidth]{{ce_noise/normal-CE_ig104-1566-4_Voc_0.911mV-crop}.pdf}
%					\subcaption{Direct integration}\label{fig:ce_noise-normal}
%				\end{subfigure}
%				\qquad
%				\begin{subfigure}[t]{0.35\textwidth}
%					%\includegraphics[width=1\textwidth]{ce_noise/normal-spiro_vs_TAEs-TPVCEs_nogeom-crop.pdf}
%					\includegraphics[width=1\textwidth]{ce_noise/normal-spiro_vs_TAEs-CEs-crop.pdf}
%					\subcaption{Results from direct integration}\label{fig:ce_noise-normal_ce}
%				\end{subfigure}
%				\bigskip
%				
%				\begin{subfigure}[t]{0.55\textwidth}
%					\includegraphics[width=1\textwidth]{{ce_noise/subtractDark-CE_ig104-1566-4_Voc_0.911mV-crop}.pdf}
%					\subcaption{Subtraction of noise}\label{fig:ce_noise-subtractDark}
%				\end{subfigure}
%				\qquad
%				\begin{subfigure}[t]{0.35\textwidth}
%					%\includegraphics[width=1\textwidth]{ce_noise/subtractDark-spiro_vs_TAEs-TPVCEs_nogeom-crop.pdf}
%					\includegraphics[width=1\textwidth]{ce_noise/subtractDark-spiro_vs_TAEs-CEs-crop.pdf}
%					\subcaption{Results from subtraction of noise}\label{fig:ce_noise-subtractDark_ce}
%				\end{subfigure}
%				\bigskip
%				
%				\begin{subfigure}[t]{0.55\textwidth}
%					\includegraphics[width=1\textwidth]{{ce_noise/integrateExp-CE_ig104-1566-4_Voc_0.911mV-crop}.pdf}
%					\subcaption{Integration of fitting}\label{fig:ce_noise-integrateExp}
%				\end{subfigure}
%				\qquad
%				\begin{subfigure}[t]{0.35\textwidth}
%					%\includegraphics[width=1\textwidth]{ce_noise/integrateExp-spiro_vs_TAEs-TPVCEs_nogeom-crop.pdf}
%					\includegraphics[width=1\textwidth]{ce_noise/integrateExp-spiro_vs_TAEs-CEs-crop.pdf}
%					\subcaption{Results from integration of fitting}\label{fig:ce_noise-integrateExp_ce}
%				\end{subfigure}
				\begin{subfigure}[t]{1\textwidth}
					\includegraphics[width=0.45\textwidth]{{ce_noise/normal-CE_ig104-1566-4_Voc_0.911mV-crop}.pdf}	
					\qquad
					\includegraphics[width=0.35\textwidth]{ce_noise/normal-spiro_vs_TAEs-CEs-crop.pdf}
					\subcaption{Direct integration of raw data}\label{fig:ce_noise-normal}
				\end{subfigure}
				\bigskip
				
				\begin{subfigure}[t]{1\textwidth}
					\includegraphics[width=0.45\textwidth]{{ce_noise/subtractDark-CE_ig104-1566-4_Voc_0.911mV-crop}.pdf}
					\qquad
					\includegraphics[width=0.35\textwidth]{ce_noise/subtractDark-spiro_vs_TAEs-CEs-crop.pdf}
					\subcaption{Subtraction of morphed noise profile}\label{fig:ce_noise-subtractDark}
				\end{subfigure}
				\bigskip
				
				\begin{subfigure}[t]{1\textwidth}
					\includegraphics[width=0.45\textwidth]{{ce_noise/integrateExp-CE_ig104-1566-4_Voc_0.911mV-crop}.pdf}
					\qquad
					\includegraphics[width=0.35\textwidth]{ce_noise/integrateExp-spiro_vs_TAEs-CEs-crop.pdf}
					\subcaption{Integration of an exponential fitting}\label{fig:ce_noise-integrateExp}
				\end{subfigure}
			}
		}
		\mycaption[Strategies of noise reduction for \gls{ce} integration.]{On the left, a single \gls{ce} decay from a \gls{fto}\-/\dTiOtwo\-/\mpTiOtwo\-/\acr{csfamapbibr}\-/\tae3\-/Au is integrated without noise reduction in (\textbf{a}), adapting the noise profile from the dark measurement and subtracting it in (\textbf{b}), or fitting the decay and integrating the fit in (\textbf{c}). On the right, the charge versus voltage trends obtained applying the respective noise reduction methods.}\label{fig:ce_noise}
	\end{figure}
	
	

\section{Interpretation of Charge Extraction}\label{interpretation_ce}

\paragraph{Charge extracted} The integrated charge is assumed to include the excess free charges in the valence and conduction bands. With \emph{excess} we refer to the difference between the charge concentration in the conditions of interest and the stabilized dark condition. For a non perfectly crystalline material, localized shallow traps constituted by the tails of the valence and conduction bands density of states inside of the so-called mobility gap \cite{Pieters2009} are not negligible and should also contribute to the extracted charge amount\cite{Kirchartz2012}. On the contrary, charges trapped in deep traps contributing to SRH trap mediated recombination, with energies far from the band edges, should not be possible to extract in a \acr{ce} experiment. 


\paragraph{\Acr{ce} time constant}
The free charges extraction time is related to the RC time of the \SI{50}{\ohm} resistor and the capacitance of the solar cell device. We can see in \cref{fig:chargeExtraction_RCtime} a weak covariance (Pearson correlation coefficient of 0.3) between the RC time obtained extrapolating the dark capacitance from \acr{dc} (which is the geometric capacitance) and the extraction time (as obtained by an exponential fitting to a single \acr{ce} current decay) at low light intensity (enough for having a signal but far from 1~sun light intensity). At higher light intensities, the correlation is weaker as the capacitance is less defined as the cell is in a transition between illuminated (high capacitance) and dark (low capacitance) status. Anyway, the extraction time does not change much between low light intensity and 1~sun with an increase from \SIrange{1.1}{2.4}{times} (first and third quartile). More discussion on this topic can be found on \authoryear{Montcada2018}.

\begin{SCfigure}%[!hbtp]%
	\centering
	\includegraphics[width=0.45\textwidth]{chargeExtraction_RCtime/CEaBitOfSunExpTime_vs_RCdarkTime.pdf}
	\mycaption[Charge extraction time is related to a RC time.]{Covariance of \acr{ce} extraction time at low light intensity versus the expected time from geometric capacitance (as obtained from dark \acr{dc}). Each point is a different device for a total of 78 devices, many different structures studied during my PhD are represented. The green line indicates the 1 to 1 relationship.}\label{fig:chargeExtraction_RCtime}
\end{SCfigure}

\paragraph{\Acr{ce} time constant and \acr{tpv} time constant -- Corrections} 
During this time, and depending on its location in the device stack, some free charge can recombine. One could argue that a \acr{ce} measurement is valid only if the extraction is faster than the recombination time as measured via \acr{tpv} \cite{Ryan2017a} or that the extracted charge should be corrected considering the recombination \cite{Credgington2011}. Considering the charges accumulated in the depletion layers in the selective contacts, these will flow to the electrodes without crossing the perovskite/selective contacts interfaces, where has been reported that most of the recombination occurs \cite{Barnea-Nehoshtan2014,Stolterfoht2018a,Stolterfoht2018}. So this part of the extracted charge, distinguishable as the linear part of the charge versus voltage plot, as represented on the right column of \cref{fig:ce_noise} should not be corrected. Instead, regarding the charge accumulating in the perovskite layer, which we assume can be assigned to a chemical capacitance and can be recognized as the exponential part on the right column of \cref{fig:ce_noise}, it may be that a correction \cite{Shuttle2008a,Shuttle2008b} is needed, but this has not be done in this thesis.

\paragraph{\Acr{ce} time constant and \acr{tpv} time constant -- Correlation?} Some covariance (Pearson correlation coefficient of~0.4) can be observed in \cref{fig:ce_1sun_time_vs_tpv_1sun_time} between the \acr{ce} and the \acr{tpv} time constants at 1 sun illumination. This is unexpected and weird as the two times change with very different trends with light bias (when changing the preconditioning light intensity, \gls{ce} extraction time changes just slightly while \gls{tpv} decay time varies over various orders of magnitude). In case a stronger proof of correlation is found, this could indicate that both processes, even if not of the same nature, are limited by the same diffusion process, for example the migration of free charges from all the absorber to the absorber/contacts interfaces.

\begin{SCfigure}%[!hbtp]%
	\centering
	\includegraphics[width=0.45\textwidth]{ce_1sun_time_vs_tpv_1sun_time/ce_1sun_time_vs_tpv_1sun_time.pdf}
	\mycaption[Comparison between \gls{ce} and \gls{tpv} exponential decay times.]{Covariance of \acr{ce} extraction time at 1~sun light intensity versus the \gls{tpv} mono-exponential decay time at 1~sun light intensity. Each point is a different device for a total of 79 devices, including many different structures. The green line indicates the 1 to 1 relationship.}\label{fig:ce_1sun_time_vs_tpv_1sun_time}
\end{SCfigure}

\paragraph{Charge versus light bias trend - Exponential part}
In \gls{osc} literature the charge versus light bias voltage trend is described simply as the exponential shape which describes a Maxwell--Boltzmann distribution for a two levels scenario. For a common solar cell working conditions, the Maxwell--Boltzmann classical particles approximation should be valid as the distance between Fermi level energy and the band edges is expected to be always much bigger than $k_BT$. This could be false for high applied voltages, where Fermi--Dirac distribution for fermions should be used.
$$n_{CE} = n_{DOS} \exp\left(\frac{qV - E_g}{k_BT}\right) = n_{0,CE} \exp\left(\frac{qV}{k_BT}\right)$$
In some cases an ideality factor $m$ is introduced \cite{Kirchartz2012}, which can help to account for the shape of the density of states of the conduction band, so the expression can be found as:
$$n_{CE} = n_{0,CE} \exp\left(\frac{qV}{mk_BT}\right)$$

\paragraph{Charge versus light bias trend - Linear part}
In perovskite solar cells a linear trend sums to the exponential part, accounting for the accumulation in the selective contacts and electrodes, in a parallel plate capacitor fashion \cite{Gelmetti2017,Ryan2017a}: $Q = C_g \cdot V = \frac{\epsilon_0 \epsilon_r A}{d} \cdot V$ where $C_g$ is the geometric capacitance, $A$ is the active area, and $d$ is the thickness of the dielectric. More exactly, $d$ is the distance between the regions where the opposed charges are getting accumulated, which is the space charge layers (usually depletion layers) in the electrodes. So this value can be somewhat wider than just the distance separating the two electrodes interfaces. By consequence, $\epsilon_r$ should be considered as a thickness-weighted mean of the $\epsilon_r$ of each material between the two accumulation zones.

\paragraph{Charge versus light bias trend - Linear part in presence of ions}
The presence of mobile ions in perovskite materials which can accumulate at the perovskite/contacts interfaces, adds an additional capacitance, which sums up to the geometric capacitance, as we showed in \authoryear{Moia2019}. Nevertheless, as pointed out in \cpageref{ce_limitations_perovskite}, the \acr{ce} measurements are never carried on for long enough to include the ionic migration, and so also the ionic accumulation capacitance gets ignored in our experiments. This can be visualized with the simulation reported in \cref{}.

\begin{SCfigure}%[!hbtp]%
	\centering
	\includegraphics[width=0.45\textwidth]{.pdf}
	\mycaption[Simulation of CE experiment: charge versus light bias with or without mobile ions.]{.}\label{fig:}
\end{SCfigure}



\FloatBarrier
\section{Transient PhotoVoltage (TPV)}
	\epigraph{\textit{"Imma firin mah lazor\\pewpew pewpewpew"}}
	
	\paragraph{Concept} While a complete device is kept open circuit under constant illumination, a small extra illumination is added via a short laser pulse.
	The \gls{voc}, originally at its steady state value, will be perturbed due to the greater generation rate during the laser pulse.
	From the \gls{voc} versus illumination relation for photodiodes reported in \cref{eq:voc_vs_phi} follows that the \gls{voc}, at this new higher illumination, increases (this is not always the case, as for non-stabilized perovskite solar cells \cite{Calado2016}).
	After the short pulse the \gls{voc} will slowly go back to the steady state value relative to the constant illumination.
	The dynamics of this \gls{voc} relaxation back to the steady state value is the focus of Transient PhotoVoltage experiments. 
	
	\paragraph{Procedure} The device is kept under 1~sun illumination by a white \gls{led} ring at open circuit until stabilization is reached.
	1~sun equivalent illumination is defined as the illumination at which the a silicon photodiode gives the same \gls{jsc} as under calibrated 1~sun from the solar simulator.
	Then an additional illumination pulse is provided by a nitrogen laser. The pulse duration (\SI{\approx 1.5}{\ns}) is shorter than the oscilloscope resolution we usually employ, so we assume that the measurement happens when the pulse is already over.
	In the literature, this is not always the case as other research groups use a \gls{led} diode for the pulsed illumination \cite{Calado2016}.
	Usually a wavelength of \SI{650}{\nm} is selected using a Rhodamine B solution\cite{RadiantDyesLaser}, this wavelength illuminates in depth the perovskite layer (in contrast to a blue light where the illumination would be absorbed within the first hundreds of nanometres of the material \cite{Bi2016}).
	During all the process, the device is connected to an oscilloscope, registering the open circuit voltage profile (the \SI{1}{\Mohm} resistance of the oscilloscope is a good approximation of open circuit). %An auxiliary output from the square wave pulse generator used for the pulse is used for the trigger of the oscilloscope.
	The voltage profile gets averaged over a few tens of pulses in order to increase the signal to noise ratio.
	Then the background light intensity is slightly decreased and, after the stabilization to the new steady state, the new  \gls{voc} is registered and more transients are registered. This process is repeated over a few tens of light illuminations from 1~sun down to dark.
		
	\paragraph{Small perturbations regime} The intensity of the laser pulse is attenuated using a variable neutral density filter (a partially reflecting wheel with different positions for different transmittivities) so that the voltage perturbation caused by the light pulse does not exceed \SI{10}{\mV} with 1~sun background illumination intensity.
	We consider this a "small-enough" perturbation.
	Clearly, the pulse intensity which could be considered a "small-enough" perturbation at high background illumination is not small any more at lower nackground illuminations and definitively cannot be small at dark conditions.
	We \emph{do not} regulate the pulse intensity depending from the background illumination for being able to use this data also for \acr{dc} studies. The \acr{dc} does not intrinsically need the usage of a constant pulse intensity but it needs the measurement of a \acr{tpc} for each pulse intensity. The switch from \acr{tpv} to \acr{tpc} setups and back would be complex and time demanding with the current setup. Anyway quite all the information from the \acr{tpv} is obtained from the high background illumination intensity measurements.

	For each illumination intensity, the reported decay is the result of averaging around 30~transients. This manages to reduce the noise.

	%For the interpretation, see \cpageref{interpretation_tpv}; for the implementation, see \cpageref{r_tpv}.


	\paragraph{Noise treatment}\label{tpv_robust}

	Most of the observed noise (\SI{< 2E-7}{\s}) is due to the radiofrequency emitted by the spark in the nitrogen laser which gets absorbed by all the non-coaxial cables (coaxial ones don't) and from the circuitry of the samples holder acting as a receiving antenna. On the contrary to what happens for \acr{ce}, the short times noise seems to not follow a constant pattern, so averaging the measurement over a few repetitions (usually 30) manages to reduce the noise.

	This noise can affect the exponential or bi-exponential fitting, for this reason a robust fitting routine has been used, which gives a lower weight to outlier points. An example can be seen in \cref{fig:tpv_robust}.

	\begin{figure}
		\centering
		\includegraphics[width=0.9\textwidth]{{tpv_robust/TPV_ig52-68-3_0.883897_V-monoexp}.pdf}
		\mycaption[Robust and normal fitting comparison.]{In grey the fitted points, in yellow the points not considered for the fitting, the solid black line is the \gls{loess}. The normal non-linear least squares fitting (in green) is affected by noise, outliers and characteristics not of interest by the model. The non-linear robust fitting (in brown) manages to reduce the weight of these points.}\label{fig:tpv_robust}
	\end{figure}


	\paragraph{Voltage Increase Value}\label{tpv_deltaV}

	The value of the voltage increase due to the additional illumination is needed for \acr{dc} measurement.

	In the group this $\Delta V$ value was traditionally obtained subtracting the steady state \gls{voc} from the maximum voltage point in the measured transient. This is obviously heavily affected by the aforementioned noise when a short time window is used.

	The following alternatives were tested:
	\begin{itemize}
		\item The linear factor in an exponential fit was used, but it can fail if the decay does not have a simple exponential shape (often a bi-exponential, sum of two exponentials, is observed);
		\item The sum of the two linear factors in a bi-exponential fit, which could work but one have to carefully set boundary values to the fitting parameters for avoiding a fast exponential matching just some noise;
		\item The maximum value of a \gls{loess} local regression was used, but this underestimates the value, especially when the peak top are just few points (when the measurement time window is large);
		\item The average of the values registered starting from the maximum voltage point and during a specified time lapse.
	\end{itemize}

	This last option is the one currently in place. The average was performed over \SI{50}{ns} after the peak and this allowed us to get a reliable $\Delta V$ value.

\section{Interpretation of Transient PhotoVoltage}\label{interpretation_tpv}

\paragraph{Factors Affecting the \acr{tpv}}
The decays we can observe are limited at long times by the discharge of the extra charge through the oscilloscope resistance and through the device shunt resistance, whatever is the smallest. This happens with an RC time of the circuit composed by the capacitance of the device and the \SI{1}{\Mohm} resistance of the oscilloscope or the internal device resistance. The oscilloscope resistance could be varied using an attenuating probe (usually 10X or 100X). This limit to long times is often observed at low light intensities as a plateau in the \acr{tpv} graph. XXXXXXXXXXXXXXXXXXXXXXXXXXXXXXXXXXXXXXXXXXXXXXXXXXXXXXXXXXXXXXXXXXXXXXXXXXXXXXXXX

\begin{SCfigure}
	\centering
	\includegraphics[width=0.45\textwidth]{tpv_RCtime/TPVdarkTime_vs_RCdarkTime.pdf}
	\mycaption[\gls{tpv} time has an upper bond due to discharge through oscilloscope.]{Dark \acr{tpv} time (from a robust exponential fit) versus RC time derived from the geometric capacitance from \acr{dc} and the \SI{1}{\Mohm} of the oscilloscope. Each point is a different device for a total of 76 devices, including many different structures. The green line indicates the 1 to 1 relationship.}\label{fig:tpv_RCtime}
\end{SCfigure}

\section{Interpretation of Transient PhotoVoltage Referenced to Charge Extraction}\label{interpretation_tpvce}

recombination order $\Phi = \lambda + 1$ \cite{Shuttle2008d,Credgington2011}



\section{Transient PhotoCurrent (TPC)}

	The device is kept under 1~sun illumination by a white \gls{led} ring at short circuited through a \SI{50}{\ohm} resistor until stabilization is reached. 1~sun equivalent illumination is defined as the illumination at which the a silicon photodiode gives the same \gls{jsc} as under calibrated 1~sun from the solar simulator. Then an additional illumination pulse is provided by a nitrogen laser.

	The signal is acquired by an oscilloscope in parallel to the \SI{50}{\ohm} resistor. This allows us to measure a potential drop across the resistor and the related current via Ohm's law. Subtracting the constant current due to the background illumination and integrating the transient over time gives the charge generated by the laser pulse.

	This process is repeated at 1~sun and at dark background illumination conditions.

	For each illumination intensity, the reported decay is the result of averaging around 30~transients. This manages to reduce the noise.

	For the interpretation, see \cpageref{interpretation_tpc}; for the implementation, see \cpageref{r_tpc}.

\section{Differential Capacitance (DC)}

	This is a meta-measurement as it just combines the data from \acr{tpv} and \acr{tpc} without requiring any additional experiment\cite{Shuttle2008}, sometimes also referred to as "differential capacitance".

	As explained in \cpageref{interpretation_dc}, the electrical capacitance of a solar cell is not a constant (as in most of the commercial capacitors), indeed it depends on the applied voltage bias or light bias.

	The charge obtained with the \acr{tpc} (in case the dark and illuminated results were different, the third quartile of all \acr{tpc} measurements was used) is divided by an array of values obtained from \acr{tpv}, one for each illumination intensity. The needed value is the \gls{voc} increase due to the laser pulse, prior to the decay to steady state, for each illumination intensity (obtained as described in \cpageref{tpv_deltaV}).

	This allows us to estimate the capacitance of the solar cell device at open circuit with various illumination intensities.

	For the interpretation, see \cpageref{interpretation_tpc}; for the implementation, see \cpageref{r_tpc}.

\section{Impedance Spectroscopy}

\section{Stark Spectroscopy (ElectroAbsorbance)}

\section{Interpretation of Kelvin Probe Force Microscopy}\label{interpretation_kpfm}

\section{Molecular Characterization}
\subsection{Interpretation of Band Gap Values Obtained via Tauc Plot, PhotoLuminescence and Computational Simulations}\label{interpretation_bg}

%	Otra cosa, flipé mucho con la respuesta del Vidal y me puse a intentar
%	ver que se supone que se saca del espectro experimental y desde cual
%	pico. Como el dijo, las simulaciones son correctas.
%	Pero, creo que sea mi concepto (el HOMO-LUMO está bastante lejos del
%	absorption onset, como demostrado de sus simulaciones donde el
%	absoprtion onset, por ejemplo de TAE-1, está a 2.9 eV y el HOMO-LUMO gap
%	está a 5.06 eV) que lo de Vidal (el pico de máxima absorción no tiene
%	nada a que ver con el HOMO-LUMO gap, en mi opinión para nada) estaban
%	totalmente equivocados y que habría que enviar a medir el UPS de las
%	moléculas para sacar el HOMO-LUMO gap.
%
%	El concepto está explicado en el primer párrafo de esta pagina:
%	https://chemical-quantum-images.blogspot.com/2013/06/why-is-homo-lumo-gap-not-good-guess-of.html
%
%	Lo que he puesto, o sea que lo hemos medido por Tauc plot, creo colaría
%	porque todo el mundo lo mide así y está bien aceptado como método. Pero
%	la verdad es que es valido solo para semiconductores donde los orbitales
%	están bien deslocalizados, no como nuestra molécula en solución donde el
%	estado es bien localizado en la molécula.

\section{Our Solar Cells Characterization Steps}

	In this section I'll describe the routinary characterization performed in Palomares group.




\chapter{Comparison of HTM in Bottom Cathode CsFAMAPbIBr Solar Cells by Means of PhotoPhysical and Chemical Characterization}\label{ch:tae}
\chaptermark{Comparing HTM}
	\graphicspath{ {./contents_img/tae/} }
	\section{Recombination analysis via TPV}
\section{Stored charge profile via DC}
\section{Layers workfunction via KPFM}
\section{Conclusions}
\section{Critical Assessment}

\chapter{Insights on the Charge Storage \textit{via} Thickness Variation of Layers in Top Cathode MAPI Solar Cells by Means of PhotoPhysical Characterization}\label{ch:thicknesses}
\chaptermark{Thickness of Layers}
	\graphicspath{ {./contents_img/thicknesses/} }
	\epigraph{\textit{"Let's change as little as possible"}}

\myparagraph{Abstract} A series of top cathode perovskite solar cells have been fabricated with slightly different layers thicknesses. We verified that these variations affected as little as possible the rest of the solar cell stack, so that we could unequivocally relate the observations to the changes. The characterization via current-voltage sweeps and photophysical techniques allowed us to study the charge distribution in these devices.

\myparagraph{Publications} Part of this chapter has been published in \fullcite{Gelmetti2017}.

\section{Introduction}

\myparagraph{Disentangling parameters' impact} Perovskite synthesis is a very easy and very fragile process at the same time. When varying a fabrication parameter, for example during an optimization, it is quite likely to provoke a "butterfly effect" with the resulting device differing from the reference one by much more than the characteristic under study. A principal component analysis of the fabrication parameters would be needed for a rational optimization, but such a complex procedure is further hindered by the difficulty of identifying all relevant contributions.

\myparagraph{Varying the thickness} In this study we vary a set of parameters that hopefully have a foreseeable relation with the resulting device structure: The spin coating speed for each layer. The effect of this variation should just affect the layers thicknesses with a minor influence on the other device physical features. This will allow us to univocally relate each layer thickness variation to the characterization results variation. The complete devices were studied by means of current-voltage sweeps, \acr{ce}, \acr{tpv}, and \acr{dc}.

\myparagraph{The devices} The chosen architecture was a top cathode \gls{ito}/\gls{pedotpss}/\gls{mapi}/\gls{pcbm70}/\ch{Ag} device (fabrication described in \cpagerefrange{methods_top}{methods_top_end}) and the layers whose thickness was independently varied are the \gls{pedotpss} (\gls{htm}), the \gls{mapi} (absorber), and the \gls{pcbm70} (\gls{etm}).

\section{Interpretation of \gls{jsc} and \gls{ff} from Current-Voltage Sweeps}

\section{Interpretation of Transient PhotoCurrent}\label{interpretation_tpc}

\section{Interpretation of Differential Capacitance}\label{interpretation_dc}

A common commercial capacitor has a capacitance which is constant regardless to the voltage applied to it, as can be seen in \cref{fig:cap_voltage_dependence_commercial}.

\begin{figure}%[!hbtp]%
	\centering
	\begin{subfigure}[t]{0.45\textwidth}
		\includegraphics[width=1\textwidth]{cap_voltage_dependence/reference220nF/reference220nF.pdf}
		\subcaption{Commercial \SI{220}{\nano\F} capacitor.}\label{fig:cap_voltage_dependence_commercial}
	\end{subfigure}
	\qquad
	\begin{subfigure}[t]{0.45\textwidth}
		\includegraphics[width=1\textwidth]{cap_voltage_dependence/TAE-1_ig94-1559-1/DC-capacitance-TAE-1_ig94-1559-1.pdf}
		\subcaption{Perovskite solar cell.}\label{fig:cap_voltage_dependence_tae1}
	\end{subfigure}
	\mycaption[Capacitance dependence on applied voltage.]{In (a) the capacitance of a commercial capacitor is reported, it was measured using \acr{ce} with applied voltage bias instead of the classical light bias used for solar cells. The capacitance is obtained as the extracted charge over the applied voltage prior to short circuiting. In (b) the typical capacitance versus voltage profile of a \gls{fto}/\dTiOtwo/\mpTiOtwo/\acr{csfamapbibr}/\tae1/Au device is shown. In this case the indicated voltage is originated by various illumination intensities at open circuit prior to short circuiting.}\label{fig:cap_voltage_dependence}
\end{figure}

\section{Interpretation of Transient PhotoVoltage Referenced to Differential Capacitance}\label{interpretation_tpvdc}


\section{Varying \gls{mapi} Thickness (Absorber Layer)}


\section{Varying \gls{pcbm70} Thickness (\gls{etm} Layer)}
\section{Varying \gls{pedotpss} Thickness (\gls{htm} Layer)}
\section{Conclusions}
\section{Critical Assessment}

\chapter{Influence of Ionic Motion on Electronic Recombination as Observable in Impedance Spectroscopy}\label{ch:impedance}
\chaptermark{Impedance Sectroscopy}
	\graphicspath{ {./contents_img/impedance/} }
	
\myparagraph{Abstract} 

\myparagraph{Publication} This study has been published in %\fullcite{}.

\section{Interpretation of Impedance Spectroscopy}

\section{Interpretation of Stark Spectroscopy (ElectroAbsorbance)}




\section{Measured and simulated impedance spectra characteristics}
\section{Ionically gated interfacial transistor}
\section{Ionic-to-electronic current amplification}
\section{Capacitor-like and inductor-like behaviour}
\section{Critical Assessment}


\chapter{Modelling Common Characterization Techniques with Time-Dependent Drift-Diffusion Model of Perovskite Solar Cells}\label{ch:modelling}
\chaptermark{Modelling}
	\graphicspath{ {./contents_img/modelling/} }
	\epigraph{\textit{"GIGO: Garbage In, Garbage Out"}}

\paragraph{Abstract} During my 3-month stay in Dr. Piers R. F. Barnes and Prof. Jenny Nelson groups I utilized and expanded a modelling software developed by Dr. Phil Calado, Dr. Mohammed Azzouzi, Benjamin Hilton, and Piers R. F. Barnes written in Matlab. As the modelling software had already demonstrated a great descriptive power\cite{Belisle2017, Calado2016}, I implemented a few more characterization techniques with the objective of reproducing and understanding real-world data.

\paragraph{Publications} Most of the source code of software described in this chapter has been released and can be accessed on \url{https://github.com/barnesgroupICL/Driftfusion/}

\section{DriFtFUSION Time Resolved 1-D Drift-Diffusion Modelling}

\section{DriFtFUSION Homo-junction and Hetero-junction Versions}


\section{Improvements and Additions to the DrIFtFUSION Core}

\section{Charge Extraction}

\section{Transient PhotoVoltage}

\section{Impedance Spectroscopy in Time Domain}
\epigraph{\textit{"I thought you could implement this"\\"Ehm... Do you mean, by tomorrow?"\\"That would be amazing!"}}

\section{Impedance Spectroscopy in Frequency Domain}




\section{ElectroAbsorption}

\subsection{Ideality Factor}\label{dd_ideality}
\paragraph{Ideality Factor from \gls{voc}}

\paragraph{Ideality Factor from Current-Voltage Points}\label{dd_ideality_dark_jv}
https://www.pveducation.org/pvcdrom/characterisation/measurement-of-ideality-factor


\subsection{Techniques Which Could Be Implemented}
\epigraph{\textit{"Wow, this is a gold mine!"}}





\paragraph{IMVS}

\paragraph{IMPS}

\paragraph{Mott-Schottky}
\url{https://en.wikipedia.org/wiki/Mott-Schottky_plot}

\chapter{Software Development for Data Acquisition and Data Analysis}\label{ch:software}
\chaptermark{Software Development}
	\graphicspath{ {./contents_img/software/} }
	
\section{PyPV: An easy Current-Voltage curves acquisition interface}

\epigraph{\textit{"Ok, I finally completed it, by the way, why did you start developing this?"\\"Well, it was just a proof of concept, but it's nice you worked on it"}}


	\subsection{Previous software}
		\begin{SCfigure}%[!hbtp]%
			\centering
			\includegraphics[width=0.5\textwidth]{old_iv_software/old_iv_software.png}
			\caption{}\label{fig:old_iv_software}
		\end{SCfigure}

	\subsection{Original Project}
		I received a proof-of-concept software developed by Daniel Fernandez Pinto and decided to continue the development. At that point the software had an interface with few buttons and a working Keithley communication library.

	\subsection{User's requests}
	\subsection{Implementation and user interface}

		\myparagraph{Autoscale} As it was explained in page~\pageref{autoscale}, the automatic scale setting of the Keithley is detrimental for perovskite solar cells dynamic measurements.

		\myparagraph{Auto-measure}\label{automeasure}
		
	\subsection{Limitations}
		
		The interface development has been started with the "Monkey Studio" software, which development has ceased even before the start of PyPV. This demonstrated to be a big failure in long term development planning.

\section{Robust and quick data analysis via R scripts}
\epigraph{\textit{"Is there an Origin version for Linux?"\\"No"}}


	\subsection{Charge Extraction (\acr{ce})}\label{r_ce}


	\subsection{Transient PhotoVoltage (\acr{tpv})}\label{r_tpv}
	
		\subsection{Transient PhotoCurrent (\acr{tpc})}\label{r_tpc}
		
			\subsection{Differential Capacitance (\acr{dc})}\label{r_dc}

\section{Maximum Power Point Tracking}\label{software_mppt}
\epigraph{\textit{"A student of mine sells a complete system for that, just buy it"}}

CITE Cimaroli2017

http://candlelight-systems.com/

\section{Time Resolved 1-D Drift and Diffusion Modelling}
	\subsection{Improvements and Additions to the DrIFtFUSION Core}
	\subsection{Impedance Spectroscopy}
	\subsection{ElectroAbsorption}
	
	\subsection{Ideality Factor}\label{dd_ideality}
		\myparagraph{Ideality Factor from \gls{voc}}
		
		\myparagraph{Ideality Factor from Current-Voltage Points}\label{dd_ideality_dark_jv}
https://www.pveducation.org/pvcdrom/characterisation/measurement-of-ideality-factor

\subsection{Techniques Which Could Be Implemented}
\epigraph{\textit{"Wow, this is a gold mine!"}}

\myparagraph{IMVS}

\myparagraph{IMPS}

\myparagraph{Mott-Schottky}
\url{https://en.wikipedia.org/wiki/Mott-Schottky_plot}

\chapter*{Overall Conclusions}\label{ch:conclusions}
	\addcontentsline{toc}{chapter}{Overall Conclusions}
	%\epigraph{\textit{"AAAAAAAAAAAAAAAAAAA"}}

In this thesis, perovskite solar cells have been fabricated and characterised by means of advanced techniques.
These techniques have been critically analysed taking advantage of drift\hyp{}diffusion modelling.
The provided contributions have widened the understanding of perovskite solar cells characterisation output, allowing the scientific community to get more information from the already available techniques.
This knowledge will help in the identification of performance bottlenecks of photovoltaic devices, easing the solar cells optimization.

The main findings can be summarised as follows:
\begin{enumerate}
	\item The advanced characterisation output has to be interpreted with great care and the theoretical framework from previous type of photovoltaic devices is no longer enough for perovskite solar cells.
		In \cref{ch:characterization} I described and critically analysed the employed characterisation techniques.
		For each, the strategy for noise reduction is explained and compared with the tested alternatives.
		Taking advantage of the insight obtained from drift\hyp{}diffusion modelling the following hypotheses are thrown:
		time resolved photoluminescence will also evolve with time due to slow ionic profile adaptation to the new illumination conditions;
		charge extraction experiments should show a long lasting ionic displacement current giving information about ionic capacitance rather than geometric capacitance;
		I confirm that the exponential trend in charge extraction can be explained with the chemical capacitance and should start at light biases close to the built-in voltage;
		the fast component of bi\hyp{}exponential decays in transient photovoltage observed at low background light intensity could be due to ionic profile update before the generated charge recombines.

	\item The difference between the contacts' energy levels provides an unreliable estimation of the built-in voltage of perovskite solar cells.
		In \cref{ch:tae} and in \cite{Gelmetti2019} we have fabricated bottom cathode devices with four different \gls{htm} ensuring to fulfil the conditions for having a fair comparison.
		Comparing the built-in voltage of the different solar cells using \gls{ce} and \gls{dc} techniques we realised that the involved energy levels are rather different than the values expected from cyclic voltammetry measurements.
		After excluding major differences in the carriers lifetimes in the four devices, we have measured the \textsl{in situ} work function of the \gls{htm}.
		We have observed significant deviations from the substrate value just for some of the \gls{htm} and used this result for explained the measured \gls{voc} and built-in voltage.

	\item In top cathode \gls{ito}\-/\gls{pedotpss}\-/\gls{mapi}\-/\gls{pcbm70}\-/\ch{Ag} perovskite solar cells, the photogenerated charge at low light intensity and at open circuit conditions is storage location has been observed.
		In \cref{ch:thicknesses} and in \cite{Gelmetti2017} we have shown that the holes are accumulated where expected, which is in a depletion layer in the \gls{pedotpss} at the interface with the perovskite layer.
		Regarding the electrons, we have demonstrated that the storage location is not at the \gls{pcbm70} interface with the perovskite, rather it can be close to the metallic electrode\-/selective contact interface or through the whole \gls{etm} layer.

	\item Giant and negative capacitance observed in perovskite solar cells \textsl{via} impedance spectroscopy do not involve a giant charge accumulation.
		In \cref{ch:impedance} and in \cite{Moia2019} we showed that they rather stem from the influence of the ionic movement respectively on the recombination and on the injection barriers.
		Additionally, the whole apparent capacitance spectra is explained taking advantage of the drift\hyp{}diffusion results.
		The simulation method and modular structure is explained in detail.
\end{enumerate}

\chapter*{Acknowledgements}
	\addcontentsline{toc}{chapter}{Acknowledgements}
	\epigraph{\textit{"Ciao mamma!"}}

AAAAAAAAAAAAAAAAAAAAAAAA

	%\selectlanguage{}

	\selectlanguage{british}
	
	AAAAAAAAAAAAAAAAAAAAAAAA
	
		\selectlanguage{catalan}
		
		Gracies a l'Emilio per confidar en mi

	\appendix

\chapter{Code}
	\label{ch:code}
	\graphicspath{ {./contents_img/code/} }
	\epigraph{\textit{"Please put comments after each line"}}

\paragraph{Abstract} Even if the code is available online, I though that having it here could help some readers to understand how things were implemented.

\paragraph{Publications} The source code of software included in this chapter has been released and can be accessed on \url{https://github.com/barnesgroupICL/Driftfusion/}

\section{Drift-Diffusion Modelling}
\epigraph{\textit{"Ten levels of nested for loops??"\\"Yes, that's the way"}}

\subsection{Core Utilities}

\lstinputlisting[caption = {}]{contents_code/asymmetricize.m}

\lstinputlisting[caption = {}]{contents_code/changeLight.m}

\lstinputlisting[caption = {}]{contents_code/findOptimVoc.m}

\subsection{Impedance Spectroscopy and ElectroAbsorbance}
\lstinputlisting[caption = {}]{contents_code/ISwave_EA_single_demodulation.m}


	\printindex

\chapter*{List of Oral and Poster Presentations at Conferences}
	\begin{itemize}
		\item 45 min plenary oral presentation, at \textit{International Krutyn Summer School on Advanced Perovskite, Hybrid and Thin-film Photovoltaics} in Krutyn, Poland on June the 14th, 2016 titled \textit{"How to make efficient perovskite solar cells and charge transfer reactions in perovskite solar cell"}
		\item 20 min plenary oral presentation at \textit{Workshop on Flexible Electronics} in DEEEA-URV, Tarragona, Spain on June the 29th, 2016 titled \textit{"Photophysical characterization of charge transfer and recombination in hybrid lead halide perovskite solar cells"}
		\item 15 min plenary oral presentation at \textit{Stability of Emerging Photovoltaics Conference} (SEPV18) in Barcelona, Spain on February the 20th, 2018 titled \textit{"The Relevance of the Energy Alignment Shift in Organic Semiconductor/Perovskite Interface: Influence in the Open Circuit Voltage"}
		\item Poster at \textit{International Conference on Hybrid and Organic Photovoltaics} (HOPV17) in Lausanne, Switzerland on May the 22nd, 2017 titled \textit{"Quasi-Fermi Energy Shift for Hole Transport Material in Perovskite Solar Cells"}
		\item 30 min invited seminar oral presentation in INAM-UJI, Castelló, Spain on June the 26th, 2018 titled \textit{"Simulating Impedance Spectroscopy of Pervoskite-based Devices via Time-Dependent 1D Drift-Diffusion Model"}
		\item 15 min plenary oral presentation in \textit{Graduate Students Meeting on Electronics Engineering} in DEEEA-URV, Tarragona, Spain on June the 28th, 2018 titled \textit{"Simulating the impedance spectroscopy of an hybrid perovskite solar cells via time-dependent 1D driftdiffusion model"}
		\item 15 min oral presentation in \textit{NanoBio Conference 2018} in Heraklion, Crete, Greece on September the 27th, 2018 titled \textit{"Perovskite solar cells impedance spectroscopy explained via 1D time dependent drift-diffusion modelling"}
	\end{itemize}

	\printbibliography[category=mypapers, title=List of Publications]

	\printbibliography[notcategory=mypapers]


\end{document}
