
\hspace{\parindent}In \cref{ch:intro}, an overview on the solar cells main concepts and working mechanisms are presented.
Perovskite solar cells are inserted in the provided framework and the research state of the art is briefly described.

In \cref{ch:methods}, materials, equipment, and fabrication of the studied perovskite solar cells are described in detail.

In \cref{ch:characterization}, followed conventions, routine and advanced characterisation techniques are described.
For each characterisation technique the involved concepts, formalism, and equations required for the data analysis are explained.
Additionally, I presented here my own observations and thoughts about the interpretations of the characterisation output, presenting also unpublished results from drift\hyp{}diffusion modelling.

In \cref{ch:tae}, the performances of bottom cathode perovskite solar cells fabricated using four different hole transporting materials are compared.
Then the origin of the observed differences are studied by means of small perturbation transient techniques.
Part of this chapter has been published in \cite{Gelmetti2019}.

In \cref{ch:thicknesses}, top cathode solar cells have been fabricated exploring the thickness of each layer.
For each device, the advanced characterisation output have been compared obtaining insight on the charge storage location.
Part of this chapter has been published in \cite{Gelmetti2017}.

In \cref{ch:impedance}, drift\hyp{}diffusion modelling of perovskite solar cells is employed for understanding the features observed in impedance spectroscopy.
Specifically, the implemented simulation is described and the resulting apparent capacitance at different illumination, bias, and frequency is explained.
Part of this chapter has been published in \cite{Moia2019}.

In \cref{ch:software}, various software I implemented during my thesis is described.
First, the implementation of more functions and simulations for drift\hyp{}diffusion modelling is shown.
Second, the development of a graphical user interface for current\hyp{}voltage sweeps acquisition is presented.
Third, the scripts used for automated and reliable data analysis are reported; they have been used for plotting most of the graphics included in this thesis.

An updated version of this thesis can be found on \url{https://github.com/ilario/documents_in_latex-PhD_thesis/}.