
\section{Contributions to Driftfusion}
	\epigraph{\textit{\enquote{Wow, this is a gold mine!}}}

	\subsection{Introduction}
		During my 3-month stay in Dr.\ Piers R.\ F.\ Barnes and Prof.\ Jenny Nelson groups in Imperial College London, I utilized and expanded a modelling software developed by Dr.\ Phil Calado, Dr.\ Mohammed Azzouzi, Benjamin Hilton, and Piers R.\ F.\ Barnes.
		As the modelling software had already demonstrated a great descriptive power \cite{Belisle2017,Calado2016,Calado2019}, I implemented a few more characterization techniques with the objective of reproducing and understanding real-world data.
		The impedance spectroscopy simulation is not included here as it has been treated in \cref{ch:impedance}.
		Driftfusion \cite{Calado2018a,Calado2016} is a time resolved, one dimensional, drift\hyp{}diffusion modelling platform.
		It takes advantage of Matlab \texttt{pdepe} solver for partial differential equations with boundary conditions in time and one spatial dimension.
		%Solve initial-boundary value problems for parabolic-elliptic PDEs in 1-D.
		Currently, it can simulate the electrostatic potential, and the density of electrons, holes and one ionic specie at specified depths in a multi\hyp{}layered electronic device and at specified time points.
		Even if an heterojunction version of the code is available, I implemented functions for the homojunction (every material has the same bandgap) release, which is much more stable and tested.
		All the described scripts can be obtained from \url{https://github.com/barnesgroupICL/Driftfusion/tree/2018-EIS}.


	\subsection{Additions to the Driftfusion Core}
		\epigraph{$>>$ 0.1 + 0.2 == 0.3\\
			ans =\\
			logical\\
			0}{MATLAB}



\paragraph{\texttt{equilibrate\_minimal} -- Generate a reduced set of solutions at standard conditions}
It's a reduced version of \texttt{equilibrate}. 
Uses analytical initial conditions and runs to equilibrium and steady state, obtaining just the most important solutions
 Takes the parameters from \texttt{pinParams.m} file and tries
 to obtain a steady state solution (as if the device has been left for
 a long period of time). This solution can then be used as accurate
 initial conditions for other simulations, \textsl{e.g.} a JV scan.
 Note that the stabilization time is consistently adjusted to appropriate values for to
 ensure there are numerous mesh points where large gradients in the time
 dimension are present.
 
\textit{Inputs:} optional, struct containing the needed parameters as obtained
     from \texttt{pinParams.m}.
     
\textit{Outputs:} \texttt{sol\_eq} - short circuit, dark, no mobile ionic defects, no \gls{srh};
   \texttt{sol\_i\_eq} - short circuit, dark, mobile ionic defects, no \gls{srh};
   \texttt{sol\_i\_eq\_SR} - short circuit, dark, mobile ionic defects, with \gls{srh};
   \texttt{ssol\_i\_eq} - open circuit, dark, mobile ionic defects, no \gls{srh};
   \texttt{ssol\_i\_eq\_SR} - open circuit, dark, mobile ionic defects, with \gls{srh};
   \texttt{sol\_i\_1S\_SR} - short circuit, 1 sun, mobile ionic defects, with \gls{srh};
   \texttt{ssol\_i\_1S\_SR} - open circuit, 1 sun, mobile ionic defects, with \gls{srh}.

\textit{Requires:} \texttt{pindrift}, \texttt{pinParams}, \texttt{stabilize}.

\textit{Required by:} \texttt{examples\_unit\_test}.

		\paragraph{\texttt{verify\-Stabilization} -- Verifies if the provided solution has evolved until its steady state}\label{verifyStabilization}
Checks if the solution reached a stabilized (\textsl{i.e.} steady state) status comparing the final time point with a manually specified time point. Beware that this can have false positives when the analysed solution had
 has been run over a time which is so small that the ionic movement did not even start.
 For each variable under study, the comparison integrates over the solution thickness the absolute value of the difference in the profile of this variable at two different times in the solution.
 If this integrated difference is larger than \num{1e-4} times the integrated profile of the same variable at the final time point at which its spatially average value has been subtracted, the solution is considered as not stabilized yet.
 The definition of this last threshold, will cause completely flat profiles to be unsuitable for this check.
 The variables which are employed are: ionic density, electrostatic potential, decimal logarithm of electrons density and decimal logarithm of holes density.
 
\textit{Inputs:} 1) a solution matrix, not the full structure, as contained in
     structures created by \texttt{pindrift};
   2) the time mesh array, as generated by \texttt{meshgen\_t} and included
     in structures created by \texttt{pindrift};
   3) the fraction of final time that can be used as a
     reference. In case of a stabilization, the smaller the stricter the
     check, for example if a solution has been run over \SI{10}{\s}, and the provided fraction is \num{1e-5}, the script will compare the solution at \SI{0.1}{\us} with the final one at \SI{10}{\s}. In case of a periodic time evolution, a specific value
     corresponding to the previous-to last cycle have to be used. The
     value needs to be strictly between 0 and 1.

\textit{Outputs:} a logic (true/false) asserting if the solution in input reached its stability.

%\textit{Requires:} nothing.

\textit{Required by:} \texttt{CE\_full\_exec}, \texttt{CE\_single\_exec}, \texttt{find\-Optim\-Voc}, \texttt{ISwave\_EA\_single\_exec}, \texttt{stabilize}, \texttt{TPVconst\_full\_exec}, \texttt{TPV\_single\_exec}.

		\paragraph{\texttt{stabilize} -- Evolves a solution until its steady state}
Simulates the solution increasing the maximum time until when steady state is reached. 
\texttt{verify\-Stabilization} is used for checking the reaching of the steady state.
For all the simulations including transients, the stabilization of the starting solution is of utmost importance.

\textit{Inputs:} 1) a solution structure as created by \texttt{pindrift};
   2) optional boolean, true by default, 
     if true forces the simulation to run at least once, if false the
     stabilization is run just when needed.
     
\textit{Outputs:} a solution structure that reached its steady state.

\textit{Requires:} \texttt{pindrift}, \texttt{verify\-Stabilization}.

\textit{Required by:} \texttt{CE\_full\_exec}, \texttt{change\-Light}, \texttt{equilibrate}, \texttt{equilibrate\_minimal}, \texttt{find\-Optim\-Voc}, \texttt{gen\-Int\-Structs\-Real\-Voc}, \texttt{gen\-Vapp\-Structs}, \texttt{ideality\_from\_dark\_jVpoints}, \texttt{ISwave\_EA\_single\_exec}, \texttt{ISwave\_full\_exec}, \texttt{ISwave\_full\_exec\_nonparallel}, \texttt{TPV\_single\_exec}.

		\paragraph{\texttt{asymmetricize} -- Break a symmetrical solution}\label{asymmetricize}
		Break a symmetrical solution (a symmetrical solution it's a repeated p-i-n-n-i-p stack, mainly used for simulating open circuit conditions, so that no boundary condition has to be set for the cathode) in two halves. Has the opposite use of the \texttt{symmetricize} function.
		Please note that further stabilization could be required.
		
		\textit{Inputs:} a symmetric structure as created by \texttt{pindrift} using the OC parameter set to 1.
		
		\textit{Outputs:} an asymmetric structure as created by \texttt{pindrift} using the OC
		     parameter set to 0 and applied voltage identical to the open circuit
		     voltage taken from the input symmetric structure.
		     
	\textit{Requires:} \texttt{pindrift}
	
	\textit{Required by:} \texttt{CE\_full\_exec}, \texttt{gen\-Int\-Structs\-Real\-Voc}, \texttt{ISwave\_full\_exec}, \texttt{ISwave\_full\_exec\_nonparallel}, \texttt{EA\_full\_exec}.

		\paragraph{\texttt{change\-Light} -- Change the illumination intensity of a solution}
		Generate solutions at a new light intensity.
		
		\textit{Inputs:} 1) a solution structure as created by \texttt{pindrift};
		2) the requested light intensity, zero is not supported as it is
		    much more robust to obtain a dark solution directly from equilibrate;
		3) the initial stabilization time, can be zero for an automatic
		     guess.
		     
		\textit{Outputs:} a solution structure at the new light intensity.
		
		\textit{Requires:} \texttt{pindrift}, \texttt{stabilize}.
		
			\textit{Required by:} \texttt{equilibrate\_minimal}, \texttt{gen\-Int\-Structs}.
			
		\paragraph{\texttt{gen\-Int\-Structs} -- Generates a list of solutions at various light intensities}\label{genIntStructs}
Generates a cell containing structures of solutions at logarithmically spaced light intensities, starting from the one at lower light illumination. Can be used for both symmetrical (for obtaining open circuit solutions) or asymmetrical solutions.

		\textit{Inputs:} 1) a solution structure as created by \texttt{pindrift} in dark conditions; 
		2) highest requested illumination;
		3) lowest requested illumination;
		4) number of illumination requested between the lowest and the highest illumination points, including extrema, except dark;
		5) logical, whether to include the dark solution in the output structure
		
\textit{Outputs:} 1) a cell containing structures of solutions at various light
     intensities;
2) an array with the voltages present in the solution, which is
     an approximation of the \gls{voc}, getting populated just if the input
     structures were at open circuit;
 3) an array with the currents present in the solutions.
 
\textit{Requires:} \texttt{change\-Light}, \texttt{pindrift}, \texttt{pinana}.

\textit{Required by:} \texttt{examples\_unit\_test}, \texttt{gen\-Int\-Structs\-Real\-Voc}.

		\paragraph{\texttt{gen\-Vapp\-Structs} -- Generates a list of solutions at various applied voltages}\label{genVappStructs}
Generates a cell containing asymmetric structures of solutions at various applied voltages.
Differently from \texttt{gen\-Int\-Structs} case, here the voltage array has to be specified.

\textit{Inputs:} 1) a solution asymmetric struct as created by \texttt{pindrift};
2) an array containing the requested applied voltages list.

\textit{Outputs:} a cell containing structures of solutions at various applied
     voltages, ordered with ascending voltages.
     
\textit{Requires:} \texttt{pindrift}, \texttt{pinana}, \texttt{stabilize}.

\textit{Required by:} \texttt{examples\_unit\_test}.

		\paragraph{\texttt{find\-Optim\-Voc} -- Find the exact \gls{voc} minimizing the current output}
Stabilize an asymmetric solution to the real open circuit voltage
which is the applied voltage where the current is minimized,
starting from the voltage present in the given solution. The suggested
workflow is to stabilize a symmetric solution (so that the charges
profile are close to OC conditions) and break in a half with
\texttt{asymmetricize}.
Differently from \texttt{findVoc}, this version requires MATLAB's Optimization Toolbox.
While the \gls{voc} can be conveniently obtained from the symmetrical solution, in same cases the zero current condition is not matched.
For this reason, the search for an applied voltage that minimizes the residual current can be needed.
		
		\textit{Inputs:} an asymmetric structure as created by \texttt{pindrift}, ideally
    breaking a symmetric solution using \texttt{asymmetricize} function so that
    the starting applied voltage is already close to the real \gls{voc}.
    
\textit{Outputs:} 1) an asymmetric structure as created by \texttt{pindrift} using the OC
     parameter set to 0 and applied voltage identical to the real open
     circuit voltage;
2) the value of the obtained open circuit voltage.

\textit{Requires:} \texttt{pindrift}, \texttt{pinana}, \texttt{stabilize}.

\textit{Required by:} \texttt{gen\-Int\-Structs\-Real\-Voc}.

		\paragraph{\texttt{gen\-Int\-Structs\-Real\-Voc} -- Generates a list of solutions at various light intensities ensuring open circuit}
		Generates a cell containing structures of asymmetric solutions at various light intensities at an accurate \gls{voc}.
		This script just uses other three scripts: \texttt{gen\-Int\-Structs}, \texttt{find\-Optim\-Voc}
		 and \texttt{asymmetricize}. Both symmetric and asymmetric solutions are supported in input, but the usage of symmetric solutions is strongly encouraged as the \texttt{find\-Optim\-Voc} will start from a condition closer to the \gls{voc}.
		
				\textit{Inputs:} 1) a solution structure as created by \texttt{pindrift} in dark
				     conditions, preferably a symmetric (open circuit) solution;
				2) higher requested illumination;
				3) lower requested illumination;
				4) number of requested illumination points between the lowest and the highest illuminations, including extrema, except dark;
				5) logical, whether to include the dark solution in the output
				     structure.
				
		\textit{Outputs:} 1) cell containing structures of asymmetric solutions at various light
		     intensities with an applied voltage equal to the \gls{voc};
		2) an array with the \gls{voc} values.
		
		\textit{Requires:} \texttt{change\-Light}, \texttt{pindrift}, \texttt{gen\-Int\-Structs},
		   \texttt{asymmetricize}, \texttt{find\-Optim\-Voc}, \texttt{stabilize}.
		   
%		\textit{Required by:} 
		
		\paragraph{\texttt{fromSteadyStateStructToTxt} -- Exports stabilized solutions data to text for plotting externally}
		Save the main data from a stabilized solution created by \texttt{pindrift} to text files, using a format easy to import with Origin (from OriginLab).
				\textit{Inputs:} 1) a structure as created by \texttt{pindrift};
				2) a char array to be used as prefix for the saved files names.
%		\textit{Outputs:} none.
%		\textit{Requires:} nothing.
%		\textit{Required by:} 

		\paragraph{\texttt{plot\_charges\_single} -- Plots various density and energy profiles over time}
Plot free charges and ions densities, and energy levels over device thickness and over time.
 This is inspired by Dr.\ Phil Calado's \texttt{pinExtract2.m} but rewritten.
 The created graphics include: profiles of net charge density and 
 comparison with the final profile; conduction and valence bands and 
 quasi-Fermi levels profiles; profiles of total concentration of 
 particles densities and comparison with the final profile, divided 
 per particle type; time evolution of voltage, current and specific 
 particles types in specific positions.

		\textit{Inputs:} 1) a symmetric or asymmetric structure of a solution evolving over time;
		2) optional char array, if provided saves the images in the current directory with the name starting with it.
%\textit{Outputs:} 
%\textit{Requires:} 
%\textit{Required by:} 

%	\subsection{Ideality Factor}
			\paragraph{\texttt{ideality\_from\_Voc} -- Obtains ideality factor from voltage \textsl{versus} illumination intensity relation}\label{dd_ideality}
	As explained in \cpageref{characterization_ideality}, the ideality factor can be obtained relating \gls{voc} to the light intensity.
	%		 data of a device illuminated at various illumination intensities
	
			\textit{Inputs:} a cell containing structures at open circuit and various
	     light intensities. The open circuit can be either obtained from
	     symmetric solutions (from \texttt{gen\-Int\-Structs}) or from asymmetric solutions
	     with applied voltage (from \texttt{gen\-Int\-Structs\-Real\-Voc}).
	
	\textit{Outputs:} the numeric ideality factor.
	
	\textit{Requires:} \texttt{pinana}.
	
	\textit{Required by:} \texttt{examples\_unit\_test}.
	
				\paragraph{\texttt{ideality\_from\_dark\_jVpoints} -- Obtains voltage dependent ideality factor from the stabilised current at different voltages}
	A voltage dependent ideality factor can be obtained from a dark current\hyp{}voltage sweep (or stabilised points in the case of hysteretic devices) \cite{Honsberg2019}.
	%		 data of a device illuminated at various illumination intensities
	
	\textit{Inputs:} 1) an asymmetric solution structure as created by \texttt{pindrift} in dark conditions;
	2) the maximum voltage point;
	3) the distance between voltage points.
	
	\textit{Outputs:} an array of ideality factor at different applied voltages.
	
	\textit{Requires:} \texttt{pindrift}, \texttt{pinana}, \texttt{stabilize}.
	
	\textit{Required by:} \texttt{examples\_unit\_test}.


	\subsection{Charge Extraction}
		Careful stabilization of illuminated solutions is crucial, especially for avoiding glitches appearing at high light illumination (minority charges accumulation at the edges of the symmetric solution).

		\paragraph{\texttt{CE\_full\_exec} -- Do Charge Extraction in a range of background light intensities}		
		
		\textit{Inputs:} 1) can be a cell structure containing structures at various background light intensities.
			This can be generated using \texttt{gen\-Int\-Structs}.
		    Otherwise it can be a single structure as created by \texttt{pindrift};
		   2) logical, whether to graph the individual solutions and
		     the overall graphics;
		   3) logical, whether to assign in volatile base
		     workspace the calculated solutions of single CE decays.
		
		\textit{Outputs:} a structure containing the most important results of the simulation.
		
		\textit{Requires:} \texttt{CE\_single\_exec}, \texttt{CE\_ISstep\_single\_analysis},   \texttt{CE\_ISstep\_subtracting\_analysis}, \texttt{CE\_full\_fit}, \texttt{CE\_full\_analysis},  \texttt{asymmetricize}, \texttt{pinana}.
		
%		\textit{Required by:} \texttt{}.


\paragraph{\texttt{CE\_single\_exec} -- Do a single Charge Extraction experiment}		

\textit{Inputs:} 1) a single asymmetric structure as created by \texttt{pindrift};
2) a first guess for maximum time which should be enough for registering the full extraction.

\textit{Outputs:} a structure with a solution evolving after the start of the CE experiment.

\textit{Requires:} \texttt{pindrift}, \texttt{verify\-Stabilization}.

\textit{Required by:} \texttt{CE\_full\_exec}.


		\paragraph{\texttt{CE\_full\_analysis} -- Plot Charge Extraction in a range of background light intensity \textsl{versus} open circuit voltage}


\textit{Inputs:} a structure containing the most important results of the CE simulation.

%		\textit{Outputs:} .

%		\textit{Requires:} \texttt{}.

\textit{Required by:} \texttt{CE\_full\_exec}.


		\paragraph{\texttt{CE\_full\_fit} -- Calculate geometric and chemical capacitance from CE data}		
		
		\textit{Inputs:} 1) an array containing the voltage drops which were
		     imposed in the CE experiments. As usually the CE experiments goes from
		     open circuit to short circuit, the voltage drop is equal to the \gls{voc};
		   2) an array with the extracted charges from CE at various light
		     intensities.
		
		\textit{Outputs:} an array with the values from the linear + exponential fit:
		     linear gradient, exponential function linear factor, exponent.
		
%		\textit{Requires:} \texttt{}.
		
		\textit{Required by:} \texttt{CE\_full\_exec}.
		
		\paragraph{\texttt{CE\_ISstep\_single\_analysis} -- Calculate charge extracted by Charge Extraction integrating the current}		
		
		\textit{Inputs:} 1) a structure with a solution being perturbed by CE;
		   2) original light intensity is needed in input as the solution provided
		     in input for CE is in dark;
		   3) logical, whether the residual current should
		     be subtracted, usually false for CE;
		   4) logical, whether to calculate the ionic contribution and graph the solution.
		
		\textit{Outputs:} 1) array of cumulative charge extracted at each time point;
		2) index of the previous array indicating the end of the early extraction, time at which the electronic charge have been extracted but the ionic defects did not move yet.
		
%		\textit{Requires:} \texttt{}.
		
		\textit{Required by:} \texttt{CE\_full\_exec}.
		
		\paragraph{\texttt{CE\_ISstep\_subtracting\_analysis} -- Calculate the charge excess in device under illumination}
			The carriers density profile in the illuminated steady state solution is subtracted from the profiles
			 at all time points of the solution evolving during the charge extraction experiment.
			 These profiles are integrated over the device thickness, obtaining a time resolved excess charge amount.
			This is used as a reference for Charge Extraction.
		
		\textit{Inputs:} 1) structure of the stable solution before the CE experiment, to be used as a reference;
		2) structure of the solution perturbed by CE.
		
		\textit{Outputs:} 1) array of excess charges in the whole device versus time;
		2) array of excess charges in the intrinsic layer versus time.
		
%		\textit{Requires:} \texttt{}.
		\textit{Required by:} \texttt{CE\_full\_exec}.
		

	\subsection{Transient PhotoVoltage}

		\paragraph{\texttt{TPVconst\_full\_exec} -- Simulate Transient PhotoVoltage in a range of background light intensities without varying the pulse intensity}	
The pulse intensity is tuned to be a small perturbation compared to the 1
 sun background illumination, then the \gls{tpv} is simulated for all the
 provided solutions without changing the chosen pulse intensity.
 
		\textit{Inputs:} 1) can be a cell structure containing structures at various background
		     light intensities. This can be generated using \texttt{gen\-Int\-Structs}.
		     Otherwise it can be a single structure as created by \texttt{pindrift};
		   2) logical, whether to assign in volatile base
		     workspace the calculated solutions of single \gls{tpv} decays.
				
		\textit{Outputs:} a structure containing the most important results of the simulation.
		
		\textit{Requires:} \texttt{TPV\_single\_exec}, \texttt{TPV\_single\_analysis}.
		
%		\textit{Required by:} \texttt{}.
		
		
				\paragraph{\texttt{TPVvariab\_full\_exec} -- Simulate Transient PhotoVoltage in a range of background light intensities adapting the pulse intensity for each simulation}
				For each of the provided solutions, the pulse intensity is tuned to be a small perturbation compared to the background illumination intensity, then the \gls{tpv} is simulated.
				
				\textit{Inputs:} 1) can be a cell structure containing structures at various background
				light intensities. This can be generated using \texttt{gen\-Int\-Structs}.
				Otherwise it can be a single struct as created by \texttt{pindrift};
				2) logical, whether to assign in volatile base
				workspace the calculated solutions of single \gls{tpv} decays.
				
				\textit{Outputs:} a structure containing the most important results of the simulation.
				
				\textit{Requires:} \texttt{TPV\_single\_exec}, \texttt{TPV\_single\_analysis}.
		
		
		\paragraph{\texttt{TPV\_single\_exec} -- Perform a single TPV simulation}
		
		\textit{Inputs:} 1) a single symmetric (\textsl{i.e.} open circuit) structure as created by \texttt{pindrift};
		2) a first guess for maximum time which should be enough for simulating the full decay, if zero is provided a value of \SI{1e-5}{\s} is used and no further exploration of the simulation time is performed (this is useful when we are just interested in the amount of voltage variation, rather than in the full decay);
		3) intensity of the perturbing light pulse.
		
		\textit{Outputs:} a structure with a solution being perturbed by a light pulse.
		
		\textit{Requires:} \texttt{pindrift}, \texttt{verify\-Stabilization}.
		
		\textit{Required by:} \texttt{TPVconst\_full\_exec}, \texttt{TPVvariab\_full\_exec}.
		
		

		
%		\textit{Required by:} \texttt{}.
		
		\paragraph{\texttt{TPV\_full\_analysis} -- Plot TPV}
			Both mono and bi exponential fitting results are plotted \textsl{versus} light-bias.
			
				\textit{Inputs:} a structure containing the most important results of the \gls{tpv} simulation.
		
%		\textit{Outputs:} .
		
%		\textit{Requires:} \texttt{}.
		
		\textit{Required by:} \texttt{TPVconst\_full\_exec}, \texttt{TPVvariab\_full\_exec}.
		
		\paragraph{\texttt{TPV\_single\_analysis} -- Fit voltage decays from single TPV solutions using MultiStart and do graphics}
		Fit the voltage \textsl{versus} time profile of the provided perturbed solution with a mono\hyp{}exponential and a bi\hyp{}exponential formula.
		Use Matlab's MultiStart from Global Optimization Toolbox in order to find the best bi\hyp{}exponential fitting avoiding local minima.
		An initial guess from a previous fitting can be provided in input.
		
		\textit{Inputs:} 1) a structure with a solution being perturbed by a light pulse;
		2) optional, initial guess for linear factor of fast decay of bi-exponential fit;
		3) optional, initial guess for life-time of fast decay of bi-exponential fit;
		4) optional, initial guess for linear factor of slow decay of bi-exponential fit;
		5) optional, initial guess for life-time of slow decay of bi-exponential fit.
		
		\textit{Outputs:} 1) steady state voltage being present before the pulse;
		   2) maximum voltage variation due to the pulse;
		   3) linear factor of mono-exponential fit;
		   4) life-time of mono-exponential fit;
		   5) linear factor of fast decay of bi-exponential fit;
		   6) life-time of fast decay of bi-exponential fit;
		   7) linear factor of slow decay of bi-exponential fit;
		   8) life-time of slow decay of bi-exponential fit.
		
%		\textit{Requires:} \texttt{}.
		
		\textit{Required by:} \texttt{TPVconst\_full\_exec}, \texttt{TPVvariab\_full\_exec}.
		
		\paragraph{\texttt{TPV\_single\_analysis\_noMultiStart} -- Fit voltage decays from single TPV solutions and do graphics}
		Fit the voltage \textsl{versus} time profile of the provided perturbed solution with a mono\hyp{}exponential and a bi\hyp{}exponential formula.
		Try various starting point for helping the convergence of bi\hyp{}exponential fitting.
		An initial guess from a previous fitting can be provided in input.
		
		\textit{Inputs:} 1) a structure with a solution being perturbed by a light pulse;
2) optional, initial guess for linear factor of fast decay of bi-exponential fit;
3) optional, initial guess for life-time of fast decay of bi-exponential fit;
4) optional, initial guess for linear factor of slow decay of bi-exponential fit;
5) optional, initial guess for life-time of slow decay of bi-exponential fit.

\textit{Outputs:} 1) steady state voltage being present before the pulse;
2) maximum voltage variation due to the pulse;
3) linear factor of mono-exponential fit;
4) life-time of mono-exponential fit;
5) linear factor of fast decay of bi-exponential fit;
6) life-time of fast decay of bi-exponential fit;
7) linear factor of slow decay of bi-exponential fit;
8) life-time of slow decay of bi-exponential fit.
		
%		\textit{Required by:} \texttt{}.
		
		%
		%
		%\subsection{Impedance Spectroscopy in Time Domain}
		%\epigraph{\textit{"I thought you could implement this"\\"Ehm\dots Do you mean\dots by tomorrow?"\\"That would be amazing!"}}
		%
		%\subsection{Impedance Spectroscopy in Frequency Domain}




	\subsection{ElectroAbsorption}\label{software_electroabsorbance}
		\epigraph{First Law of Software Quality:\\
			errors = more code\textsuperscript{2}\\
			$e=mc^2$}

The public release of these routines is waiting for the publication of the related work.
We refer to ElectroAbsorption or Stark spectroscopy as the study of the dependency of the absorbance on the oscillating voltage applied on a complete device with semi\hyp{}transparent electrodes.
We simulate the whole $p$-$i$-$n$ stack but consider the absorption just in the intrinsic layer.
Clearly, the Stark effect has to be evaluated locally at any position through the perovskite layer.
The attention has been focussed on the first and the second harmonic effects, respectively with the same and the double frequency as the applied voltage.

The first harmonic component is driven by the product of the steady state electric field $E_|DC|(x)$ and the amplitude of the first harmonic of the oscillating electric field $|E_|AC||$ at each spatial point.
The second harmonic component is related to the square of the oscillating electric field amplitude $|E_|AC|^2|$ at each spatial point.

%		photo induced absorption stark spectroscopy on perovskite solar cells \cite{Roiati2014,Li2016}


\paragraph{\texttt{EA\_full\_exec} -- Perform ElectroAbsorption simulation on all the provided solutions}
Similarly to the impedance simulation performed with \texttt{ISwave\_full\_exec}, this routine applies an oscillating voltage on all the provided solutions.
One relevant difference is the fact that the current calculation is disabled, making the analysis faster.
If Matlab's Parallel Computing Toolbox is available, this scripts runs in parallel on all the available physical CPU cores, otherwise it just runs on one core.

\textit{Inputs:} 1) can be a cell structure containing structures at various background
light intensities. This can be generated using \texttt{gen\-Int\-Structs}.
Otherwise it can be a single structure as created by \texttt{pindrift};
2) higher frequency limit;
3) lower frequency limit;
4) number of frequency points to simulate;
5) voltage oscillation amplitude in volts, one mV should be enough;
6) logical, after stabilization sets the mobility of
ionic defects to zero;
7) logical, whether to graph the individual solutions and
the overall graphics.

\textit{Outputs:} a structure containing the most important results of the simulation.

\textit{Requires:} \texttt{asymmetricize}, \texttt{ISwave\_EA\_single\_exec},
\texttt{EA\_single\_analysis}, \texttt{EA\_full\_analysis\_phase}, \texttt{pinana}, \texttt{EA\_full\_analysis\_Efield}.

%		\textit{Required by:} \texttt{}.

\paragraph{\texttt{EA\_single\_analysis} -- Calculate ElectroAbsorbance and phase}
The first and second harmonics are locally calculated and averaged over the device thickness.

\textit{Inputs:} 1) a structure where a solution has been perturbed by an
oscillating voltage, as generated from \texttt{ISwave\_EA\_single\_exec};
2) logical, when true graphics does not get created, useful when
launched under parallelization (where graphics does not work anyway);
3) logical, whether to fit the position resolved data;
4) logical, get phase via demodulation instead of using a fitting;
5) string, set the directory where to save \texttt{.fig} and \texttt{.png}
image files for second harmonic versus position.
To save in the current directory set the dot directory \texttt{"."}.
To disable saving set \texttt{"missing"} value.

\textit{Outputs:} 1) an array of bias, amplitude and phase of the first
harmonic;
2) a matrix of bias, amplitude and phase of the first
harmonic, each column is a space point;
3) an array of bias, amplitude and phase of the first
harmonic for ionic profile;
4) a matrix of bias, amplitude and phase of the first
harmonic for ionic profile, each column is a space point;
5) an array of bias, amplitude and phase of the second
harmonic;
6) a matrix of bias, amplitude and phase of the second
harmonic, each column is a space point;
7) an array of bias, amplitude and phase of the second
harmonic for ionic profile;
8) a matrix of bias, amplitude and phase of the second
harmonic for ionic profile, each column is a space point;
9) the value of the average squared amplitude of the
AC electric field. It would be identical to the amplitude of the
second harmonic if the contributions from each spatial point was in
phase;
10) a matrix of bias, amplitude and phase of the AC electric
field, each column is a space point.

\textit{Requires:} \texttt{ISwave\_EA\_single\_fit}, \texttt{ISwave\_EA\_single\_demodulation}.

\textit{Required by:} \texttt{EA\_full\_exec}.



		\paragraph{\texttt{EA\_full\_analysis\_Efield} -- Plot electric field amplitude from ElectroAbsorption in a range of background light intensities or applied DC voltages}
		The amplitude of the spatially averaged electric field first harmonic $(x_2-x_1)^-1 \int_{x_1}{x_2}E_|DC|(x) \cdot E_|AC|(x) \dd x$ and second harmonic $(x_2-x_1)^-1 \int_{x_1}{x_2} E_|AC|^2(x) \dd x$ are plotted \textsl{versus} the applied voltage frequency.
		
				\textit{Inputs:} a structure containing the most important results of the EA simulation.
		
%		\textit{Outputs:} .
		
%		\textit{Requires:} \texttt{}.
		
		\textit{Required by:} \texttt{EA\_full\_exec}.
		
		
		\paragraph{\texttt{EA\_full\_analysis\_phase} -- Plot phase Bode plots from EA in a range of background light intensities or applied DC voltages}
		Plots the first and second harmonic phase shift with regards to the applied voltage.
		
		\textit{Inputs:} a structure containing the most important results of the EA simulation.
		
%		\textit{Outputs:} .
		
%		\textit{Requires:} \texttt{}.
		
		\textit{Required by:} \texttt{EA\_full\_exec}.
		

%		\subsection{Techniques Which Could Be Implemented}
%		\paragraph{IMVS}
%		\paragraph{IMPS}
%		\paragraph{Mott-Schottky}
		%\url{https://en.wikipedia.org/wiki/Mott-Schottky_plot}
		%mott-schottky in organics under illumination 10.1103/PhysRevApplied.7.034018
		%impedance and mott-schottky article in osc \cite{Brus2016}		
		%dyakonov sobre mott-shottky 10.1021/acsaem.8b01119




%\pagebreak[2]
\newpage
\mysection[PyPV]{PyPV: An easy Current-Voltage curves acquisition interface}
	\epigraph{\textit{\enquote{Why nobody tried to fix this bug?}}}

	\subsection{Introduction}
		The route to a reliable and efficient device is inevitably a long iterative process.
		In each optimization step a device is fabricated, measured, and the design modified for the next step.
		Part of the evaluation is just a repetitive process which can be automatized without any loss.
		Current-voltage sweeps is the most frequently employed characterization technique for solar cells, for this reason a fast and easy data acquisition software has been developed.
		The complete software can be downloaded from \url{https://github.com/ilario/PyPV}.

		\begin{SCfigure}%[!hbtp]%
			\centering
			\includegraphics[width=0.7\textwidth]{old_iv_software/old_iv_software.png}
			\mycaption[Legacy software for current-voltage sweep measurement.]{}\label{fig:old_iv_software}
		\end{SCfigure}

		\paragraph{History of the project}
		As a proof of concept, Dr.\ Daniel Fernandez Pinto developed in 2015 a Python library and a small \gls{gui} for communicating with Tektronix Keithley 2400 equipment \textsl{via} NI-VISA \cite{NationalInstruments2019} and running basic operations like current\hyp{}voltage sweeps.
		I went on adding functions and expanding the \gls{gui} until obtaining a fully functional alternative to the legacy LabView software shown in \cref{fig:old_iv_software}.
		In comparison to the mentioned legacy software, PyPV has the advantage of not needing any commercial license which allowed us to use it on various computers.
		Additionally some of the features which will be described below were just missing in the previous software.
		As far as I know, PyPV is currently used in the following research institutions: ICIQ (Institut Català d’Investigació Química), URV (Universitat Rovira i Vrigili), İzmir Katip Çelebi Üniversitesi, and Karamanoglu Mehmetbey University.

		%I received a proof-of-concept software developed by  and decided to continue the development. At that point the software had an interface with few buttons and a working Keithley communication library.

		\paragraph{User's requests}
		The feedback from the new users of the software helped me improving PyPV with the following features: solar simulator shutter control, more robust connection to the GPIB-USB-HS adapter, better installation instructions.

	\subsection{Implementation and user interface}
		\epigraph{\textit{\enquote{Ok, I finally completed it, by the way, why did you start developing this?}\\\enquote{Well, it was just a proof of concept, but it's nice you worked on it}}}

		Python~2, rather than the most recent Python~3, has been chosen for the software development.
		This apparently stupid choice was motivated by the requirement of running the software on the obsolete Windows~XP machines still used for the data acquisition.
		
		\begin{sidewaysfigure}
								\centering
			\includegraphics[width=0.9\textwidth]{PyPV/pypv-gui.png}
			\mycaption[PyPV graphical user interface.]{}\label{fig:pypv-gui}
		\end{sidewaysfigure}

		\paragraph{Organisation of the main user interface}
		The components of the graphical user interface are organised with the following order (from top left to bottom right):
		\begin{itemize}
			\item user and experiment identification;
			\item experiment setup;
			\item quick screening buttons;
			\item sample identification and single measurement setup;
			\item single measurement output;
			\item measurement buttons;
			\item saved measurements log, from newest to oldest.
		\end{itemize}
		
		
		\paragraph{User and experiment identification data}
		The user name is prompted, this will be stored in the savings header and, if the "AutoSave" or "AutoMeasure" function is used, a directory named after the user name will be created.
		The experiment codename is prompted for being prepended in the savings suggested and automatic filenames.
		The active area is also prompted at this point.

		\paragraph{Voltage range and step}
		Then, the lowest and highest voltage limits ((\textsl{i.e.} \texttt{:SOURce:VOLTage:STARt} or \texttt{:SOURce:VOLTage:STOP}) are prompted, together with the voltage step (\textsl{i.e.} \texttt{:SOURce:VOLTage:STEP}).
		The number of measured points (used for the buffer size in \texttt{:TRACe:POINts} and for the trigger count in \texttt{:TRIGger:COUNt}) can be calculated dividing the voltage range by the voltage step.
		
		\paragraph{Compliance}
		This introduces a maximum limit to the current that can flow in the connected device, in order to avoid overheating and related damages to the solar cell.
		In order to respect this limit, Keithley equipment reduces the applied voltage (this does not get reflected in the saved output, where the capped regions appears simply as a plateau).
		The compliance value (\textsl{i.e.} \texttt{:SENSe:CURRent:PROTection}) has to be smaller than the chosen scale.
		
		\paragraph{Scale and "Auto Scale"}
		As it was explained in \cpageref{autoscale}, the "automatic scale" feature of the Keithley is detrimental for perovskite solar cells dynamic (like a current\hyp{}voltage sweep) measurements.
		For this reason a check and a combo box have been added for enabling\-/disabling automatic scale and setting the manual scale (\textsl{i.e.} \texttt{:SOURce:CURRent:RANGe}).
		
		\paragraph{Integration time, delay time, and scan speed calculation}
		Integration time (\textsl{i.e.} \texttt{:SENSe:VOLTage:NPLCycles}) is the number of power line cycles during which each current point is measured \cite{KeithleyInstruments2011}.
		For example, on a \SI{50}{\Hz} electrical network, an integration time of 1 means that the current is measured during \SI{20}{\ms} for each voltage point.
		Delay time (\textsl{i.e.} \texttt{:SOURce:DELay}) is the time waited for the current stabilisation after the voltage step.
The scan speed $s[V/s]$ is obtained from the voltage step \texttt{:SENSe:VOLTage:STEP} $V_|step|[V]$, integration time $i$ and the delay time \texttt{:SOURce:DELay} $t_|delay|[s]$ Keithley's parameters with the following \textit{empirical} expression estimated by Werther Cambarau:

\begin{equation}
s = \frac{V_|step|}{0.003 + t_|delay| + 0.06 \cdot i}
\end{equation}

where $t_|delay|$ is \SI{1}{\ms} for our measurement conditions.

\paragraph{Shutter control}
If the employed solar simulator has a remotely controlled shutter, this should be possible to connect to the RS-232 serial port on Keithley.
Two parameters control the opening and closing of the illumination shutter: "pre-delay off" and "pre-delay on".
The former indicates how long the illumination should be off, after this, the shutter gets opened and the latter time is waited before running the sweep.
The off time is mainly useful when using the "AutoMeasure" feature, which will be mentioned further in the text.
During all these steps, the device is kept at open circuit.
If the parameters are set to zero, the shutter control is disabled.

\paragraph{Quick screening buttons}
Open circuit voltage real time measurement is available.
This helps the user to have a quick idea of which diodes are worth measuring.

\paragraph{Irradiance, device, and diode identification}

		\paragraph{Auto-measure}\label{automeasure}
		The devices are kept under illumination at open circuit conditions, then a reverse sweep is measured and, immediately after this, the forward sweep is measured.
		The reason for measuring the reverse sweep first, is that the reverse sweep starts from a high voltage point which is closer to the stabilized condition of the devices prior to the measurement (illuminated and open circuit).


		\paragraph{Resistances} The shunt and series resistances estimation from current-voltage sweeps have been implemented but should not be considered for measurements on hysteretic devices, as explained in \cpageref{resistances}.


Numeric up-down

\begin{SCfigure}
	\centering
	\includegraphics[width=0.7\textwidth]{PyPV/pypv-plot.png}
	\mycaption[Current\hyp{}voltage plot window of PyPV.]{The plot window includes the main performance data and can be manually saved as an image.}\label{fig:pypv-plot}
\end{SCfigure}

\paragraph{Additional output}

\begin{figure}
	\makebox[\textwidth][c]{
		\parbox{1.1\textwidth}{
			\centering
			\begin{subfigure}[t]{0.61\textwidth}
				\includegraphics[width=1\textwidth]{PyPV/pypv-console.png}
				\subcaption{Console output}\label{fig:pypv-console}
			\end{subfigure}
			\qquad
			\begin{subfigure}[t]{0.41\textwidth}
				\includegraphics[width=1\textwidth]{PyPV/keithley.jpg}
				\subcaption{Keithley display output}\label{fig:pypv-keithley}
			\end{subfigure}
					\mycaption[Additional output of PyPV.]{In (\textbf{a}) the textual output from console is shown.
						The printed information is usually not important for normal software usage but can help for debugging.
						In (\textbf{b}) instructions to the user are delivered \textsl{via} Keithley's display and buzzer.
				}\label{fig:pypv-other}
		}
	}
\end{figure}

\paragraph{Directories and files naming}
When using the "AutoSave" of the "AutoMeasure" functions, the following directory structure is created, if not already existing: \texttt{user name\-/YYYY-MM-DD} where the subdirectory is the measurement date in little endian format.
Inside this directory, the measurements are named with the following schema: \texttt{experiment name-device name-\textit{<diode>}-\textit{<direction>}.txt} where \textit{\texttt{<diode>}} is the identification number of the specific diode under measurement, in case there is more than one per substrate (\textsl{e.g.} in our case, each device has four independent diodes, as can be seen in \cref{fig:device}) and \texttt{\textit{<direction>}} can be either \texttt{forward} or \texttt{reverse}, as explained in \cpageref{characterization_fwdrev} for the sweep scan direction.
In order to improve the compatibility with different file systems, the accents in the user name, experiment name, and device name fields are removed using \texttt{unicode\-data.\-normalize} function.

\paragraph{Files content}
In the txt files, the first 22 lines include some information about: user, date and time, experiment name, device name, diode number, scan direction, lowest and highest sweep voltage, voltage step size, compliance (maximum current), scale, calculated voltage and current density of maximum power point, integration time and the related scan speed, delay time for each measured point, pre-delays for closed and opened illumination shutter, calculated series resistance (from the high voltage points) and shunt resistance (from close\hyp{}to\hyp{}zero voltage points), active area, irradiance (illumination intensity in \si{\mW\per\square\cm}), calculated \gls{jsc}, \gls{voc}, fill factor, and \gls{pce}.
The \nth{23} row include the header for the data.
From the \nth{24} row on, two columns with applied voltage and measured current follow, separated by a tabulation.
The rows are terminated by a \texttt{\n} newline, following the Unix standard.

\paragraph{Configuration saving}
Usually, the setup does not change between experiment and experiment.
To reduce the risk of setting wrongly some parameters, some of them are saved in the \texttt{last\_configuration.lnk} file at the program's closing and restored at its opening.
The parameters which get saved are: user name, experiment name, lower and higher voltages, voltage step size, compliance value, scale, integration time, and cell area.

\paragraph{Logging and recovery files}
The software execution dates and the main parameters of all the performed measurements are stored in a log file: \texttt{measurements\_log.txt}.
In case the user closes the software forgetting to save the last measurement, the data is also saved in two temporary files: \texttt{last\_measurement\_raw.txt} and \texttt{last\_measurement.txt} containing respectively the raw and the processed measurement output.

	\subsection{Limitations}

		The interface development has been started with the "Monkey Studio" software, which development has ceased even before the start of PyPV. This demonstrated to be a big failure in long term development planning.

%\pagebreak[2]
\newpage
\section{Robust and quick data analysis \textsl{via} R scripts}
	\epigraph{\textit{\enquote{Is there an Origin version for Linux?}\\\enquote{No}}}

	\subsection{Introduction}
		To process the large amount of data which can be obtained using multiple techniques during the characterization of solar cells can be a very tedious and repetitive work.
		I developed routines for data processing and representation, like fittings, integrations and plots.
		I am describing the routines here in order to allow others to use and modify them.
		The described routines have been tested to work on Linux and Windows systems.
		All of the described functions can be downloaded from \url{https://github.com/ilario/photophysics-data-processing-R}.

	\subsection{Helpers}
		\paragraph{\texttt{limits\_for\_graphics.R} -- Set graphical parameters}
		In this file I tried to centralize the parameters which should be edited by the user and will be used by the scripts.
		This is for avoiding the need to edit "hardcoded" values in the middle of complex scripts.
		The parameters which can be customized here are: device active area, limits for the axes, string to prepend to the file names, switch between raster or vector image output, resolution of raster images output, size of vector image output.

		\paragraph{\texttt{devices-identification.R} -- Define grouping of devices}
		The devices are labelled with an alphanumeric codename, which is not insightful at the time of categorizing them for a statistical analysis.
		In this file the samples codes are grouped depending on one or two characteristics.
		Up to now, this system is just employed by \texttt{iv-generate\_mydata\_with\_comments.R} for adding supplementary information to the extracted parameters and by \texttt{iv-generate\_statistics.R} for plotting the boxplots separating the samples by kind.

		\paragraph{\texttt{run-all-photophysics.R} -- Runs routine data analysis on one device's data}
		An extremely basic but multiplatform \gls{gui} has been implemented.
		Running it, the data folder gets prompted.
		By default, all the data analysis routinely performed in Palomares group is run.
		It can be edited to unselect the non-needed data analysis routines.
		
		\paragraph{\texttt{run-all-photophysics-comparisons.R} -- Plots comparisons between devices}
		This basic \gls{gui} prompts the user for the directories containing the data of each of the devices to be compared.
		Then it plots the characterization data of all the selected devices in single plots, in order to help the visual comparison of the results.
		
		%\subsection{Photophysical Characterization \textsl{via} Optical and Electrical Transients}


	\subsection{Charge Extraction (\glsentryshort{ce})}\label{r_ce}

		\paragraph{\texttt{from\_ce\_to\_table.R} -- Converts CE output}
		The output from \gls{ce} measurement software is rather minimal, as it lacks the time column.
		This script generates a two column file output which is more convenient for plotting with other commonly employed software.
		It works both for the TPV and for the TPC outputs.
		
%		\paragraph{\texttt{ce.R} -- Integrates CE transients directly}
%		\paragraph{\texttt{ce-subtractDark.R} -- Integrates CE transients after subtracting a transformed noise profile}
%		\paragraph{\texttt{ce-integrateExp.R} -- Integrates a fit to CE transients}
%		\paragraph{\texttt{ce-from\_output\_to\_graph.R} -- Plots the extracted charge \textsl{versus} light bias}
%		\paragraph{\texttt{ce-from\_many\_output\_to\_graph.R} -- Compares between CE of various devices}
%		\paragraph{\texttt{ce-time-many.R} -- Compares single CE transients of various devices}
%		\paragraph{\texttt{ce-with\_limits.R} -- Compares CE transients with RC decays}

	\subsection{Transient PhotoVoltage (\glsentryshort{tpv})}\label{r_tpv}

		\paragraph{\texttt{from\_tpv\_tpc\_to\_table.R} -- Converts CE output}
		Similarly to \texttt{from\_ce\_to\_table.R}, this converts the one column raw file to a two columns text file.

%		\paragraph{\texttt{tpv.R} -- Fits TPV decays}
%		
%				\paragraph{\texttt{tpv\_onlyDeltaV.R} -- Gets the voltage increase from TPV decays}
%				
%		\paragraph{\texttt{tpv-from\_output\_to\_graph.R} -- Plots TPV life\hyp{}times \textsl{versus} light bias}
%		
%		
%\paragraph{\texttt{tpv-from\_output\_to\_graph-with\_limits.R} -- Compares life\hyp{}time with RC time}
%
%		\paragraph{\texttt{tpv-from\_many\_output\_to\_graph.R} -- Compares between TPV of various devices}
%
%	\subsection{Transient PhotoCurrent (\glsentryshort{tpc})}\label{r_tpc}
%
%		\paragraph{\texttt{tpc.R} -- Integrates TPC transients}
%		\paragraph{\texttt{tpc-vs-tpv-vs-ce.R} -- Compares the single transients from various techniques}
%
%
%	\subsection{Differential Capacitance (\glsentryshort{dc})}\label{r_dc}
%
%		\paragraph{\texttt{dc-from\_output\_to\_graph.R} -- Calculates and plots capacitance and charge from DC}
%		
%		\paragraph{\texttt{cedc.R} -- Compares the charge \textsl{versus} light\hyp{}bias obtained from CE or DC}
%		
%		\paragraph{\texttt{dc-from\_many\_output\_to\_graph\_capacitance.R} and \texttt{dc-from\_many\_output\_to\_graph\_charge.R} -- Compares between DC of various devices}
%		
%
%	\subsection{Transient PhotoVoltage referred to charge from CE or DC (\glsentryshort{tpvce} and \glsentryshort{tpvdc})}\label{r_tpvcedc}
%
%
%
%		\paragraph{\texttt{tpvce-from\_output\_to\_graph.R} -- Relate TPV life\hyp{}times to CE charge}
%		
%				\paragraph{\texttt{tpvce-from\_many\_output\_to\_graph.R} and \texttt{tpvdc-from\_many\_output\_to\_graph.R} -- Compare TPV life\hyp{}times related to either CE or DC charge of various devices}
%
%	\subsection{Current-Voltage Sweeps}
%		\paragraph{\texttt{extractdata-curves-vi-separated\_files.R} -- Functions for extracting parameters from current\hyp{}voltage sweeps¨}
%		\paragraph{\texttt{iv-generate\_mydata.R} and \texttt{iv-generate\_mydata\_with\_comments.R}}
%		\paragraph{\texttt{iv-generate\_images.R}}
%		\paragraph{\texttt{iv-generate\_statistics.R}}
%		\paragraph{\texttt{iv-jsc\_voc\_vs\_light\_intensity.R}}
%		\paragraph{\texttt{iv-plot\_list.R}, \texttt{iv-plot\_list-suppinfo.R}, and \texttt{iv-light\_intensity.R}}
%
%		\paragraph{\texttt{jsc-stability.R} and \texttt{voc-stability.R}}
%
%	\subsection{Other}
%	
%				\paragraph{\texttt{jrec-ce.R} and \texttt{jrec-dc.R} -- Reconstructs the \gls{jsc}}
%					\cpageref{jsc_reconstruction}
%					
%		\paragraph{\texttt{mppt\_to\_graph.R} -- Plots tracking of maximum power point}
%		\paragraph{\texttt{profilometry.R} -- Plots profilometer measurements and fit heights}
%		\paragraph{\texttt{tas-downsample.R} -- Reduces the dimension of TAS decays}
%%		\paragraph{\texttt{tas-graph.R} -- Plots TAS decays}
%		\paragraph{\texttt{uvvis-pl.R} -- Plots absorbance and photoluminescence spectra and fits Tauc plots}

\section{Contributions to mppTracker}\label{software_mppt}
	\epigraph{\textit{\enquote{A student of mine sells a complete system for that, just go and buy it}}}

	As seen in \cpageref{characterization_hysteresis}, a \gls{mppt} system ensures to obtain a \gls{pce} value closer to the real world working conditions \cite{Zimmermann2016}.
	The hysteretic phenomena in perovskite solar cells requires new procedures for \gls{mppt} \cite{Cimaroli2017,Pellet2017} and a few commercial systems are already available for that \cite{CicciResearchsrl2019,CandlelightSystems}.
	An interesting project for having robust \gls{mppt} measurable with Tektronix Keithley 2400 is mppTracker \cite{AFMD2017}.
	I made a few contributions which are accessible on \url{https://github.com/AFMD/mppTracker/pulls?q=is%3Apr}.
	As the effort in this project were very limited on my side, I'll not further describe that.
	If confirmed, recent reports on "bistable short circuit current" in perovskite solar cells by \authoryear{Calado2018} would further complicate the implementation of a \gls{mppt}.
%	introduce local maxima making the \gls{mppt} system design even more complex.
