%\epigraph{\textit{"AAAAAAAAAAAAAAAAAAA"}}

In this thesis, perovskite solar cells have been fabricated and characterised by means of advanced techniques.
These techniques have been critically analysed taking advantage of drift\hyp{}diffusion modelling.
The provided contributions have widened the understanding of perovskite solar cells characterisation output, allowing the scientific community to get more information from the already available techniques.
This knowledge will help in the identification of performance bottlenecks of photovoltaic devices, easing the solar cells optimization.

The main findings can be summarised as follows:
\begin{enumerate}
	\item The advanced characterisation output has to be interpreted with great care and the theoretical framework from previous type of photovoltaic devices is no longer enough for perovskite solar cells.
	In \cref{ch:characterization} I described and critically analysed the employed characterisation techniques.
	For each, the strategy for noise reduction is explained and compared with the tested alternatives.
	Taking advantage of the insight obtained from drift\hyp{}diffusion modelling the following hypotheses are thrown:
	time resolved photoluminescence will also evolve with time due to slow ionic profile adaptation to the new illumination conditions;
	charge extraction experiments should show a long lasting ionic displacement current giving information about ionic capacitance rather than geometric capacitance;
	I confirm that the exponential trend in charge extraction can be explained with the chemical capacitance and should start at light biases close to the built-in voltage;
	the fast component of bi\hyp{}exponential decays in transient photovoltage observed at low background light intensity could be due to ionic profile update before the generated charge recombines.
	
	\item The difference between the contacts' energy levels provides an unreliable estimation of the built-in voltage of perovskite solar cells.
	In \cref{ch:tae} and in \cite{Gelmetti2019} we have fabricated bottom cathode devices with four different \gls{htm} ensuring to fulfil the conditions for having a fair comparison.
	Comparing the built-in voltage of the different solar cells using \gls{ce} and \gls{dc} techniques we realised that the involved energy levels are rather different than the values expected from cyclic voltammetry measurements.
	After excluding major differences in the carriers life-times in the four devices, we have measured the \textsl{in situ} work function of the \gls{htm}.
	We have observed significant deviations from the substrate value just for some of the \gls{htm} and used this result for explained the measured \gls{voc} and built-in voltage.
		
	\item In top cathode \gls{ito}\-/\gls{pedotpss}\-/\gls{mapi}\-/\gls{pcbm70}\-/\ch{Ag} perovskite solar cells, the photogenerated charge at low light intensity and at open circuit conditions is storage location has been observed.
	In \cref{ch:thicknesses} and in \cite{Gelmetti2017} we have shown that the holes are accumulated where expected, which is in a depletion layer in the \gls{pedotpss} at the interface with the perovskite layer.
	Regarding the electrons, we have demonstrated that the storage location is not at the \gls{pcbm70} interface with the perovskite, rather it can be close to the metallic electrode\-/selective contact interface or through the whole \gls{etm} layer.
	
	\item Giant and negative capacitance observed in perovskite solar cells \textsl{via} impedance spectroscopy do not involve a giant charge accumulation.
	In \cref{ch:impedance} and in \cite{Moia2019} we showed that they rather stem from the influence of the ionic movement respectively on the recombination and on the injection barriers.
	Additionally, the whole apparent capacitance spectra is explained taking advantage of the drift\hyp{}diffusion results. 
	The simulation method and modular structure is explained in detail.
\end{enumerate}