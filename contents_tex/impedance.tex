\epigraph{\textit{"We don't see that plateau"}}


\paragraph{Abstract}

\paragraph{Publications} Part of this chapter has been published in \fullcite{Moia2019}.



%Our recent drift-diffusion simulations indicate that the electrostatic potential arising from mobile ion redistribution controls electronic charge transfer processes such as recombination in mixed conducting lead halide perovskite devices and can give rise to apparent capacitive behavior.[1] The capacitive features of a mixed conductor are traditionally described within equivalent circuit models by a chemical capacitance, which accounts for bulk polarization effects.[2,3] and scales with the thickness of the device, complemented by an interfacial capacitance, related to electronic and ionic accumulation at the interfaces with the contacts. Since the effect described is not simply associated with an accumulation or depletion of charges in the active layer, it cannot be adequately described with a conventional capacitor. We identify the bipolar transistor as a most appropriate circuit element to reproduce analytically the observation of ionically-gated electron transfer at interfaces and include it into an equivalent circuit model that accounts for and couples ionic and electronic carrier dynamics.[1] We propose that the huge capacitive and/or inductive behavior[4] observed for thin film perovskite solar cells is best described by electronic currents that are in fact an amplified version of the ionic current flowing in the bulk of the active layer. The resulting model yields good fits to the experimental impedance spectra as a function of applied voltage bias and light intensity, and can reproduce large perturbation transient measurements such as hysteretic current-voltage characteristics.[5] It also enables to evaluate parameters such as the ionic conductivity of the hybrid perovskite, the observed ideality factor of the device and the properties of the space charge within the active layer and in the contact regions. The limits under which the present model coincides with the traditional picture are discussed.
%References:
%[1] Moia, D.; Gelmetti, I.; Calado, P.; Fisher, W.; Stringer, M.; Game, O.; Hu, Y.; Docampo, P.; Lidzey, D.; Palomares, E.; et al. Ionic-to-Electronic Current Amplification in Hybrid Perovskite Solar Cells. arXiv:1805.06446v2 [physics.app-ph] 2018.
%[2] Senocrate, A.; Moudrakovski, I.; Kim, G. Y.; Yang, T.-Y.; Gregori, G.; Grätzel, M.; Maier, J. The Nature of Ion Conduction in Methylammonium Lead Iodide: A Multimethod Approach. Angew. Chem. Int. Ed. Engl. 2017, 56 (27), 7755–7759.
%[3] Jamnik, J.; Maier, J. Generalised Equivalent Circuits for Mass and Charge Transport: Chemical Cpacitances and Its Implications. Phys. Chem. Chem. Phys. 2001, 3 (9), 1668–1678
%[4] Almora, O.; Cho, K. T.; Aghazada, S.; Zimmermann, I.; Matt, G. J.; Brabec, C. J.; Nazeeruddin, M. K.; Garcia-Belmonte, G. Discerning Recombination Mechanisms and Ideality Factors through Impedance Analysis of High-Efficiency Perovskite Solar Cells. Nano Energy 2018, 48 (February), 63–72.
%[5] Snaith, H. J.; Abate, A.; Ball, J. M.; Eperon, G. E.; Leijtens, T.; Noel, N.; Stranks, S. D.; Wang, J. T. W.; Wojciechowski, K.; Zhang, W. Anomalous Hysteresis in Perovskite Solar Cells. J. Phys. Chem. Lett. 2014, 5 (9), 1511–1515

\section{Large perturbations}\label{impedance-large_perturbations}




\section{Measured and simulated impedance spectra characteristics}
\section{Ionically gated interfacial transistor}
\section{Ionic-to-electronic current amplification}
\section{Capacitor-like and inductor-like behaviour}
\section{Critical Assessment}
