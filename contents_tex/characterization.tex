\epigraph{\textit{""}}

\paragraph{Abstract} Characterization of perovskite solar cells is a non-trivial subject, the techniques researchers successfully employed for \gls{osc} and \gls{dssc} needs to be re-validated for this new kind of solar cells.
The presence of ionic migration in the absorber can be a game-changer for which special care has to be taken.

\section{Conventions and General Remarks}

	All the characterization on complete devices was performed keeping them in an air-tight holder filled with nitrogen.
	The electrical connection from the cell electrode to the external end of the holder was obtained using gold tips connected \textit{via} a printed circuit board to a coaxial cable.

	\subsection{Sign Convention and Parameters Definitions}

		\paragraph{Fermi level} The electrons electrochemical potential, also known as Fermi level, is defined as the energy required for adding an electron in a specified position.
		Its value depends on the electrostatic potential $V_E$ in that position and on the internal chemical potential $\mu$ which in our case depends mainly on the concentration of electrons (not related to their electric charge, similarly to the density of a gas).
		As the Fermi level is going to be used mainly for comparisons, its zero is not going to be defined thesis-wide, instead it will be defined to a convenient reference just where needed.

		\paragraph{Cathode and anode} Considering a solar cell device at steady state under illumination and in open circuit conditions, its cathode is defined as the contact where the electrons electrochemical potential $\bar\mu$  is the highest.
		By consequence the other contact is the anode.
		The naming of the two contacts holds to the one defined in illuminated, open circuit conditions even in conditions where the contacts' electrochemical potential is in the reversed order.

		\paragraph{Voltage} The voltage $V$ is always used as a relative value, defined subtracting the electrons electrochemical potential of the cathode from the anode's one.
		So in the aforementioned solar cell example, the voltage is positive.
		The unit is the Volt.

		\paragraph{Electrical power} The electrical power $P$ is defined as positive when the device absorbs electrical energy (incoming, passive element) and negative when it generates energy (outgoing, active element).
		It can be expressed in extensive form with power (Watt) unit or in intensive form "electrical power density" with power over active area unit (Watt over square centimetre).

		\paragraph{Current} The current $J=P/V$ is measured through an external circuit and the sign is a consequence of the voltage and electrical power definition: A current ("conventional current", flow of positively charged particles) being released from the device's anode and being received from the device's cathode is defined as negative.
		This can be thought as: Inside the device, somehow, a positive charge was moved from the high Fermi level contact to the low Fermi level contact, increasing its electrochemical energy, the opposite to what would happen in a resistor, whose current is always positive.
		In a solar cell device, the current and the electrical power can be either positive or negative depending on the illumination and voltage conditions.
		It can be expressed in extensive form with current (Amperes) unit or in intensive form "current density" with current (Amperes) per active area (square centimetre) unit.

		\begin{SCfigure}
			\centering
			\includegraphics[width=0.5\textwidth]{iv_params/IV-revIVs.pdf}
			\mycaption[Parameters extraction from current-voltage sweeps.]{A typical current-voltage sweep is represented.
				MPP stands for maximum power point, \gls{jsc} stands for \glsdesc{jsc}, \gls{voc} stands for \glsdesc{voc}.
				The ration between the small and the large rectangles areas is the \glsdesc{ff} (\gls{ff}).}\label{fig:iv_params}
		\end{SCfigure}

		\paragraph{\Glsdesc{voc}} \Gls{voc} parameter is defined as the voltage $V$ at which the current is zero while the solar cell device is illuminated at 1~sun conditions and in steady state (positive by its own definition).

		\paragraph{\Glsdesc{jsc}} \Gls{jsc} parameter is defined as the unsigned value of the current density flowing in an external circuit short circuiting (zero resistance) the solar cell device's contacts while illuminated at 1~sun and in steady state.
		It is usually reported in current (milli Amperes) over active area (square centimetre) unit.

		\paragraph{Maximum power density} The maximum power density is defined as the unsigned minimum of electrical power density which can be obtained by $P(V) = J(V) \cdot V$.
		It is usually reported using power (Watt) over active area (square centimetre) unit.

		\paragraph{\Glsdesc{pce}} \Gls{pce} parameter is defined as the maximum power density over the illuminating power density, which at 1~sun AM 1.5G is defined to \SI{100}{\mW\per\square\cm}.
		It is usually reported as a percentage.

		\paragraph{\Glsdesc{ff}} \Gls{ff} parameter is defined as the ratio between \gls{pce} and the product of \gls{voc} and \gls{jsc}.
		This parameter does not have a physical meaning, but it represents how much the series and shunt resistances affect the device efficiency.

		\paragraph{Forward and reverse bias} Forward bias is a device condition where the voltage is positive, reverse bias is the case where the voltage is negative.

		\paragraph{Forward and reverse scan} In current-voltage sweeps, a scan where the voltage is increasing over time is a forward scan, while a voltage variation in the opposite direction constitutes a reverse scan.

		\paragraph{Ideality factor} An ideality factor $n_|id|$ different from 1 describes deviations from the ideal photo-diode.
		The Shockley diode equation adapted for photogeneration becomes \cite{Calado2018b}:
		\begin{equation} \label{eq:photodiode}
			J(V,\phi) = J_|ph|(\phi) - J_0\left[\exp(\frac{qV}{n_|id|k_|B|T})-1\right]
		\end{equation}
		where $J_|ph|$ is the total photo-generated current (negative sign), $J_0$ is the dark diode saturation current (the current flowing in dark when applying a reverse bias, negative sign), $q$ is the elementary charge, $k_|B|$ is the Boltzmann constant, and $T$ is the temperature.
		If recombination losses at short circuit are negligible (which can be measured either with $\jsc$ \textit{versus} $\phi$, see \cpageref{jsc-phi}, or with \gls{tpc}, see \cpageref{characterization_tpc}), the photo-generated current can be approximated with the short circuit current $J_|ph|(\phi) \approx \jsc(\phi)$.
		Clearly the reported equation just offers a simplified model.
		For example, it can be improved adding the contribution from the series resistance $R_|s|$ and would become
		$$J = J_|ph|(\phi) - J_0\left[\exp(\frac{q(V+JR_|s|)}{n_|id|k_|B|T})-1\right]$$
		The function is now an implicit one, requiring numerical solving even for obtaining $\jsc$.
		Additionally, we can include the leakage current due to the internal shunt resistance $R_|sh|$ \cite{Tvingstedt2017}:
		$$J = J_|ph|(\phi) - J_0\left[\exp(\frac{q(V+JR_|s|)}{n_|id|k_|B|T})-1\right] + \frac{V+JR_|s|}{R_|sh|}$$
		
		\paragraph{Top and bottom of devices} The point of view of the manufacturer is used for defining the physical top and bottom of a device: the bottom is the glass substrate and the top is the last deposited layer.
		This is opposite with the usage of a solar cell in the real world and with most of the solar simulators (but not all of them, for example Paios from Fluxim has an illumination from below, more convenient for contacting the electrodes without a samples holder \cite{Fluxim}).

	\subsection{Usage of Shadowing Mask}

		\begin{SCfigure}
			\centering
			\includegraphics[width=0.5\textwidth]{shadowing_mask/shadowing_mask.pdf}
			\mycaption[Illuminated area after a shadowing mask.]{This schema is just for explaining the concept described in the text, its dimensions are not realistic.}\label{fig:shadowing_mask}
		\end{SCfigure}

		In literature is generally suggested to use a shadowing mask when measuring the solar cell devices in order to better define the illuminated area (as in the broken \cite{Brinser2017} form from Nature publisher \cite{NatureResearch2017}).
		In our case the active area is just \SI{0.09}{\square\cm} so the mask aperture should be extremely small and its exact positioning troublesome.
		Additionally, the fact that the illumination reaches the mask from a wide angle (the illuminating source dimension, which is not just the lamp as the illumination passes through spread lenses, is not small compared to the lamp-cell distance) allows the light to spread through the substrate glass (\SI{2.2}{\mm} for \gls{fto} substrates, other groups use even thicker glass substrates) reaching a significantly larger area on the active layer at the other side of the glass, as represented in \cref{fig:shadowing_mask}.
		In our solar simulator a linear widening of 8~\% over \SI{2}{\mm} was estimated, this makes an illuminated area 16~\% larger than the mask aperture.
		Even if the total incident power is still determined from the mask aperture, the illumination intensity is not 1~sun any more, compromising the validity of a measurement done with a shadowing mask.


	\subsection{Stability During the Measurement and Small/Large Perturbations}

		Most of the reported hybrid lead halide perovskite materials can show rather impressive changes in their structure on long time scales, for example due to ionic migration \cite{Calado2016}, degradation \cite{OKane2019}, and self-healing \cite{Ceratti2018}.
		This have to be taken into account for all the measurements techniques output which either takes too long time to be measured or employs large perturbations.

		\paragraph{Long lasting measurement} An example of the first case is the impedance spectroscopy where during the long lasting measurement various phenomena can occur, like: a slow current evolution due to perovskite well known hysteretic behaviour prior to stabilization; a degradation process changing the current; the heating of the device changing its properties.
		This slow current evolution can easily be misinterpreted for capacitive current \cite{Jacobs2018} and introduce artefacts like loops in the Nyquist plots \cite{Moia2019}.

		\paragraph{Large perturbations} \label{perturbation}
		Large perturbations regime means that the independent variable being perturbed is changed by an amount large enough to cause the quantity under study to not follow the approximation given by the first term of its series expansion.
		Let's take some example.

		\paragraph{Large perturbations -- \gls{tpv}}
		For example, if too intense, the light pulse in \acr{tpv} could change the voltage by a less-than-linear amount.
		In this case, the light pulse is not only probing the recombination, but it is adding some, so a large perturbation has to be avoided.
		This effect has been reported for \gls{dssc} in \authoryear{Barnes2013}.

		\paragraph{Large perturbations -- impedance}
		Another example: a too wide sinusoidal voltage oscillation amplitude in impedance measurements can cause a non-sinusoidal current output.
		This is not a problem for the measurement itself, as the lock-in amplifiers are perfectly able to extract the amplitude and the phase of the signal first harmonic, ignoring the higher harmonics caused by the too large perturbation.
		But artefacts could arise and cause misinterpretations, as explained in \cpageref{impedance-large_perturbations}.

		\paragraph{Large perturbations -- \gls{trpl}}
		Last example: in the \glsdesc{trpl} a laser pulse illuminates the otherwise unilluminated absorber layer.
		This pulse induces a the migration of the ionic defects to a new profile depending on the pulse intensity \cite{Levine2018}, by a small extent due to its short duration.
		The fact that the relaxation time of the ionic migration is usually much larger than the laser repetition rate (\si{\ms} to seconds for the ions \cite{Jacobs2018} and \si{\ms} to \si{\us} for typical \gls{trpl} lasers \cite{EdinburghInstruments}) implies that the ionic profile variation slowly "builds up" pulse after pulse.
		As the ionic profile affects the free charges concentration, and this in turn rules the radiative recombination, the measurements of \gls{trpl} have to be done with extreme care.
		An example of hysteretic behaviour observed with \gls{trpl} can be found in \authoryear{Motti2016} and elsewhere \cite{Chen2015,Chen2017}.
		This should also be considered when comparing life-times from different laser fluences \cite{Manser2014}.

\section{Current-Voltage Sweeps}

	After calibrating the light intensity in the solar simulator (see \cpageref{solarsimulator}), the devices were exposed to the illumination at open circuit for some seconds in order to have a stabilized open circuit voltage.
	Then usually the curves were measured with the auto-measure function of the PyPV software (see \cpageref{automeasure}) which measures the reverse scan and then the forward scan.

	\paragraph{Parameters Extraction from Sweeps}
	For the devices studied in this thesis, the reported values of \gls{voc}, \gls{jsc}, \gls{pce} and \gls{ff} are extracted from a forward or reverse current-voltage sweep.
	This is in accordion to the tradition of solar cells reporting but for hysteretic devices, like perovskite solar cells, a static measurement should be preferred.

	\paragraph{Parameters Extraction from Sweeps -- \gls{pce}} Regarding the \gls{pce}, and by consequence the \gls{ff}, a proper measurement is made difficult due to the cell evolution over time (hysteresis).
	The voltage associated to the maximum power point is drifting and its localization affects its evolution.
	A proper \gls{mppt} system needs to be bought or developed, see \cpageref{software_mppt} for thoughts about possible implementations.

	\paragraph{Auto-scale}\label{autoscale} In literature one can easily find current-voltage curves with discontinuities or "kinks" \cite{Li2016,Snaith2014,Zhang2015} like the one reported in \cref{fig:autoscale}.
	Some even lucubrate about the origin of these in perovskite solar cells.
	Indeed this is likely just caused by the auto-scale feature of the Keithley equipment, disabling this, the discontinuities disappears.

	\begin{SCfigure}%[!hbtp]%
		\centering
		\includegraphics[width=0.5\textwidth]{autoscale/ig1-3-1-int4.pdf}
		\mycaption[Kinks in JV sweep due to autoscale.]{A current-voltage sweep of an hysteretic perovskite solar cell with Keithley autoscale active.
			Both the forward (dashed) and the reverse (solid line) present small discontinuities around \SI{1}{\mA} and \SI{0.1}{\mA}.}\label{fig:autoscale}
	\end{SCfigure}

	\paragraph{Scan speed} The used sweep speed is \SI{500}{\mV\per\s}, which was arbitrarily chosen for avoiding bumps leading to currents higher than \gls{jsc}, like the one in \cref{fig:iv_ugly} (seldom reported in literature, like in figure~S3 of \cite{Du2018} and simulated with drift-diffusion in \cite{Walter2018}).
	Our arbitrary choice allowed us to make fair comparisons between devices, but the absolute values should be considered as approximations.
	\begin{SCfigure}%[!hbtp]%
		\centering
		\includegraphics[width=0.5\textwidth]{iv_ugly/ig47-b32-int2-4.pdf}
		\mycaption[Hysteretic current-voltage scan.]{At the employed scan speed, the hysteresis phenomena causes the reverse (solid line) scan to reach currents higher than the \gls{jsc}.}\label{fig:iv_ugly}
	\end{SCfigure}
	We wanted to underline that due to hysteresis phenomena, no scan speed, direction, or precondition is the correct one.
	Rather, a static measurement or a \acr{mppt} should be used for obtaining a accurate and realistic result.
	This comment regards also the so-called "hysteresis-free" perovskite solar cells, which can also have hysteretic phenomena\cite{Jacobs2018,Du2018}.

	\paragraph{Noise} The noise often observed in current-voltage sweeps at high scan speeds in this thesis is mainly caused by oscillations in the solar simulator illumination intensity, as an example see \cref{fig:iv_params}.
	For reducing the noise impact on the \gls{jsc} and \gls{voc} parameters extraction, these values were extracted \textit{via} a parabolic fitting.

	\paragraph{Stabilized or dynamic current-voltage sweeps} One very appealing alternative to current-voltage sweeps are the so-called "stabilized current-voltage sweeps", where at each voltage point a fixed stabilization time is waited and the stabilised current is reported \cite{Unger2014, Christoforo2015}.
	An improvement of this technique is named "dynamic current-voltage sweeps", here the stabilization step is of variable duration, until the current-time derivative falls below a threshold (\textit{e.g.}\ \SI{0.2}{\%\per\minute}) \cite{Dunbar2017,Dunbar2017a}.
	In this thesis, these techniques have not been used.

	\paragraph{Shunt and series resistances} \label{resistances} In our group the shunt and series resistances are evaluated by the current \textit{versus} voltage derivative of a dark current-voltage sweep respectively at zero and at high-enough voltage.
	This methodology is inherited from organic solar cells and, as is easy to foresee, is unreliable on hysteretic devices: in the case of perovskite solar cells a measurement of the stabilised current at a few points have better chances to produce a useful result.
	The measurement of the current at a voltage close to zero is enough for estimating the shunt resistance while two points at high voltages are needed for estimating the series resistance.

\section{V\textsubscript{OC} and J\textsubscript{SC} Dependence on Light Intensity}
	The solar simulator illumination intensity $\phi$ is reduced \textit{via} neutral density filters with transmittance of 0.05, 0.12, 0.25, 0.51, 0.81 and 1 (no filter).
	The values of $\voc(\phi)$ and $\jsc(\phi)$ can be obtained from static measurements or from current-voltage sweeps.
	The static measurement of $\voc(\phi)$ at high light intensities is troublesome as it can easily damage the device.
	In this thesis, the used method is specified case by case.

	\subsection{\Glsentryshort{jsc} \textit{versus} $\phi$}\label{jsc-phi}
		The \gls{jsc} dependency on the light intensity $\phi$ is close to linear and can be fitted with a power law:
		\begin{equation} \label{eq:jsc-phi}
			\jsc \propto \phi^\alpha
		\end{equation}
		giving $\alpha$ values usually from 0.95 to 1.

		\paragraph{Interpretation} 
		An $\alpha$ value lower than 1 indicates that not all the photo-generated charges get extracted, neither at short circuit conditions.
		This can happen due to recombination processes (non-geminate ones, for geminate recombination see \cpageref{intro_geminate}) at short circuit \cite{Credgington2011} or of other factors limiting the charge collection. 

	\subsection{\Glsentryshort{voc} \textit{versus} $\phi$ and the Ideality Factor $m$}
		Setting $J=0$, which corresponds to open circuit conditions, in \cref{eq:photodiode} (without considering the series resistance correction) we can obtain a relation between \gls{jsc} and \gls{voc}:
		$$\jsc(\phi) = J_0\left[\exp(\frac{q\voc (\phi)}{n_|id|k_|B|T})-1\right]$$
		This equation can already be used for obtaining the $n_|id|$ and $J_0$ values fitting the \gls{jsc} and \gls{voc} measured at different light intensities $\phi$ (also varying the temperature could be used for fitting the values, but it was not done during this thesis \cite{Tvingstedt2016}).
		Solving for \gls{voc} we obtain:
		$$\voc = \frac{n_|id|k_|B|T}{q}\cdot\ln\left(\frac{\jsc}{J_0} + 1\right)$$
		Considering that the saturation current $J_0$ (current in dark under reverse bias) is much smaller than $\jsc$ for the light intensities we usually employ (down to \SI{0.05}{suns}), we can approximate to:
		$$\voc \approx u_1 + \frac{n_|id|k_|B|T}{q}\cdot\ln(\jsc)$$
		where $u_1$ is a useless constant.
		Then if the $\alpha$ value is close enough to~1, we can use \cref{eq:jsc-phi} and further approximate for plotting against light intensity $\phi$:
		\begin{equation}\label{eq:voc_vs_phi}
			\voc (\phi) \approx u_2 + \frac{n_|id|k_|B|T}{q}\cdot\ln(\phi)
		\end{equation}
		This is the equation we commonly employ for fitting and obtaining ideality factors \cite{Nelson2003}.
		In some cases this latest expression is used as ideality factor definition as it conveniently uses a zero current measurement (\gls{voc}) so that series resistance can be completely ignored \cite{Kirchartz2012}.
		The shunt resistance will still affect the measurement \cite{Tvingstedt2017}.
		The so-obtained ideality factor $n_|id|$ is usually from 1 to 2.
		A voltage dependent ideality factor can also be measured either from a derivative form of \cref{eq:voc_vs_phi} \cite{Tvingstedt2017} or from a current-voltage sweeps in dark.
		The latter method has not been tested in this thesis as it would be affected by hysteresis.
		A critical analysis of these methods and the proposal of a new "Transient Suns-\gls{voc}" method (employing a pre-biassing for flattening the ionic profile) can be found in \authoryear{Calado2018b}.

		\paragraph{Interpretation} %\The \gls{voc} dependency on the light intensity $\phi$ can be fitted with a natural logarithmic dependence obtaining the ideality factor $m$. 
		\authoryear{Pockett2015} measured ideality factors of planar perovskite solar cells \textit{via} stabilized \gls{voc} obtaining, for some cases, values as high as 5.
		Also in organic \cite{Kirchartz2011,Kirchartz2012} and silicon solar cells \cite{Breitenstein2006} ideality factors greater than 2 has been observed and explained.
		According to \authoryear{Calado2018b} and \authoryear{Kirchartz2012}, the ideality factor, once obtained in the correct way, is 1 when studying most of the recombination types and 2 for mid-gap trap mediated recombination in regions where the electrons and holes concentrations are similar, $n \approx p$.

\section{Charge Extraction (\glsentryshort{ce})}

	\paragraph{Concept} The charge extraction experiment has been designed to quantify the free charges available in a device \cite{Duffy2000,Barnes2011}.
	After stabilization of a device at a light intensity and an applied voltage (in our case always at \gls{voc}), the illumination is switched off and the electrodes of the device are short circuited.
	The transient current flowing through the short circuiting circuit can be measured and integrated to estimate the available charge.
	The integrated charge has to be considered as a lower bound to the actually present excess charge, as part of it could recombine inside the device during the extraction time \cite{ORegan2005}.

\paragraph{Procedure}
The device is kept under 1~sun equivalent illumination by a white \gls{led} at open circuit conditions until stabilization is reached.
1~sun equivalent illumination is defined as the illumination at which the a silicon photodiode gives the same \gls{jsc} as under calibrated 1~sun from the solar simulator.
The \gls{led}-solar spectral mismatch affects slightly the measurement, but in no case a \gls{pce} is reported from any \gls{led}-illuminated experiment.
After stabilization the illumination is switched off and, at the exactly same moment, the device is short circuited through a small and known resistance of \SI{50}{\ohm}.
This is repeated decreasing the light intensity from 1~sun down to dark (in dark no signal should be observed, indeed some residual charge can usually be seen, the reason of this could be ionic profile updating or an insufficient darkness) and a single decay is measured for each illumination point, over approximatively 30 illumination points.
The equipment includes two transistors (in a home made circuit by Dr.\ Javier Pérez Hernández) connected to a pulse generator providing a square pulse long at least as the measurement window.
From my experience, I recommend to use a short dark period in order to save time for the following stabilization step.
The measurement is carried out with an oscilloscope in parallel to the known small resistance.
In the first microseconds, most of the free charge flows through the resistor generating a voltage drop across it which is measured by the oscilloscope.
This potential drop can be converted to current using the Ohm's law, which, integrated over time, gives the amount of extracted charge.

\paragraph{Noise sources} \label{ce_noise}
The lack of complete stabilization of the device before the extraction of charge can introduce both an error in the measured \gls{voc} and in the extracted charge.
Regarding the \gls{voc}, in perovskite solar cells it not only depends on the illumination intensity, but it also evolves slowly until the stabilization at the steady state value, so a well defined stabilization procedure is key for achieving reproducibility in \acr{ce} experiments.
Regarding the extracted charge, the ionic profile can influence the amount of accumulated charge, as shown for two extreme cases of presence/absence of ionic charges in \cref{fig:ce_full_dd}, so a reproducible procedure for device stabilization will also improve the reproducibility of the integrated current amount.
Additionally, the measurement equipment introduces some electronic noise whose effect can be mitigated through data post-processing.

\begin{figure}
	\makebox[\textwidth][c]{
		\parbox{1.1\textwidth}{
			\centering
			\begin{subfigure}[t]{1\textwidth}
				\includegraphics[width=0.45\textwidth]{{ce_noise/normal-CE_ig104-1566-4_Voc_0.911mV-crop}.pdf}
				\qquad
				\includegraphics[width=0.35\textwidth]{ce_noise/normal-spiro_vs_TAEs-CEs-crop.pdf}
				\subcaption{Direct integration of raw data}\label{fig:ce_noise-normal}
			\end{subfigure}
			\bigskip
			
			\begin{subfigure}[t]{1\textwidth}
				\includegraphics[width=0.45\textwidth]{{ce_noise/subtractDark-CE_ig104-1566-4_Voc_0.911mV-crop}.pdf}
				\qquad
				\includegraphics[width=0.35\textwidth]{ce_noise/subtractDark-spiro_vs_TAEs-CEs-crop.pdf}
				\subcaption{Subtraction of morphed noise profile}\label{fig:ce_noise-subtractDark}
			\end{subfigure}
			\bigskip
			
			\begin{subfigure}[t]{1\textwidth}
				\includegraphics[width=0.45\textwidth]{{ce_noise/integrateExp-CE_ig104-1566-4_Voc_0.911mV-crop}.pdf}
				\qquad
				\includegraphics[width=0.35\textwidth]{ce_noise/integrateExp-spiro_vs_TAEs-CEs-crop.pdf}
				\subcaption{Integration of an exponential fitting}\label{fig:ce_noise-integrateExp}
			\end{subfigure}
			
			\mycaption[Strategies for reducing the instrumental noise in a single \glsentryshort{ce} integration.]{
				On the left, single \gls{ce} decays from a \gls{fto}\-/\ch{d-TiO2}\-/\acr{csfamapbibr}\-/\gls{tae3}\-/Au device \cite{Gelmetti2019} is integrated without noise reduction in (\textbf{a}), adapting the noise profile from the dark measurement and subtracting it in (\textbf{b}), or fitting the decay and integrating the fit in (\textbf{c}).
				On the right, the charge \textit{versus} voltage trends obtained applying the respective noise reduction methods.}\label{fig:ce_noise}
		}
	}
\end{figure}

\paragraph{Reduction of instrumental noise}\label{r_ce_noise}
Most of the observed short-time noise (\SI{< 5E-7}{\s}) observable in \cref{fig:ce_noise-normal} is related to the opening and closing of the transistor switches included in the home-made circuit.
The characteristic frequencies of the observed noise are not small compared to the measurement window, so its time-integral does not necessarily sum to zero.
In order to reduce its impact, various approaches have been tested and here described.
The noise can be ignored in some cases, but it's a problem if the charge \textit{versus} light bias (\gls{voc} generated at a given illumination intensity) profile, reported on the right hand column of \cref{fig:ce_noise}, has to be studied in detail, as in \cref{ch:tae}.
In the mentioned case, just the exponential part of its linear plus exponential behaviour, reported in the bottom of each right hand figure, was of interest and it is evident that the employed noise-reduction method influences it heavily.

\paragraph{Reduction of instrumental noise -- subtraction of dark noise}
Annoyingly, the noise profile is characteristic of the cell and of the circuitry, so a simple average over many decays does not help in cancelling it.
Based on this consideration, I tried to subtract a pure noise profile as obtained from a dark measurement (without any light bias).
The operation was made more difficult by the slight variations of the noise profile with the light bias.
A data-to-morphed-noise fit was implemented where the $f(t)$ noise profile was transformed with: $t'= u_3 + u_4 \cdot t + u_5 \cdot t^2$ and $f'(t') = u_6 \cdot f(t') + u_7 \cdot t' \cdot f(t') + u_8 \cdot e^{-t'/u_9}$ where $u_{3-9}$ are constrained fit variables.
Then $u_6$ was set to zero and the resulting profile was subtracted from the data and the result integrated.
As can be seen in \cref{fig:ce_noise-subtractDark}, this technique is working for most of the cases but it can fail if the noise profile changes in a more complex fashion.

\paragraph{Reduction of instrumental noise -- integration of a fitting} Finally the decays were fitted with a bi-exponential formula (sum of two exponential) or, if the bi-exponential fitting was not converging, by a simple exponential and the integral of this fit was used.
In both cases a robust fitting routine was employed \cite{Maechler2018}.

	\paragraph{Limitations specific of perovskite solar cells} \label{ce_limitations_perovskite}When measuring the \gls{ce} of a perovsktie solar cell, additionally to the aforementioned limitations, one should also consider that the ionic profile update (from $V=\voc$ to $V=0$) causes a displacement current, as described in \cpageref{intro_displacement_current}.
	A simulation with DrIFtFUSION is reported for an homojunction device in \cref{fig:ce_single_dd}, it can be seen that with ionic mobile defects in \cref{fig:ce_single_dd-ions_zoom} a very weak but long lasting current appears and gives a relevant contribution to the integrated charge.
	This will happen on very large time scales and it will not affect the short measurements used for the free charges estimation, so it is rarely reported \cite{ORegan2015b}.
	The charge measured in the external circuit due to the ionic displacement current could be underestimated as the free charges rearrangements can also occur through the perovskite layer rather than through the external circuit.

	\begin{figure}
		\makebox[\textwidth][c]{
			\parbox{1.1\textwidth}{
				\centering
				\begin{subfigure}[t]{0.5\textwidth}
					\includegraphics[width=1\textwidth]{ce_single_dd/ce_single_dd-noions.pdf}
					\subcaption{Without mobile ions}\label{fig:ce_single_dd-noions}
				\end{subfigure}
				\bigskip

				\begin{subfigure}[t]{0.5\textwidth}
					\includegraphics[width=1\textwidth]{ce_single_dd/ce_single_dd-ions.pdf}
					\subcaption{With mobile ions}\label{fig:ce_single_dd-ions}
				\end{subfigure}
				\qquad
				\begin{subfigure}[t]{0.5\textwidth}
					\includegraphics[width=1\textwidth]{ce_single_dd/ce_single_dd-ions_zoom.pdf}
					\subcaption{With mobile ions, magnified left axis}\label{fig:ce_single_dd-ions_zoom}
				\end{subfigure}

		\mycaption[Simulation of a CE without or with mobile ions.]{
			The current \textit{versus} time profile of a \acr{ce} simulated experiment is shown on the left axis just for the 1~sun illumination intensity.
			On the right axis the cumulative integration of the extracted charge.
			Clearly, the extraction is unrealistically quick as the resistance included in a real \acr{ce} experiment was not included in the simulation.
			In (\textbf{a}) the measurement of a device without mobile ions is shown, we can observe just one current peak contributing to the integrated charge; in (\textbf{b}) the presence of the mobile ions introduces a long-times contribution to the extracted charge, the current causing this is very weak and long lasting, it can be better observed in the magnification in (\textbf{c}).}\label{fig:ce_single_dd}
					}
	}
	\end{figure}


%	\subsection{Interpretation of Charge Extraction}\label{interpretation_ce}

		\paragraph{Charge extracted}
		The integrated charge is assumed to include the excess free charges in the valence and conduction bands.
		With \emph{excess} we refer to the difference between the charge concentration in the conditions of interest and the stabilized dark condition.
		For a non perfectly crystalline material, localized shallow traps constituted by the tails of the valence and conduction bands density of states inside of the so-called mobility gap \cite{Pieters2009} are not negligible and should also contribute to the extracted charge amount \cite{Kirchartz2012,Du2018}.
		On the contrary, charges trapped in deep traps contributing to SRH trap mediated recombination, with energies far from the band edges, should not be possible to extract in a \acr{ce} experiment.

		\paragraph{\Acr{ce} time constant}
		The free charges extraction time is related to the RC time of the \SI{50}{\ohm} resistor and the capacitance of the solar cell device.
		We can see in \cref{fig:chargeExtraction_RCtime} a weak covariance (Pearson correlation coefficient of 0.3) between the RC time obtained extrapolating the dark capacitance from \acr{dc} (which is the geometric capacitance) and the extraction time (as obtained by an exponential fitting to a single \acr{ce} current decay) at low light intensity (enough for having a signal but far from 1~sun light intensity).
		At higher light intensities, the correlation is weaker as the capacitance is less defined as the cell is in a transition between illuminated (high capacitance) and dark (low capacitance) status.
		Anyway, the extraction time does not change much between low light intensity and 1~sun with an increase from \SIrange{1.1}{2.4}{times} (first and third quartile).
		More discussion on this topic can be found on \authoryear{Montcada2018}.

		\begin{SCfigure}%[!hbtp]%
			\centering
			\includegraphics[width=0.45\textwidth]{chargeExtraction_RCtime/CEaBitOfSunExpTime_vs_RCdarkTime.pdf}
			\mycaption[Charge extraction time is related to a RC time.]{
				Covariance of \acr{ce} extraction time at low light intensity \textit{versus} the expected time from geometric capacitance (as obtained from dark \acr{dc}).
				Each point is a different device for a total of 78 devices, many different structures studied during my PhD are represented.
				The green line indicates the 1 to 1 relationship.}\label{fig:chargeExtraction_RCtime}
		\end{SCfigure}

		\paragraph{\Acr{ce} time constant and \acr{tpv} time constant -- Corrections}
		During this time, and depending on its location in the device stack, some free charge can recombine.
		One could argue that a \acr{ce} measurement is valid only if the extraction is faster than the recombination time as measured via \acr{tpv} \cite{Ryan2017a} or that the extracted charge should be corrected considering the recombination \cite{Credgington2011,Credgington2014}.
		Considering the charges accumulated in the depletion layers in the selective contacts, these will flow to the electrodes without crossing the perovskite/selective contacts interfaces, where has been reported that most of the recombination occurs \cite{Barnea-Nehoshtan2014,Stolterfoht2018a,Stolterfoht2018}.
		So this part of the extracted charge, distinguishable as the linear part of the charge \textit{versus} voltage plot, as represented on the right column of \cref{fig:ce_noise} should not be corrected.
		Instead, regarding the charge accumulating in the perovskite layer, which we assume can be assigned to a chemical capacitance and can be recognized as the exponential part on the right column of \cref{fig:ce_noise}, it may be that a correction \cite{Shuttle2008a,Shuttle2008b} is needed, but this has not be done in this thesis.

		\paragraph{\Acr{ce} time constant and \acr{tpv} time constant -- Correlation?}
		Some covariance (Pearson correlation coefficient of~0.4) can be observed in \cref{fig:ce_1sun_time_vs_tpv_1sun_time} between the \acr{ce} and the \acr{tpv} time constants at 1 sun illumination.
		This is unexpected and weird as the two times change with very different trends with light bias (when changing the preconditioning light intensity, \gls{ce} extraction time changes just slightly while \gls{tpv} decay time varies over various orders of magnitude).
		In case a stronger proof of correlation is found, this could indicate that both processes, even if not of the same nature, are limited by the same diffusion process, for example the migration of free charges from all the absorber to the absorber/contacts interfaces.

		\begin{SCfigure}%[!hbtp]%
			\centering
			\includegraphics[width=0.45\textwidth]{ce_1sun_time_vs_tpv_1sun_time/ce_1sun_time_vs_tpv_1sun_time.pdf}
			\mycaption[Comparison between \glsentryshort{ce} and \glsentryshort{tpv} exponential decay times.]{
				Covariance of \acr{ce} extraction time at 1~sun light intensity \textit{versus} the \gls{tpv} mono-exponential decay time at 1~sun light intensity.
				Each point is a different device for a total of 79 devices, including many different structures.
				The green line indicates the 1 to 1 relationship.}\label{fig:ce_1sun_time_vs_tpv_1sun_time}
		\end{SCfigure}

		\paragraph{Charge \textit{versus} light bias trend - Exponential part in \gls{osc}}\label{ce_exp_osc}
		In \gls{osc} literature the charge \textit{versus} light bias voltage trend is described simply as the exponential shape which describes a Maxwell--Boltzmann distribution for a two levels scenario.
		For a common solar cell working conditions, the Maxwell-Boltzmann classical particles approximation should be valid as the distance between Fermi level energy and the band edges is expected to be always much bigger than $k_|B|T$.
		This could be false for high applied voltages, where Fermi-Dirac distribution for fermions should be used.
		$$n_|CE| = n_|DOS| \exp(\frac{qV - E_|g|}{k_|B|T}) = n_|pre| \exp(\frac{qV}{k_|B|T})$$
		where $n_|pre|$ prefactor is the equilibrium carrier concentration from a Boltzmann distribution of a 2 levels system. %just a pre-factor with no direct physical interpretation.
		In some cases an ideality factor $m$ is introduced \cite{Kirchartz2012}, which can help to account for the shape of the density of states of the conduction band, so the expression can be found as:
		\begin{equation}\label{eq:ce_osc}
			n_|CE| = n_|pre| \exp(\frac{qV}{mk_|B|T})
		\end{equation}
		For perovskite solar cells, an alternative interpretation can be found indicating that the exponential trend reflects the extraction from exponential subgap states, \textit{i.e.} shallow traps \cite{Du2018}.
		This can be used for explaining values of $m$ greater than 2, which would be indicative of the exponential tail width as shown for \gls{osc} by \authoryear{Kirchartz2012}.
		Moreover, in \gls{osc} has been shown that an inhomogeneous carriers concentration profile through the device thickness, likely to happen in thin films, would also cause high $m$ values \cite{Kirchartz2012} especially for low doping levels resulting in drift-driven solar cells \cite{Deledalle2015,Deledalle2014}.
		
		
	\begin{figure}
	\makebox[\textwidth][c]{
		\parbox{1.1\textwidth}{
			\centering
			\begin{subfigure}[t]{0.5\textwidth}
				\includegraphics[width=1\textwidth]{ce_single_dd_charge/ce_ions_1sun.pdf}
				\subcaption{1 sun}\label{fig:ce_single_dd_charge-1sun}
			\end{subfigure}
			\qquad
			\begin{subfigure}[t]{0.5\textwidth}
				\includegraphics[width=1\textwidth]{ce_single_dd_charge/ce_ions_1000suns.pdf}
				\subcaption{1000 suns}\label{fig:ce_single_dd_charge-1000suns}
			\end{subfigure}
		
			\mycaption[Simulated electron density profile during a CE experiment.]{
				The electron density profile of a homojunction cell with mobile ions is plotted at the interface between perovskite (position \SI{< 700}{\nm}) and \gls{etm} (position \SI{> 700}{\nm}) at different times during a \acr{ce} experiment.
				The green solid line represent the situation as stabilized at open circuit conditions, before the \acr{ce} experiment. Corresponding energy levels for (\textbf{a}) in \cref{fig:ce_single_dd_levels-1sun} and for (\textbf{b}) in \cref{fig:ce_single_dd_levels-1000suns}.
				The dashed orange line represent the profile at short times, after the extraction of the free charges but before the reorganization of the mobile ions.
				The light blue area represents the extra charge extracted during a typical \acr{ce} experiment.
				In (\textbf{a}) the device is stabilized at 1 sun light intensity before performing the charge extraction, in this case most of the extra charge is accumulated in the contacts.
				In (\textbf{b}) the device is stabilized at 1000 suns, in which case the charge is mainly accumulated in the perovskite layer.
				The dotted violet line is the final electronic profile, at long times after the mobile ions migration, which is identical for (\textbf{a}) and (\textbf{b}) cases. Corresponding energy levels in \cref{fig:ce_single_dd_levels-dark}.
			}\label{fig:ce_single_dd_charge}
		}
	}
\end{figure}

		\paragraph{Charge \textit{versus} light bias trend -- Linear part}
		With the introduction of selective contacts, a linear contribution starts to grow in importance summing up to the exponential part.
		It has been observed both in \gls{osc} \cite{Ryan2017a,Credgington2014} and in perovskite solar cells \cite{Gelmetti2017,Wheeler2017,Du2018}.
		This linear trend accounts for the accumulation in the selective contacts' depletion layers and in the electrodes, in a parallel plate capacitor fashion: $n = C_|g| \cdot V = \frac{\epsilon_0 \epsilon_|r| A}{d} \cdot V$ where $C_|g|$ is the geometric capacitance, $A$ is the active area, and $d$ is the thickness of the dielectric.
		More exactly, $d$ is the distance between the regions where the opposed charges are getting accumulated, which is the space charge layers (usually depletion layers) in the electrodes.
		So this value can be somewhat wider than just the distance separating the two electrodes interfaces.
		By consequence, $\epsilon_|r|$ should be considered as a thickness-weighted mean of the relative permittivities of each material between the two accumulation zones.
		This carriers accumulation in the contacts can be visualized with the simulation reported in \cref{fig:ce_single_dd_charge-1sun}.
		Extending \cref{eq:ce_osc}, the complete equation becomes:
		\begin{equation}\label{eq:ce_full}
			n_|CE| = C_|g| \cdot V + n_|pre| \cdot \left[\exp(\frac{qV}{mk_|B|T}) - 1\right]
		\end{equation}
		where the $-1$ was introduced for forcing the curve to cross the origin, as in steady state dark conditions both the \gls{voc} and the extracted charge are defined as zero.
		A voltage dependent geometric capacitance (as mentioned, it could change due to the widening-shrinking of the depletion layers) can be obtained from dark \acr{ce} measurements applying reverse voltage biases \cite{Kiermasch2018} or \textit{via} impedance spectroscopy in dark with a constant voltage bias \cite{Brus2016}.

		\paragraph{Charge \textit{versus} light bias trend -- Energy levels point of view}\label{ce_energy_levels}
		As we mentioned, the linear trend is related to the accumulation in the contacts depletion layers (or, generically, space charge layers), this can be seen from the decrease of the perovskite-contacts conduction band offset between \cref{fig:ce_single_dd_levels-dark,,fig:ce_single_dd_levels-1sun}.
		Increasing the light illumination at open circuit conditions, the quasi-Fermi levels splitting approaches the built-in voltage represented by the contacts' band edges: $V_|bi| = E_|CB|^{\mathrm{ETM}} - E_|VB|^{\mathrm{HTM}}$.
		This means the depletion layers are close to saturation, like in \cref{fig:ce_single_dd_levels-1000suns} where \SI{1000}{suns} illumination was simulated, and the charge accumulates in the perovskite layer, as shown in \cref{fig:ce_single_dd_charge-1000suns}, and the exponential trend gains importance.

					\begin{figure}
			\makebox[\textwidth][c]{
				\parbox{1.1\textwidth}{
					\centering
					\begin{subfigure}[t]{0.3\textwidth}
						\includegraphics[width=1\textwidth]{ce_single_dd_levels/ce_ions_dark_levels.pdf}
						\subcaption{dark}\label{fig:ce_single_dd_levels-dark}
					\end{subfigure}
					\qquad
					\begin{subfigure}[t]{0.3\textwidth}
						\includegraphics[width=1\textwidth]{ce_single_dd_levels/ce_ions_1sun_levels.pdf}
						\subcaption{1 sun, OC}\label{fig:ce_single_dd_levels-1sun}
					\end{subfigure}
					\qquad
					\begin{subfigure}[t]{0.3\textwidth}
						\includegraphics[width=1\textwidth]{ce_single_dd_levels/ce_ions_1000suns_levels.pdf}
						\subcaption{1000 suns, OC}\label{fig:ce_single_dd_levels-1000suns}
					\end{subfigure}
					
					\mycaption[Simulated energy levels in a homojunction device with mobile ions at open circuit conditions at different light intensities.]{The simulated energy levels of a homojunction $p$(\SI{200}{\nm})--$i$(\SI{500}{\nm})--$n$(\SI{200}{\nm}) device with mobile ions in the intrinsic layer. In (\textbf{a}) the device is in dark, in (\textbf{b}) it is illuminated at \SI{1}{sun} light intensity, and in (\textbf{c}) illuminated at \SI{1000}{suns}.
					}\label{fig:ce_single_dd_levels}
				}
			}
		\end{figure}

		\paragraph{Charge \textit{versus} light bias trend -- Linear part with mobile ions}
		The presence of mobile ions in perovskite materials which can accumulate at the perovskite/contacts interfaces, adds an additional capacitance $C_{ion}$, which sums up to the geometric capacitance $C_|g|$ increasing the weight of the linear component, as we showed in \authoryear{Moia2019}.
		Nevertheless, as pointed out in \cpageref{ce_limitations_perovskite}, the \acr{ce} measurements are never carried on for long enough to include the ionic migration, and so also the ionic accumulation capacitance gets ignored in our experiments.
		This can be visualized with the simulation reported in \cref{fig:ce_full_dd} where the long timescale (where the current is monitored until complete stabilization) and the short timescale (few tens of microseconds) \acr{ce} experiment are compared.
		The difference between the short and long timescale extracted charges is the linear contribution by the ionic capacitance discharge, observable as electronic current thanks to the relative displacement current.
		A simulation with frozen ions (the ionic profile was stabilized at open circuit and frozen during the \acr{ce} experiment) was also performed but not reported, the extracted charge is identical to the reported short timescale extraction simulation.
		
			\begin{figure}%[!hbtp]%
		\makebox[\textwidth][c]{
			\parbox{1.1\textwidth}{
				\centering
				\includegraphics[width=0.9\textwidth]{ce_full_dd/ce_full_dd-crop.pdf}
				\mycaption[Simulation of a complete CE experiment: charge \textit{versus} light bias with or without mobile ions.]{
					A simulation for a homo-junction device up to 1000~sun illumination (rightmost point of each set) is reported, with three points for decade.
					For each set of points, the bigger size one indicates the 1~sun pre-illumination.
					The green crosses simulates the experimentally utilised conditions: the charge gets integrated over few microseconds, while the mobile ions didn't have enough time to start migrating.
					The orange pluses considers the charge integrated until the device stabilization, over various seconds, including also the ionic displacement current.
					The purple circles simulates a mobile-ions free device.
					The solid lines represents the linear plus exponential fit crossing (0,0) obtained with the \cref{eq:ce_full} $n = C_|g| V + n_|pre| \{\exp[q V / (mk_|B|T)] - 1\}$ where $C_|g|$, $n_|pre|$, and $m$ are free fitting parameters.}\label{fig:ce_full_dd}
			}
		}
	\end{figure}

		\paragraph{Charge \textit{versus} light bias trend -- Exponential part with or without mobile ions}
		As can be seen in \cref{fig:ce_full_dd}, the simulated geometric capacitance is similar to the one obtained from short time extraction with mobile ions but the exponential part is considerably different.
%		when a device with no mobile ions is simulated, longer extraction time doesn't result in more charge, as expected due to the suppression of the ionic contribution.
		This is caused by the very different free charge accumulation profile: the presence of an un-shielded electric field in the absorber layer causes the free carriers to accumulate close to the respective selective layer, in other words it keeps the charges away from the respective recombination centres (\textit{e.g.}\ perovskite/\gls{htm} for electrons).
		This allows the simulated ions-free device to store more charge at the same illumination intensity.

		\FloatBarrier
\section{Transient PhotoVoltage (TPV)}
	\epigraph{\textit{"Imma firin mah lazor\\pewpew pewpewpew"}}

	\paragraph{Concept}
	While a complete device is kept open circuit under constant illumination, a small extra illumination is added \textit{via} a short laser pulse.
	The \gls{voc}, originally at its steady state value, will be perturbed due to the greater generation rate during the laser pulse.
	From the \gls{voc} \textit{versus} illumination relation for photodiodes reported in \cref{eq:voc_vs_phi} follows that the \gls{voc}, at this new higher illumination, increases (this is not always the case, as for non-stabilized perovskite solar cells \cite{Calado2016}).
	After the short pulse the \gls{voc} will slowly go back to the steady state value relative to the constant illumination.
	The dynamics of this \gls{voc} relaxation back to the steady state value is the focus of Transient PhotoVoltage experiments (also known as PhotoVoltage Decay experiments) which has been applied to \gls{osc} \cite{Shuttle2008}, to \gls{dssc} \cite{ORegan2005,ORegan2004,ORegan2006}, and recently also to perovskite solar cells \cite{Roiati2014a,Marin-Beloqui2014}.

	\paragraph{Procedure}
	The device is kept under 1~sun illumination by a white \gls{led} ring at open circuit until stabilization is reached.
	Failure to reach stabilization of the ionic profile will affect the measurement results \cite{ORegan2015b}.
	1~sun equivalent illumination is defined as the illumination at which the a silicon photodiode gives the same \gls{jsc} as under calibrated 1~sun from the solar simulator.
	Then an additional illumination pulse is provided by a nitrogen laser.
	The pulse duration (\SI{\approx 1.5}{\ns}) is shorter than the oscilloscope resolution we usually employ, so we assume that the measurement happens when the pulse is already over.
	In the literature, this is not always the case as other research groups use a \gls{led} diode for the pulsed illumination \cite{Calado2016}.
	Usually a wavelength of \SI{650}{\nm} is selected using a Rhodamine B solution\cite{RadiantDyesLaser}, this wavelength illuminates in depth the perovskite layer (in contrast to a blue light where the illumination would be absorbed within the first hundreds of nanometres of the material \cite{Bi2016,Tress2016}).
	During all the process, the device is connected to an oscilloscope, registering the open circuit voltage profile (the \SI{1}{\Mohm} resistance of the oscilloscope is a good approximation of open circuit).
	The voltage profile gets averaged over a few tens of pulses in order to increase the signal to noise ratio.
	Then the background light intensity is slightly decreased and, after the stabilization to the new steady state, the new  \gls{voc} is registered and more transients are registered.
	This process is repeated over a few tens of light illuminations from 1~sun down to dark.

\paragraph{Noise treatment}\label{tpv_robust}
Most of the observed noise (\SI{< 2E-7}{\s}) is due to the radiofrequency emitted by the spark in the nitrogen laser which gets received by all the non-coaxial cables (coaxial ones don't) and from the circuitry of the samples holder acting as an antenna.
On the contrary to what happens for \acr{ce} (see \cpageref{ce_noise}), the short times noise does not follow a constant pattern, so averaging the measurement over a few repetitions (usually 30) manages to reduce the noise.
This noise can affect the exponential or bi-exponential fitting, for this reason a robust fitting routine has been used, which gives a lower weight to outlier points.
An example can be seen in \cref{fig:tpv_robust}.
\begin{figure}
	\centering
	\includegraphics[width=0.9\textwidth]{{tpv_robust/TPV_ig101-1555-1_0.815394_V-monoexp}.pdf}
	\mycaption[Robust and normal fitting comparison.]{Plot of a voltage profile from a single \gls{tpv} decay.
		The 12500 voltage points are represented in a 2D histogram for avoiding the overplotting problem.
		The normal non-linear least squares fitting (in orange) is affected by initial noise (a faster decay if we consider this as a bi-exponential decay), outliers and characteristics not of interest by the model.
		The non-linear robust fitting (in magenta) manages to reduce the weight of these points.
		The studied device is a \gls{fto}\-/\ch{d-TiO2}\-/\acr{csfamapbibr}\-/\gls{tae4}\-/Au \cite{Gelmetti2019} with 1~sun background illumination.}\label{fig:tpv_robust}
\end{figure}

\paragraph{Importance of stable steady state starting point}\label{tpv_negativePeaks}
The values extracted from \acr{tpv} are strongly sensible to the surface recombination and electric fields in the absorber, which in turn vary during ionic profile stabilization.
Comparison of decay times with different stabilization times can be found in \authoryear{ORegan2015b}, while in \authoryear{Calado2016} even negative peaks are reported and explained with the presence of a residual electric field in the absorber previous to ionic profile stabilization.

\paragraph{Small perturbations regime}\label{tpv_perturbation}
The intensity of the laser pulse is attenuated using a variable neutral density filter (a partially reflecting wheel with different positions for different transmittivities) so that the voltage perturbation caused by the light pulse does not exceed \SI{10}{\mV} with 1~sun background illumination intensity.
We consider this a "small-enough" perturbation with regards to the measured \gls{voc} (see \cpageref{perturbation} for a definition of small perturbation).
For example, comparing the excess charge in a \gls{fto}\-/\ch{d-TiO2}\-/\acr{csfamapbibr}\-/\gls{spiro}\-/Au solar cell at \gls{voc} and at \gls{voc}~+~\SI{10}{\mV} which can be obtained from the data in \cref{fig:ce_noise-integrateExp} fitted with \cref{eq:ce_full} we can obtain respectively a value of \SI{8.2E-8}{} and \SI{8.9E-8}{\coulomb\per\square\cm}.
Even if we're not considering the dark charge concentration (not measurable in a \acr{ce} experiment), the smallness of the charge perturbation is arguable.
Still, we're limited to \SI{> 3}{\mV} perturbations due to the strong noise observed in this transient measurement.
Clearly, the pulse intensity which could be considered a "small-enough" perturbation at high background illumination is not small any more at lower illumination and definitively cannot be small at dark background conditions.
We could regulate the pulse intensity depending on the background light intensity, to ensure its smallness, but we \emph{do not} do this in order to be able to use the \acr{tpv} data for calculating \acr{dc}, as explained in \cpageref{dc_perturbation}.
This does not usually affect the parameter extraction from \acr{tpv} as just the high illumination intensity points are considered, as seen in \cpageref{tpv_tau_vs_intensity}, in order to study the device close to its expected working conditions.

	\paragraph{Voltage transient and charge concentration relationship}
	As we assume to be in the small perturbation regime, we can study the \cref{eq:ce_full} up to the first term of its series expansion:
	\begin{dmath*}
		n(V_0 + \Delta V) \approx n(V_0) + \Delta V \cdot \left.\frac{\partial n}{\partial V}\right\rvert_{V=V_0} = n(V_0) + \Delta V \cdot \left[C_|g| + \frac{n_|pre| q}{mk_|B|T}\exp(\frac{qV_0}{mk_|B|T})\right]
%		n(V_0 + \Delta V) \approx n(V_0) + \Delta V \cdot \eval{\dv{n}{V}}_{V=V_0} = n(V_0) + \Delta V \cdot \left(C_|g| + \frac{n_|pre| q}{mk_|B|T}\exp\left(\frac{qV_0}{mk_|B|T}\right)\right)
	\end{dmath*}
	Assigning the last addend to $\Delta n$ we can see that the charge amount variation not only is linear with $\Delta V$ (as we're in the small perturbation regime), but it also depends on the steady state voltage $V_0$.
	This relation is studied in the differential capacitance (\acr{dc}) experiment, described further in this chapter.

	\paragraph{Voltage re-equilibration dynamics}
	For what concerns a \acr{tpv} experiment, we're just interested in the analysis of the time needed for re-equilibration to steady state conditions.
	As aforementioned, the relation between $\Delta V$ and $\Delta n$ in small perturbation regime is linear, so the lifetime extracted from a $\Delta V$ decay will have the same lifetime of the underlying $\Delta n$, which is the interesting quantity when speaking of recombination.
	This means that we can observe the variations in voltage for having a correct kinetic description of the charge amount variation.
	At steady state conditions, the time derivative of the amount of charge is zero $\partial n_0 / \partial t = g(\phi) - U(n_0) = 0$ where $g$ is the generation and $U$ is the recombination.
	Considering the situation after the end of the light pulse, so while $g$ is constant but $n$ has been increased by $\Delta n$, and using a simplified expression for the recombination including just two contributions with reaction order 1 and 2 we can write:
	\begin{dmath*}
%		\frac{\partial (n_0 + \Delta n)}{\partial t} = g - k_1(n_0 + \Delta n) - k_2(n_0 + \Delta n)^2 = (g - k_1 n_0 - k_2 n_0^2) - (k_1 \Delta n + 2 k_2 n_0 \Delta n) - (k_2 \Delta n ^2) \approx - (k_1 + 2 k_2 n_0 ) \Delta n
		\pdv{(n_0 + \Delta n)}{t} = g - k_1(n_0 + \Delta n) - k_2(n_0 + \Delta n)^2 = (g - k_1 n_0 - k_2 n_0^2) - (k_1 \Delta n + 2 k_2 n_0 \Delta n) - (k_2 \Delta n ^2) \approx - (k_1 + 2 k_2 n_0 ) \Delta n
	\end{dmath*}
	where the zeroth order term is the steady state value, so it's zero, and the second order order term can be neglected (if $\Delta n \ll n_0$ then $k_2 \Delta n^2 \ll k_2 n_0 \Delta n$).
	So the rate equation is a simple pseudo first order reaction, and the kinetic behaviour follows an exponential description like:
	\begin{equation}\label{eq:tpv_monoexp}
		n (t) = n_0 + \Delta n_0 \cdot e^{-(k_1 + 2 k_2 n_0) t} = n_0 + \Delta n_0 \cdot e^{-t / \tau}
	\end{equation}
	where $\tau = (k_1 + 2 k_2 n_0)^{-1}$ is the small perturbation life-time.
	For a single recombination with a generic recombination reaction order $\Phi$ we can thus write \cite{Shuttle2008}:
	\begin{equation}\label{eq:tpv_tau_order}
		\tau \approx (\Phi k_\Phi n_0^{\Phi-1})^{-1}
	\end{equation}
	
	
	\begin{figure}
		\makebox[\textwidth][c]{
			\parbox{1.1\textwidth}{
				\centering
				\begin{subfigure}[t]{0.5\textwidth}
					\includegraphics[width=1\textwidth]{tpv/monoexp/spiro_vs_TAEs-TPVs-robustmonoexp-crop.pdf}
					\subcaption{Mono-exponential life-times \textit{versus} light bias}\label{fig:tpv-monoexp}
				\end{subfigure}
			\qquad	
				\begin{subfigure}[t]{0.5\textwidth}
					\includegraphics[width=1\textwidth]{tpv/biexp/tpv-tpv-mixedbimono-crop.pdf}
					\subcaption{Bi-exponential life-times \textit{versus} light bias for a \gls{famapbibr} device}\label{fig:tpv-biexp_full}
				\end{subfigure}
				\bigskip
				
							\begin{subfigure}[t]{0.6\textwidth}
				\includegraphics[width=1\textwidth]{{tpv/biexp/tpv/TPV_ig57-475-2_0.424021_V-biexp-crop}.pdf}
				\subcaption{Single bi-exponential decay}\label{fig:tpv-biexp_single}
			\end{subfigure}
				
				\mycaption[Example of life-times \textit{versus} light bias plot obtained from a TPV experiment.]{In (\textbf{a}) the small perturbation life-times obtained from robust mono-exponential fitting are plotted against light bias for \gls{fto}\-/\ch{d-TiO2}\-/\acr{csfamapbibr}\-/\gls{htm}\-/Au devices \cite{Gelmetti2019}. In (\textbf{b}) the life-times of a \gls{fto}\-/\ch{d-TiO2}\-/\acr{famapbibr}\-/\gls{spiro}\-/Au device as obtained from a robust bi-exponential fitting for low light bias, where the decays are clearly biphasic, and mono-exponential for higher illuminations. In (\textbf{c}) the full data for the point at \SI{0.42}{V} from (\textbf{b}) is reported.}\label{fig:tpv}
			}
		}
	\end{figure}

	
	\paragraph{Bi-exponential decays -- high background illumination}
	For some devices, rather than an exponential decay, a bi-exponential decay is observed.
	This is more frequent in bottom cathode devices including mesoporous titania layers \cite{Carnie2015,ORegan2015b} but has also been reported for silicon solar cells \cite{Kiermasch2018}.
	With \emph{bi-exponential} decay we refer to the sum of two exponential decays with two different half-life times, like:
	\begin{equation}\label{eq:tpv_biexp}
		V (t) = V_0 + \Delta V_1 \cdot e^{-k_1t} + \Delta V_2 \cdot e^{-k_2t}
	\end{equation}
	The presence of two different recombination processes is not enough for justifying a bi-exponential decay.
	This is clear looking back to \cref{eq:tpv_monoexp}, where even treating two recombination processes with even two different reaction orders we obtained a simple exponential decay.
	What we have to remember here, is that \cref{eq:tpv_monoexp} was obtained for a zero dimensional case, like what would happen in a homogeneous chemical reaction.
	In our case, the multiple recombination centres could be spatially separated.
	If the charge concentration in the two centres is dis-entangled (is not identical at any time), the sum of two exponential decays can be expected rather than a simple exponential.
	This dis-entanglement is possible if the time needed for the free carriers to migrate from a recombination centre to the other is larger than the shorter recombination life-time.
	This can happen thanks to a large-enough distance between recombination centres together with a slow free carriers mobility.
	For example for recombination centres at different depths in the solar cell stack, like at the two perovskite/contacts interfaces.
	For laterally spaced recombination centres (\textit{e.g.} pinholes \textit{versus} well covered regions) the high mobility in the electrodes is expected to equal the quasi-Fermi levels in the two adjacent recombination centres avoiding bi-exponential decays, nevertheless reports of this phenomena have been reported \cite{Montcada2017}.

	\paragraph{Mobility limited case}\label{tpv_mobility}
	As we just saw, the mobility is a crucial parameter for the \acr{tpv} experiment.
	As a mental exercise we can imagine a case where the mobility is so slow that the free charges take a long time to diffuse (we consider diffusion as perovskite solar cells are assumed to be field-free in the absorber layer, but the same concept would hold with charges' drift) from the generation zone (in the absorber) to the recombination centre (\textit{e.g.} at a contact/absorber interface).
	In such an extreme case the provisioning of charges to the recombination centres could become a bottleneck rather than the recombination itself.
	In this regime, a \acr{tpv} experiment would be rather insensitive to actual recombination constants and even to light intensity, giving just information about mobility.
	As the observed \acr{tpv} life-times are strongly light-intensity dependent, at least at high background illumination, we can exclude to be in this mobility-limited regime.
	This would have to be revised in case a strong illumination-dependent mobility in perovskite materials was demonstrated, as has been reported in \gls{osc} \cite{Eng2010,Shuttle2010,Deledalle2014}.
	Also a slow trapping and de-trapping of the carriers can result in a reduced and illumination-dependent mobility \cite{Du2018}.
	
	\paragraph{Inhomogeneous charge concentration profiles}
	Due to the internal electric field, the free carriers concentration can be inhomogeneous.
	For the surface recombination, this implies that the average excess carriers concentration obtained from \acr{ce} is not necessarily the carriers concentration at the recombination centre \cite{Kirchartz2012}.
	For the band-to-band recombination, this implies that the $n \cdot p$ product, integrated over the device thickness, can be much smaller than an homogeneous carriers concentration \cite{Deibel2009}.
	These considerations are key for drift-driven \gls{osc} with low doping level materials \cite{Deledalle2015,Deledalle2014} but should be of smaller impact for the diffusion-driven, field free perovskite solar cells, at least at steady state when the internal electric field is mostly shielded by the ionic accumulation at the interfaces.
	
	\paragraph{Bi-exponential decays at low background illumination -- large perturbations}
	While bi-exponential decays at high background illumination are often not observed, these are quite always present at lower background illumination.
	Clearly, the aforementioned explanation for bi-exponential decays at high illumination are still valid for the low illumination case.
	Additionally, when studying a decay measured at lower background light intensity, we have to remember that, at least for our group's \acr{tpv} procedure, we're out of the small perturbation regime and this could justify non-exponential decays in various ways.

	\paragraph{Bi-exponential decays at low background illumination -- ionic migration}\label{tpv_biexp_lowlight_ions}
	There is another process that can be active at the very long time scale of the recombination at low free carriers density (low background illumination): ionic migration.
	What has been observed, simulating a \acr{tpv} experiment on a homojunction device with mobile ions in the absorber layer, is that ionic profile can update to the pulse-perturbed cell condition before the extra free charges recombine.
	This causes a first exponential decay of the voltage, followed by a second exponential decay due to the actual recombination of the charges generated by the laser pulse.
	So in this case, the fast component life-time is linked to the rearrangement time of the ionic profile.

	\begin{figure}
		\makebox[\textwidth][c]{
			\parbox{1.1\textwidth}{
				\centering
				\begin{subfigure}[t]{0.9\textwidth}
					\includegraphics[width=1\textwidth]{tpv_full_dd/tpv_full_dd_noions-crop.pdf}
					\subcaption{Without mobile ions}\label{fig:tpv_full_dd_noions}
				\end{subfigure}
				\bigskip

				\begin{subfigure}[t]{0.9\textwidth}
					\includegraphics[width=1\textwidth]{tpv_full_dd/tpv_full_dd_ions-crop.pdf}
					\subcaption{With mobile ions}\label{fig:tpv_full_dd_ions}
				\end{subfigure}
				\mycaption[Simulated \glsentryshort{tpv} without and with mobile ions.]{The \gls{tpv} of a homojunction solar cell is simulated, with background illumination from \SIrange{1e-10}{1}{suns} (respectively the leftmost and the rightmost points) with 3 points per decade.
					The pulse intensity was regulated in order to not exceed \SI{8}{\mV} of perturbation.
					In (\textbf{a}) a device without mobile ions is simulated, the resulting decay is a simple exponential, so the life-time of an exponential fit is reported.
					In (\textbf{b}) a device with mobile ions in the absorber layer is reported, the resulting decay is a simple exponential at high light biases but a bi-exponential at lower background illuminations, so the two life-times from a bi-exponential fitting are reported.}\label{fig:tpv_full_dd}
			}}
	\end{figure}

	\paragraph{Bi-exponential decays at low background illumination -- fast life-time plateau}
	Both the fast and the slow component of bi-exponential decays hit a maximum value (plateau) at very low light intensities.
	Looking at the fast component, assigned to the rearrangement of the ionic profile, we usually observe a constant life-time.
	This points to a background light insensitive ionic rearrangement time constant, which implies a constant ionic mobility.
	This should be further studied considering the report of illumination-dependent ionic mobility by Maier's group in \authoryear{Kim2018}.

	\paragraph{Bi-exponential decays at low background illumination -- slow life-time plateau}
	The slow decay component at very low background light intensity has been assigned to the electronic recombination.
	In drift-diffusion simulations, the life-time does not show a maximum value and grows as the light intensity is decreased.
	But in the actual experiment setup, decays life-times are limited at long times by the discharge of the through the oscilloscope resistance and through the device shunt resistance, whatever is the fastest.
	The former happens with an RC time of the circuit composed by the capacitance of the device (which can be obtained \textit{via} a \acr{dc} experiment) and the \SI{1}{\Mohm} resistance of the oscilloscope or the internal device shunt resistance \cite{Tvingstedt2017} while the latter happens with a non-exponential and less commonly considered diode-capacitor discharge \cite{Tvingstedt2017,Hellen2003}.
	The oscilloscope resistance could be varied using an attenuating probe (usually 10X or 100X) or a high impedance amplifier.
	This limit is often observed at low light intensities as a plateau in the \acr{tpv} life-time \textit{versus} light bias graph \cite{Tvingstedt2017}.
	Indeed, in \authoryear{Kiermasch2018}, where a \SI{1}{\tera\ohm} input impedance amplifier is used, the plateau is observable from \SIrange{1E-1}{1}{\s}, \textit{i.e.} three orders of magnitude higher than what we observe with our \SI{1}{\Mohm} oscilloscope.
	This affirmation is corroborated by the correlation observable in \cref{fig:tpv_RCtime}.
	This limitation has to taken in consideration also when performing full signal Open Circuit Voltage Decay dynamics measurement (OCVD) \cite{Tvingstedt2017,Lederhandler1955,Mahan1981} which has not been performed in this thesis.
	For mid-illumination intensities, the measured life-time can be corrected considering the contribution from the RC time \cite{Credgington2014}.

	\begin{SCfigure}
		\centering
		\includegraphics[width=0.45\textwidth]{tpv_RCtime/TPVdarkTime_vs_RCdarkTime.pdf}
		\mycaption[\Glsentryshort{tpv} time has an upper bond due to discharge through oscilloscope.]{
			Dark \acr{tpv} time (from a robust exponential fit) \textit{versus} RC time derived from the geometric capacitance from \acr{dc} and the \SI{1}{\Mohm} of the oscilloscope.
			Each point is a different device for a total of 76 devices, including many different structures.
			The green line indicates the 1 to 1 relationship.}\label{fig:tpv_RCtime}
	\end{SCfigure}

	\paragraph{Life-time \textit{versus} light bias dependence}\label{tpv_tau_vs_intensity}
	Considering the small perturbation life-time dependency from charge concentration due to a single dominant recombination mechanism from \cref{eq:tpv_tau_order} and substituting the charge with the expression for the charge from \cref{eq:ce_full} we can obtain a relation between the life-time and the \gls{voc}.
	The main recombination mechanism in perovskite solar cells at open circuit conditions is the surface recombination (see \cpageref{intro_prv_recombination}) which, as described in \cpageref{intro_surface_recombination}, can be considered as a first-order reaction with regards to the carriers \label{tpv_chemical_charge}in the perovskite layer.
	This carrier density is related to the chemical capacitance represented as the exponential addend in \cref{eq:ce_full}, so expanding $n_0$ from \cref{eq:tpv_tau_order} we obtain:
%		For perovskite solar cells, the linear part of $n(V)$ related to the geometric capacitance of charges accumulating in the selective contacts cannot be ignored and a discussion on whether to include it or not is needed.
%	This "chemical" charge (charge stored in the perovskite layer) is represented as the exponential part of \cref{eq:ce_full}, so we obtain:
	\begin{dmath}\label{eq:tpv_tau_vs_intensity}
		\tau \approx (\Phi k_\Phi n_0^{\Phi-1})^{-1} = u_8\left[\exp(\frac{q\voc}{mk_|B|T}) - 1\right]^{1-\Phi} \approx u_8\exp(-\frac{(\Phi-1)q\voc}{mk_|B|T}) = u_8\exp(-\frac{q\voc}{vk_|B|T})
	\end{dmath}
	where we used that $qV \gg k_|B|T$ for all non-dark cases in order to neglect the $-1$ addend.
	So the life-time typically decreases exponentially with the light bias with an ideality modifier of $v = m/(\Phi-1)$.
	This relation is in accordion with equivalent studies on \gls{osc} \cite{Shuttle2008,Shuttle2008d,Credgington2011}.

\mysection[TPV-CE]{Transient PhotoVoltage Referenced to Charge Extraction (TPV-CE)}\label{tpvce}

\paragraph{Concept}
This is a meta experiment which combines the data from \acr{tpv} and \acr{ce} without needing any additional experimental step.
The voltage dependency of the small perturbation life-time $\tau(V)$ obtained from \acr{tpv} gets expanded with the $V(n)$ relation which can be obtained inverting the fitted function of $n(V)$ from a \acr{ce} experiment for obtaining a $\tau(n)$ relation.
A reaction order $\Phi$ can be obtained fitting the $\tau(n)$, this can be used for correcting the small perturbation life-time to obtain a pseudo first order life-time \textit{i.e.} total carrier life-time $\tau_|pfo|(n)$ which can be compared between devices with different recombination mechanisms.

\paragraph{Which charge from \gls{ce}}
In \gls{osc} literature, where a simple exponential charge \textit{versus} voltage is usually observed, the full $n$ obtained by \acr{ce} is taken.
In perovskite solar cells, the charge from chemical capacitance is usually considered as the relevant charge density for surface recombination, as explained in \cpageref{tpv_chemical_charge}, so just the exponential addend is taken from the fit of \acr{ce} data performed with \cref{eq:ce_full} \cite{Du2018,Gelmetti2019,Wheeler2017}.
As a reference, the \cref{fig:tpvce-full,,fig:tpvce-nogeom} can be compared where the full charge from \acr{ce} is employed in the former while just the charge from chemical capacitance was used for the latter.
Using \cref{eq:tpv_tau_order} for fitting the data, we can obtain physically unreasonable recombination orders when considering the full charge from \acr{ce} (reported in \cref{fig:tpvce-full}, with the same ordering of the legend respectively $5.4$, $3.2$, $9.8$, and $16.0$).
Instead, fitting data with the exponential component of \acr{ce} only, the recombination orders obtained are between 1 and 2, which are the expected values (reported in \cref{fig:tpvce-nogeom,,fig:tpvce-nogeom_total} with the same ordering of the legend, respectively $1.6$, $1.5$, $1.9$, and $1.8$).
	%  information on the recombination processes, it is more useful to relate the recombination life-time to the charge concentration, rather than to the light bias, as in a pure \acr{tpv} experiment.
%	One way to obtain this is taking the $\tau (V)$ relation from \acr{tpv} and substituting $V$ with the inverse of $n(V)$ obtainable from \acr{ce} experiments, obtaining the $\tau (n)$ relationship.
%	For perovskite solar cells, the linear part of $n(V)$ related to the geometric capacitance of charges accumulating in the selective contacts cannot be ignored and a discussion on whether to include it or not is needed.
%	In this thesis recombination order $\Phi = \lambda + 1$  AAAAAAAAAAAAAAAAAAAAAAAAAAAAAA

\begin{figure}
	\makebox[\textwidth][c]{
		\parbox{1.1\textwidth}{
			\centering
			\begin{subfigure}[t]{0.5\textwidth}
				\includegraphics[width=1\textwidth]{tpvce/spiro_vs_TAEs-TPVCEs-crop.pdf}
				\subcaption{Full charge from \gls{ce}}\label{fig:tpvce-full}
			\end{subfigure}
			\bigskip
			
			\begin{subfigure}[t]{0.5\textwidth}
				\includegraphics[width=1\textwidth]{tpvce/spiro_vs_TAEs-TPVCEs_nogeom-crop.pdf}
				\subcaption{Exp.\ charge from \gls{ce}}\label{fig:tpvce-nogeom}
			\end{subfigure}
			\qquad			
			\begin{subfigure}[t]{0.5\textwidth}
				\includegraphics[width=1\textwidth]{tpvce/spiro_vs_TAEs-TPVCEs_nogeom_total-crop.pdf}
				\subcaption{Exp.\ charge from \gls{ce} and life-time corrected with $\Phi$}\label{fig:tpvce-nogeom_total}
			\end{subfigure}
			\mycaption[Example of TPV-CE processed in different ways.]{
				Solid lines are the power-law fit with the formula $y = k_\Phi^{-1} x^{1-\Phi}$.
				In (\textbf{a}) the charge used for the independent variable is taken from the full charge extracted in the \acr{ce} experiment, including the charge stored in contacts and electrodes.
				The $\tau_|pfo|$ life-time as obtained from \acr{tpv} is used as dependent variable.
				In (\textbf{b}) just the charge from chemical capacitance (exponential addend in \cref{eq:ce_full}) is used as independent variable.
				In (\textbf{c}) the $\tau_|pfo|$ is corrected to obtain the total carriers life-time $\tau = \Phi\tau_|pfo|$.
				Data from \gls{fto}\-/\ch{d-TiO2}\-/\acr{csfamapbibr}\-/\gls{htm}\-/Au solar cells \cite{Gelmetti2019}.
			}\label{fig:tpvce}
		}
	}
\end{figure}

\paragraph{From small perturbations life-time to rate constant -- first-order reaction}
Let's consider a simple case where just a first order or a second order recombination is present.
From the \cref{eq:tpv_tau_order} it is clear that for the first-order reaction case it is easy to obtain $k_1 = \tau_|pfo|^{-1}$.
One fact of \cref{eq:tpv_tau_order} for the $\Phi=1$ case have to be underlined: the small perturbations life-time for a first-order recombination does not depend on the charge density, so it would appear as a plateau in the life-time \textit{versus} charge plot \cite{Kiermasch2018}.

\paragraph{From small perturbations life-time to rate constant -- higher order reaction}
For the second and higher order recombination case, to obtain $k_x$ is non-trivial as it is related to $\tau_|pfo|$ \textit{via} the charge density \cite{ORegan2007} and the reaction order \cite{Shuttle2008,Du2018,Barnes2011,Barnes2011a} as shown in \cref{eq:tpv_tau_order}.
From this equation we can notice that correcting the small perturbation life-time $\tau$ we can easily obtain a pseudo first order life-time (total carrier life-time) $\tau_|pfo|$ as:
$$\tau_|pfo| = \Phi \tau = k_\Phi^{-1} n_0^{1-\Phi}$$
This simple correction allows us to make a meaningful comparison between recombination rates, as represented in \cref{fig:tpvce-nogeom_total}.
%Because of this, when an information on underlying rate constants is sought, comparisons of $\tau_|pfo|$ have been performed taking care of having similar charge densities in the devices under study \cite{ORegan2008}, \textit{i.e.} comparing vertically (points at the same charge density) in \cref{fig:tpvce}.
%\paragraph{Obtaining the recombination reaction order}
%As shown in \cref{eq:tpv_tau_order}, we can extract the reaction order from a power-law fitting of the $\tau_|pfo|$ \textit{versus} $n$ from chemical capacitance data.

High values: high m with inhomogeneous charge profiles Kirchartz2012

thin undoped films dependency on thickness Deledalle2015 Deledalle2014

\section{Transient PhotoCurrent (\glsentryshort{tpc})}\label{characterization_tpc}

	\paragraph{Concept}
	This technique, applied to a solar cell at short circuit conditions, allows us to study the dependence of \gls{eqe} on the light illumination intensity \cite{ORegan2004}.
	It gives approximatively the same information of the \gls{jsc} \textit{versus} light intensity experiment explained in \cpageref{jsc-phi}.
	Usually, the actual value of \gls{eqe} (generated charge over incident photons ratio) is not calculated, rather just the generated charge amount for a given laser pulse intensity in measured.
	%, rather than actual value of \gls{eqe} (generated charge over incident photons ratio) is obtained as the incident power is not usually measured (on the contrary to the classical \gls{eqe} experiment).

\begin{SCfigure}
	\centering
	\includegraphics[width=0.5\textwidth]{tpc/tpc_vs_tpc-TAE-4_ig101-1555-1.pdf}
	\mycaption[Example of TPC experiment at dark and 1 sun background illumination.]{
		The current profile of a \gls{fto}\-/\ch{d-TiO2}\-/\acr{csfamapbibr}\-/\gls{tae4}\-/Au solar cell \cite{Gelmetti2019} short circuited through a small resistance either in dark or illuminated with \SI{1}{sun} and perturbed with a laser pulse.
		The voltage axis can easily be converted to the current axis dividing by the known small resistance of \SI{50}{\ohm}.
		The integrated peak resulted in \SI{0.17}{\nano\coulomb} for the dark case and \SI{0.14}{\nano\coulomb} for the \SI{1}{sun} one.
	}\label{fig:tpc}
\end{SCfigure}

	\paragraph{Procedure}
	The device is kept either at dark or under 1~sun illumination by a white \gls{led} ring short circuited through a \SI{50}{\ohm} resistor until stabilization is reached.
	1~sun equivalent illumination is defined as the illumination at which the a silicon photodiode gives the same \gls{jsc} as under calibrated 1~sun from the solar simulator.
	Then an additional illumination pulse is provided \textit{via} a nitrogen laser and the voltage across the resistor is monitored using an oscilloscope connected in parallel.
%	The signal is acquired by an oscilloscope in parallel to the \SI{50}{\ohm} resistor.
	This allows us to measure a potential drop across the resistor and the related current via Ohm's law $J = V / \Omega$.
	Subtracting the constant current due to the background illumination and integrating the transient over time gives the charge generated by the laser pulse.
	This process is repeated at 1~sun and at dark background illumination conditions.
	For each illumination intensity, the reported decay is the result of averaging around 30 transients in order to decrease the electronic noise.

\paragraph{Small resistance}
The characteristic time of a \acr{tpc} decay is often comparable with the small perturbation life-time from \acr{tpv} experiment at high background illumination.
This not necessarily invalidate the \gls{tpc} result, as the slow time could be due to the slow transport in the selective contacts.
The \gls{tpc} time can be limited by the RC time constant of the extracting circuit, in case a faster extraction is needed, a smaller resistance can be employed (clearly, the measure will be more noisy).

	\paragraph{Dependency on background illumination}\label{tpc_intensity}
	When measuring \acr{tpc} at various background illumination intensities, a constant pulse-generated charge amount can be gathered for low illuminations that starts decreasing as the background illumination exceeds \SI{1}{sun} (as reported for perovskite solar cells in fig.~S5 of \cite{Du2018} and partially in fig.~S9 of \cite{Wheeler2017}).
	In \gls{osc} this was assigned to primary geminate recombination, which, as seen in \cpageref{intro_geminate}, is negligible in perovskite materials.
	The influence of the different electric field intensity on the absorption, explained in \cpageref{intro_electroabsorbance}, can be neglected as the field is screened in most of the absorber and anyway the electro absorbance phenomena should have just a small contribution.
	Other kind of recombination can affect the extracted charge, even in short circuit conditions, if the charge extraction is not sufficiently quick.
	For this reason, in case of large discrepancies between the dark and the \SI{1}{sun} background illumination measures, the dark one better describes the photo-generated free-charge (in case the dark and illuminated results were different, the first quartile of all \acr{tpc} measurements was used).

\section{Differential Capacitance (DC)}

	\paragraph{Concept}
	\Acr{dc} allows us to measure the capacitance of the solar cells at different light biases.
	Integrating this capacitance profile over the voltage we can obtain the extra charge density \textit{versus} light bias profile.
	This information is equivalent to the results from \acr{ce}, with the advantage that this technique is a small perturbation technique, working on a stabilized device around its steady state conditions.
	\Acr{dc} is a meta-measurement as it just combines the data from \acr{tpv} and \acr{tpc} without requiring any additional experimental step \cite{ORegan2005,ORegan2006,Shuttle2008,Credgington2014,Maurano2011}, sometimes also referred to as "differential charging".
	From \acr{tpc} we obtain how much charge $\Delta n$ has been generated by the laser pulse and from \acr{tpv} we obtain the voltage increase $\Delta V$ due to the additional charge.
	We assume that the \gls{eqe} is the same at short and open circuit conditions, which means: the charge measured in \acr{tpc}, at short circuit, will be used for referencing data from \acr{tpv}, at open circuit.
The needed value of $\Delta V$ is the \gls{voc} increase due to the laser pulse, prior to the decay to steady state, for each illumination intensity.
	Adapting the definition of capacitance "the ratio of the change in an electric charge in a system to the corresponding change in its electric potential" \cite{WikipediaCapacitance2019} to an applied voltage dependent capacitance, we can write:
%	$$C(V') = \left.\frac{dn}{dV}\right\rvert_{V=V'}$$
	$$C(V') = \eval{\dv{n}{V}}_{V=V'}$$
	This allows us to estimate the capacitance of the solar cell device at open circuit with various illumination intensities.
	Additionally, this technique has been used for estimating the electronic band gap of the perovskite layer in \authoryear{Wheeler2017}, considering that also for \acr{dc} as seen for \acr{ce} in \cpageref{ce_energy_levels}, the exponential trend is expected to gain importance as the quasi-Fermi splitting approaches the built-in voltage.
	
	\paragraph{Procedure}
		The charge obtained from the \acr{tpc} experiment is divided by an array of voltage increases ($\Delta V(V')$) values obtained from \acr{tpv}, one for each illumination intensity.
		The obtained capacitance value is plotted versus the steady state light bias $V'$.
		The laser pulse intensity has to be the same for the \acr{tpc} and the \acr{tpv} experiments.
		Additionally to the assumption from \acr{tpc} of charge generation (\gls{eqe}) being independent from background light illumination in the utilised illumination range, as explained in \cpageref{tpc_intensity}, for \acr{dc} we need to add the same assumption also for the open circuit conditions used for \acr{tpv} experiment.
	
\paragraph{Capacitance dependence on applied voltage}
	The electrical capacitance of most of the commercial capacitors is independent on the applied voltage, as represented in \cref{fig:cap_voltage_dependence_commercial} where the capacitance was measured with a modified \acr{ce} experiment where the pre-conditioning was done directly applying a voltage instead of applying a light bias.
	This means that the extracted charge is linearly proportional to the applied voltage.
	On the contrary, the electrical capacitance of a solar cell does depend on the applied voltage or light bias, as shown in \cref{fig:cap_voltage_dependence_tae1}.
	\begin{figure}%[!hbtp]%
			\makebox[\textwidth][c]{
		\parbox{1.1\textwidth}{
		\centering
		\begin{subfigure}[t]{0.5\textwidth}
			\includegraphics[width=0.9\textwidth]{cap_voltage_dependence/reference220nF/reference220nF.pdf}
			\subcaption{Commercial \SI{220}{\nano\F} capacitor.}\label{fig:cap_voltage_dependence_commercial}
		\end{subfigure}
		\qquad
		\begin{subfigure}[t]{0.5\textwidth}
			\includegraphics[width=0.9\textwidth]{cap_voltage_dependence/TAE-1_ig94-1559-1/DC-capacitance-TAE-1_ig94-1559-1.pdf}
			\subcaption{Perovskite solar cell.}\label{fig:cap_voltage_dependence_tae1}
		\end{subfigure}
		\mycaption[Capacitance dependence on applied voltage.]{
			In (\textbf{a}) the capacitance of a commercial capacitor is reported, it was measured using \acr{ce} with applied voltage bias instead of the classical light bias used for solar cells.
			The capacitance is obtained as the extracted charge over the applied voltage prior to short circuiting.
			In (\textbf{b}) the typical capacitance \textit{versus} voltage profile of a \gls{fto}\-/\ch{d-TiO2}\-/\acr{csfamapbibr}\-/\gls{tae1}\-/Au device \cite{Gelmetti2019} is shown as measured by \acr{dc}.
			In this case the indicated voltage is originated by various illumination intensities at open circuit prior to short circuiting.}\label{fig:cap_voltage_dependence}
	}}
	\end{figure}

	\paragraph{Small perturbations}\label{dc_perturbation}
	As mentioned in \cpageref{tpv_perturbation}, we measure \acr{tpv} ensuring to be in the small perturbation conditions for high light bias illumination.
	But the perturbations are surely large getting close to dark illumination conditions.
	This happens because we're not changing the laser pulse intensity when decreasing the background illumination.
	This way, we can measure just one \acr{tpc} for knowing the amount of generated charge, as we assume that it depends just on the laser pulse intensity.
	The measure of as many \acr{tpc} as many laser pulse intensities would be too complex with our current experimental setup.

\paragraph{Consideration on mobility limited case}
	In order to have a meaningful voltage peak value, the charges have to equilibrate with the cell electrodes quickly compared to the recombination rate \cite{Credgington2014}.
	In a mobility limited case, as explained in \cpageref{tpv_mobility}, the charges could recombine so quickly that the $\Delta V$ value could be underestimated.

	\paragraph{Voltage peak value from \acr{tpv}}\label{tpv_deltaV}
	The $\Delta V$ value can be obtained in various ways, the following methods were tested:
	\begin{itemize}
		\item The maximum voltage point, subtracting the steady state \gls{voc}. This is the classical way to obtain $\Delta V$ but it is heavily affected by the aforementioned noise when a short time window is used.
		\item The linear factor $\Delta V$ in an exponential fit $V (t) = V_0 + \Delta V \cdot e^{-kt}$ was used, but it can fail if the decay does not have a simple exponential shape (often a bi-exponential, sum of two exponentials, is observed).
		\item In cases where the \acr{tpv} decays have a bi-exponential behaviour, the sum of the two linear factors $\Delta V = \Delta V_1 + \Delta V_2$ in a bi-exponential fit as in \cref{eq:tpv_biexp} could improve the previous method. One have to carefully set boundary values to the fitting parameters for avoiding a fast exponential matching just some noise.
		\item The maximum value of a \gls{loess} local regression was used, but this underestimates the value, especially when the peak top are just few points (when the measurement time window is large).
		\item The average of the values registered starting from the maximum voltage point and during a specified time lapse. For selecting the maximum voltage without interferences from the noise, a custom smoothing function is used.
	\end{itemize}
	This last option is the one used in this thesis. 
	The average was performed over \SI{50}{\nano\s} after the peak and this allowed us to get a reliable $\Delta V$ value.
	As an example, we can compare the $\Delta V$ obtained with the aforementioned criteria applied on the \acr{tpv} decay in \cref{fig:tpv_robust}: \SI{16.5}{\mV} from the maximum point; \SI{4.2}{\mV} from the simple mono-exponential fit; \SI{3.3}{\mV} from the robust mono-exponential fit ; \SI{5.9}{\mV} from the simple bi-exponential fit; \SI{6.1}{\mV} from the robust bi-exponential fit (see \cpageref{tpv_robust} for discussion on robust fitting); \SI{4.2}{\mV} from the maximum of the \gls{loess}; \SI{4.6}{\mV} using the average of the first points after the peak.
	To further illustrate the sensibility of \acr{dc} from the $\Delta V$ estimation, the capacitance of four devices from \cite{Gelmetti2019} has been reported in \cref{fig:dc_deltaV}.
	If the object of study is related to the presence of negative peaks, present in non-stabilized solutions as described in \cpageref{tpv_negativePeaks}, the data fitting code should be modified.
	A non stabilized ionic profile during the \acr{tpv} measurement can lead to shifts in the \acr{dc} profile \cite{ORegan2015b}.
	The biphasic behaviour of \acr{tpv} decays at low background intensity due to ionic migration, discussed on \cpageref{tpv_biexp_lowlight_ions}, does not reduce the validity of the $\Delta V$ value for capacitance determination, as at very short times the ionic profile is effectively frozen.
	
	\begin{figure}
		\makebox[\textwidth][c]{
			\parbox{1.1\textwidth}{
				\centering
				\begin{subfigure}[t]{0.5\textwidth}
					\includegraphics[width=0.75\textwidth]{dc_deltaV/deltaV-spiro_vs_TAEs-DCs-capacitance-crop.pdf}
					\subcaption{Maximum voltage point}\label{fig:dc_deltaV-deltaV}
				\end{subfigure}
				\qquad
				\begin{subfigure}[t]{0.5\textwidth}
					\includegraphics[width=0.75\textwidth]{dc_deltaV/monoexp-spiro_vs_TAEs-DCs-capacitance-crop.pdf}
					\subcaption{Linear factor in an exponential fit}\label{fig:dc_deltaV-monoexp}
				\end{subfigure}
				\bigskip
				
				\begin{subfigure}[t]{0.5\textwidth}
					\includegraphics[width=0.75\textwidth]{dc_deltaV/biexp-spiro_vs_TAEs-DCs-capacitance-crop.pdf}
					\subcaption{Sum of linear factors in bi-exp.\ fit}\label{fig:dc_deltaV-biexp}
				\end{subfigure}
				\qquad
				\begin{subfigure}[t]{0.5\textwidth}
					\includegraphics[width=0.75\textwidth]{dc_deltaV/loess-spiro_vs_TAEs-DCs-capacitance-crop.pdf}
					\subcaption{Maximum of a \gls{loess} smoothing}\label{fig:dc_deltaV-loess}
				\end{subfigure}
				\bigskip
				
				\begin{subfigure}[t]{0.5\textwidth}
					\includegraphics[width=0.75\textwidth]{dc_deltaV/firstPoints-spiro_vs_TAEs-DCs-capacitance-crop.pdf}
					\subcaption{Averaging the peak first points}\label{fig:dc_deltaV-firstPoints}
				\end{subfigure}

		\mycaption[Comparison of methods for obtaining \glsentryshort{dc} experiments.]{
			The capacitance obtained from \gls{dc} experiment on \gls{fto}\-/\ch{d-TiO2}\-/\acr{csfamapbibr}\-/\gls{htm}\-/Au devices using different methods for extracting $\Delta V$ value.
		}\label{fig:dc_deltaV}
				}
	}
	\end{figure}

	\paragraph{Capacitance from voltage and current increase during the pulse}
	In case a slow light pulse have to be employed, \textit{e.g.} from a \gls{led} source, the voltage peak is too strongly affected by the ongoing recombination.
	An alternative method considering the plateau current of a \acr{tpc} experiment \textit{versus} the voltage grow rate from \acr{tpv} has been reported \cite{ORegan2006,ORegan2015b}.

	\paragraph{Comparison of charge from DC and from CE}
	The \acr{dc} experiment has been demonstrated to output a very similar charge \textit{versus} light bias profile as a \acr{ce} experiment when used on \gls{dssc} \cite{ORegan2005,Barnes2013} and on \gls{osc} \cite{Shuttle2008a}.
	\Acr{dc} started to be employed in perovskite solar cells characterization due to the excessively large charge amount sometimes measured by \acr{ce} \cite{Wheeler2017,ORegan2015b}.

\section{Voltage and Current Reconstruction}
As a self consistency check for the photophysics techniques explained and for confirming that no important factor has been neglected, the performance parameters of the devices can be reconstructed from the fitted parameters from \gls{tpv}, \gls{ce}, and \gls{dc}.
The shown \gls{jsc} and \gls{voc} reconstruction methods can be extended to obtain the whole current-voltage sweep \cite{Maurano2011}.

\paragraph{\Gls{jsc} reconstruction}
The recombination current $J_|rec|$ can be described by the total carriers life-time (or pseudo first order life-time) $\tau_|pfo|$ and by the excess charge $n_|CE|$ with the following expression \cite{Wheeler2017,Du2018}: $J_|rec| = n_|CE| / \tau_|pfo|$.
At open circuit the photo-generated current has to match the recombination current (no external current means that these two, at least in steady state conditions, cancel out), so $J_|rec| = J_|ph|$.
If losses at short circuit are negligible, photo-generated current can be approximated by the short circuit current $J_|ph| \approx \jsc$ obtaining \cite{ORegan2015b}:
\begin{equation}\label{eq:reconstruction_jsc}
\jsc = \frac{n_|CE|}{\tau_|pfo|}
\end{equation}

\paragraph{\Gls{voc} reconstruction}
Expanding $\tau_|pfo| = \Phi \tau$ with \cref{eq:tpv_tau_order} and neglecting the intrinsic charge density (consideration valid for non-dark cases) so that $n_0 \approx n_|CE|$ we get:
$$\jsc = n_|CE| \cdot k_\Phi n_|CE|^{\Phi-1}$$
Then considering just the chemical capacitance charge as relevant for recombination processes, using the exponential addend in \cref{eq:ce_full} we obtain:
$$\jsc = n_|pre|^\Phi k_\Phi \me^{\frac{q \Phi \voc}{m k_|B| T}}$$
where we neglected $-1$ addend for non-dark cases, inverting the equation we obtain:
\begin{equation}\label{eq:reconstruction_voc}
	\voc = \frac{m k_|B| T}{q \Phi} \ln( \frac{\jsc}{n_|pre|^\Phi k_\Phi} )
\end{equation}
%Expanding \cref{eq:tpv_tau_order} with the relevant $n_0$ for recombination taken from the chemical capacitance charge represented by the exponential addend in  we obtain:
%$$\tau_|pfo| = \left\{\Phi k_\Phi n_|pre|^{\Phi-1} [\exp(\frac{q \voc}{m k_|B| T})-1]^{\Phi-1} \right\}^{-1}$$
The \cref{eq:reconstruction_voc} now contains just parameters which can be obtained from our photophysical measurements \cite{Wheeler2015,Wheeler2017,Du2018,Barnes2011a}.

\section{Impedance Spectroscopy}
\epigraph{\textit{"One does not simply understand impedance spectroscopy data"}}{Boromir}

\paragraph{Frequency domain}
Thanks to Kramers-Kronig relation, from frequency domain techniques we should be able to extract the same information as from time-domain techniques, as the aforementioned ones.
Nevertheless, frequency domain techniques are more convenient for observing one by one all the different processes happening in a solar cell device, working at their characteristic frequency.
On the other hand this also implies a much longer measurement, for which the device stability is paramount.

\paragraph{Concept}
Impedance spectroscopy can be performed with easily available equipment: most of the electrochemical units used in the research labs for measuring cyclic voltammetry include this technique.
An alternated voltage at frequency $f$ is applied to the two electrodes of the device and the resulting alternated current is observed in amplitude and phase.
For an accurate phase estimation usually a lock-in amplifier is used.
The in-phase component can be taken for calculating a resistance $Z'(f)$ and the quadrature component for calculating the reactance $Z''(f)$.
The sum of these two components give the complex impedance: $Z(f) = Z'(f) + iZ''(f)$.
The voltage oscillation should be large enough for measuring a current profile without too much noise, usually amplitudes from \SIrange{20}{200}{\mV} are employed.
Repeating this on different alternating voltage frequencies, often from \SIrange{1e-2}{1e7}{\Hz}, the spectra of impedance can be obtained.
The impedance spectroscopy can be represented in various shapes: Nyquist plots where $-Z''$ is plotted against $Z'$, Bode plots of phase, $|Z|$, $Z''$, or $Z'$ are plotted against the frequency $f$ or, as a capacitance (or apparent capacitance, as we will see in \cref{ch:impedance}) defined as $\omega^{-1}\Im(Z^{-1})$.

\paragraph{Stability}
As the measurement of a complete impedance spectra is very time demanding, the device stability is of paramount importance.
Moreover, degradation processes or stabilization happening within the studied $1/f$ time span (between hundreds of seconds and microseconds) will result in glitches which can be easily mistaken for interesting data \cite{Jacobs2018,Moia2019}.

\paragraph{Interpretation}
Due to the large amount of extracted data and the complex features usually observed when studying perovskite solar cells (\textit{e.g.} multiple arcs in the Nyquist plot), the parameters extraction is usually done \textit{via} fitting with an equivalent electrical circuit.
A bad choice of the fitting circuit can doom the meaningfulness of the obtained information, so great care has to be taken in order to have a bijective relation between the circuit elements and the relevant physical processes. 

%\section{Stark Spectroscopy (ElectroAbsorbance)}

%\section{Interpretation of Kelvin Probe Force Microscopy}\label{interpretation_kpfm}

%\section{Molecular Characterization}
%	\subsection{Interpretation of Band Gap Values Obtained via Tauc Plot, PhotoLuminescence and Computational Simulations}\label{interpretation_bg}

		%	Otra cosa, flipé mucho con la respuesta del Vidal y me puse a intentar
		%	ver que se supone que se saca del espectro experimental y desde cual
		%	pico. Como el dijo, las simulaciones son correctas.
		%	Pero, creo que sea mi concepto (el HOMO-LUMO está bastante lejos del
		%	absorption onset, como demostrado de sus simulaciones donde el
		%	absoprtion onset, por ejemplo de TAE-1, está a 2.9 eV y el HOMO-LUMO gap
		%	está a 5.06 eV) que lo de Vidal (el pico de máxima absorción no tiene
		%	nada a que ver con el HOMO-LUMO gap, en mi opinión para nada) estaban
		%	totalmente equivocados y que habría que enviar a medir el UPS de las
		%	moléculas para sacar el HOMO-LUMO gap.
		%
		%	El concepto está explicado en el primer párrafo de esta pagina:
		%	https://chemical-quantum-images.blogspot.com/2013/06/why-is-homo-lumo-gap-not-good-guess-of.html
		%
		%	Lo que he puesto, o sea que lo hemos medido por Tauc plot, creo colaría
		%	porque todo el mundo lo mide así y está bien aceptado como método. Pero
		%	la verdad es que es valido solo para semiconductores donde los orbitales
		%	están bien deslocalizados, no como nuestra molécula en solución donde el
		%	estado es bien localizado en la molécula.

%\section{Our Solar Cells Characterization Steps}
%
%	In this section I'll describe the routinary characterization performed in Palomares group.


