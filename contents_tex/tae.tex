\epigraph{\textit{"It's so easy, can't you see the shift?"}}

\paragraph{Abstract} In perovskite solar cells, the absorber is usually sandwiched between two different contacts: the \gls{htm} and the \gls{etm} with the role of extracting respectively the holes and the electrons. Without this asymmetrical extraction of charges, the photogeneration would be of no use. The classical \gls{etm} from \gls{dssc}, mesoporous titania, is slowly getting replaced by planar tin oxide \cite{Jiang2018}. On the contrary, the classical \gls{htm} from solid-state \gls{dssc}, \spiro, is still present in most of the record structures. The huge explorative work done for finding a more performing \gls{htm} had some success with a few molecules and, for some cases, with \gls{ptaa}. In most of the cases, even if the performances are at par with \spiro, the price is still too high for wide area applications. A better understanding of the \gls{htm}/perovskite interaction is needed for pinpointing the key characteristics to be looked for in the next \gls{htm} design. In this chapter, the devices fabricated using four different \gls{htm} have been compared in order to find a correlation with the \gls{htm}'s chemical properties.

\paragraph{Publications} Part of this chapter has been published in %\fullcite{}.


\section{Interpretation of \gls{voc} from Current-Voltage Sweeps}














\section{Recombination analysis via TPV}
\section{Stored charge profile via DC}
\section{Layers workfunction via KPFM}
\section{Conclusions}
\section{Critical Assessment}