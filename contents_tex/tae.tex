


\section{Introduction}
From Gelmetti2017:
Nonetheless, other device architectures can be found at the scientific literature that have in common the use of the TCO and the photoactive layer but either a combination of all-inorganic selective contacts (for example TiO2 for electrons and NiOx for holes2) or, acquiring more relevance, from the OPV school, a combination of all-organic selective contacts (for example fullerene derivatives for electrons and PEDOT:PSS for holes, reaching 15 % with iodine perovskite3-4 or 17 % doping the PEDOT:PSS layer5, 18 % for mixed iodine-bromine perovskite6). Comparable efficiencies have reported using an organic selective contact (for example fullerene derivatives for electrons) in combination with doped NiOx for holes, reaching 17 % efficiency.7-10
On the other hand, the number of semiconductor molecules and materials employed as hole transport contact layer has increased exponentially in order to replace the most commonly used spiro-OMeTAD for alternative hole transporting molecules that can be easily synthesized18-20 or that don’t require the addition of dopants.21-22 In a lesser extent, a number of electron transport layers have also been investigated but extensively ridden by the use of TiO2 in normal structures and fullerene derivatives in inverted devices.23


different substrate makes different perovskite morphology \cite{Tao2017}

\section{Interpretation of \glsentryshort{voc} from Current-Voltage Sweeps}


Solvent annealing of contact Wu2016

Energy disorder Shao2016

HOMO shift measured by CE and DC has been correlated with UPS for OSC Credgington2014

Steepness of absorbance onset also indicates the presence/absence of mid-gap states in the HTM, whose presence would favour the surface recombination \cite{Tvingstedt2017}

Impact of dopants in HTM on voc \cite{Correa-Baena2017} "Therefore, dopants act as recombination centers at the HSL interface and dominate recombination dynamics over e.g. the energetics of the HTM which is secondary as shown in a recent study by Belisle et al.39"

"the often-questionable validity of vacuum level alignment, the importance of interface dipoles, and band bending as the result of interface formation" \cite{Schulz2019}


\section{Recombination analysis \textit{via} TPV}
\section{Stored charge profile \textit{via} DC}
\section{Layers workfunction \textit{via} KPFM}
\section{Conclusions}
\section{Critical Assessment}