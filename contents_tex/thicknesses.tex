\epigraph{\textit{"Let's change as little as possible"}}

\paragraph{Abstract} A series of top cathode perovskite solar cells have been fabricated with slightly different layers thicknesses. We verified that these variations affected as little as possible the rest of the solar cell stack, so that we could unequivocally relate the observations to the changes. The characterization via current-voltage sweeps and photophysical techniques allowed us to study the charge distribution in these devices.

\paragraph{Publications} Part of this chapter has been published in \fullcite{Gelmetti2017}.

\section{Introduction}

	\paragraph{Disentangling parameters' impact} Perovskite synthesis is a very easy and very fragile process at the same time. When varying a fabrication parameter, for example during an optimization, it is quite likely to provoke a "butterfly effect" with the resulting device differing from the reference one by much more than the characteristic under study. A principal component analysis of the fabrication parameters would be needed for a rational optimization, but such a complex procedure is further hindered by the difficulty of identifying all relevant contributions.

	\paragraph{Varying the thickness} In this study we vary a set of parameters that hopefully have a foreseeable relation with the resulting device structure: The spin coating speed for each layer. The effect of this variation should just affect the layers thicknesses with a minor influence on the other device physical features. This will allow us to univocally relate each layer thickness variation to the characterization results variation. The complete devices were studied by means of current-voltage sweeps, \acr{ce}, \acr{tpv}, and \acr{dc}.

	\paragraph{The devices} The chosen architecture was a top cathode \gls{ito}\-/\gls{pedotpss}\-/\gls{mapi}\-/\gls{pcbm70}\-/\ch{Ag} device (fabrication described in \cpagerefrange{methods_top}{methods_top_end}) and the layers whose thickness was independently varied are the \gls{pedotpss} (\gls{htm}), the \gls{mapi} (absorber), and the \gls{pcbm70} (\gls{etm}).

\section{Interpretation of \glsentryshort{jsc} and \glsentryshort{ff} from Current-Voltage Sweeps}

\section{Interpretation of Transient PhotoVoltage Referenced to Differential Capacitance}\label{interpretation_tpvdc}

\section{Varying \glsentrytext{mapi} Thickness (Absorber Layer)}


\section{Varying \glsentrytext{pcbm70} Thickness (\glsentryshort{etm} Layer)}

s-shape curve when thick pcbm \cite{Wheeler2017} "build-up of space charge due to the restriction of carriers out of the device due to the slow mobility associated with the PCBM."


"The physical meaning of a small majority carrier surface-recombination velocity is an extraction barrier that would eventually lead to S-shaped light J-V curves. Experimentally, such S-shaped curves are not observed for the devices studied herein; however, S-shaped J-V curves could be observed in simulations when Smin is reduced and are discussed further in Supplemental Material [39]."
"Effect of reduced surface recombination velocityBothSmajandSminare predicted to cause S-shaped JV curves when very small, as seen infigure 7.  In the case ofSmaj,  as described by Wagenpfahlet al.,  a lowSmajrestrictsextraction of majority carriers out of the device.  This leads to a build-up of space chargein the device, lowering the VOC.  In the case of a lowSmin, restricting surface recombinationof  minority  carriers  causes  them  to  accumulate  at  the  electrode  around  open-circuit.   Inthe case of electrons at the anode, this causes the conduction band and the Fermi-level tocome closer together.  At open-circuit, by definition the Fermi-level is flat;  this forces theconduction band to bend down.  At voltages just less than VOC where charges still need tobe extracted, there is additional bulk recombination losses as electrons now have to diffuseagainst the electric field over a barrier to get collected at the cathode.  The simulated banddiagrams can be seen in figure 9."
"S-shaped JV curves as a consequence of a reduced minority surface-recombination velocity." \cite{Wheeler2015}

change in RC time due to increased PCBM layer resistance \cite{Wheeler2017}

importance of energy disorder in pcbm layer 10.1038/nenergy.2015.1

\section{Varying \glsentrytext{pedotpss} Thickness (\glsentryshort{htm} Layer)}
\section{Conclusions}
\section{Critical Assessment}