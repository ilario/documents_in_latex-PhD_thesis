\epigraph{\textit{tl; dr}}



	\vfill
	
\paragraph{English}
This thesis includes the work done in ICIQ about fabrication, characterization, and modelling of hybrid perovskite solar cells.
Coming from other kind of solar cells, the analysis tools, the methods, and, most importantly, their interpretation have been analysed and adapted to this new kind of device.
Then, these techniques have been employed for analysing and understanding the influence of four different and novel hole transport materials on perovskite solar cells voltage.
Another study focussed on the electrons accumulation in devices employing small variations in each stacked layer thickness and analysing the samples using the same techniques.
From by international stay in Dr. Piers Barnes and Prof. Jenny Nelson groups in Imperial College London another study was originated exploring the complex interpretation of impedance spectroscopy results when applied on perovskite solar cells with mobile ions.
Finally, all the free software that has been developed for data acquisition and processing and for drift-diffusion modelling of perovskite solar cells has been exposed.

%{
%	\vfill
%	\centering \rule{5cm}{1pt}\\
%	\vfill
%}
%\selectlanguage{italian}
%\EnableQuotes
%\noindent
%Fotofisica di celle di perovskite

{
	\vfill
	\centering \rule{5cm}{1pt}\\
	\vfill
}

\selectlanguage{spanish}
\EnableQuotes
\noindent

\paragraph{Spanish}
% Traducido por mi
%Esta tesis incluye el trabajo hecho en ICIQ sobre fabricación, caracterización, y modelización de celdas solares de perovskita hibrida.
%Provenientes desde la investigación en otros tipos de celdas solares, las herramientas de análisis, las metodologías, y, aún más importante, su interpretación han sido analizadas y adaptadas a este nuevo tipo de dispositivo.
%Entonces, estas técnicas han sido utilizadas para analizar y entender la influencia de cuatros diferentes y novedosos transportadores de huecos electrónicos sobre el voltaje de celdas de perovskita.
%Otro estudio ha investigado la acumulación de electrones en las celdas utilizando pequeños cambios en el grosor de cada capa y analizando las muestras con las mismas técnicas.
%Desde mi estancia internacional en los grupos del Dr. Piers Barnes y de la Prof. Jenny Nelson en Imperial College London otro estudio ha sido llevado al cabo sobre la complexa interpretación de los resultados de espectroscopia de impedancia en presencia de iones móviles en las celdas de perovskita.
%Además, se expone todos los programas libres que han sido desarrollados para la adquisición y procesamiento de datos y para la modelización deriva-difusión de celdas solares de perovskita.

% Adaptado por Jesus Jimenez Lopez
%En esta tesis se incluye todo el trabajo realizado en el ICIQ sobre la fabricación, caracterización y modelización de celdas solares de tipo perovskita híbrida.
%Tanto las herramientas de análisis, los métodos y, sobretodo, la interpretación de datos deriva de otros tipos de celdas previamente investigadas.
%Por tanto, han sido analizadas y adaptadas a este nuevo tipo de dispositivo.
%Así, estas técnicas se han utilizado para el análisis e interpretación de la influencia que tienen cuatro nuevas y diferentes moléculas utilizadas como transportadores de huecos electrónicos sobre el voltaje de las celdas solares de perovskita.
%Otro estudio, utilizando y analizando los datos obtenidos con las mismas técnicas, se ha centrado en la influencia sobre la acumulación electrónica en dispositivos al modificar mínimamente el grosor de las distintas capas que lo conforman.
%Durante la estancia internacional en los grupos del Dr. Piers Barnes y la Prof. Jenny Nelson en el Imperial College London, se desarrolló un estudio para dar respuesta a la difícil interpretación de los datos obtenidos con la espectroscopía de impedancia en celdas solares de perovskita, debido a la influencia que tienen los iones móviles sobre ellos.
%Finalmente, se expone todo el software libre desarrollado durante la tesis para la adquisición de datos, su procesamiento, y la modelización con drift-diffusion en celdas solares de tipo perovskita.

En esta tesis se incluye todo el trabajo realizado en el ICIQ sobre la fabricación, caracterización y modelización de celdas solares de tipo perovskita híbrida.
Tanto las herramientas de análisis, los métodos y, sobretodo, la interpretación de datos deriva de otros tipos de celdas previamente investigadas.
Por tanto, han sido analizadas y adaptadas a este nuevo tipo de dispositivo.
Así, estas técnicas se han utilizado para el análisis e interpretación de la influencia que tienen cuatro diferentes y nuevas moléculas utilizadas como transportadores de huecos electrónicos sobre el voltaje de las celdas solares de perovskita.
Otro estudio, utilizando y analizando los datos obtenidos con las mismas técnicas, se ha centrado en la influencia sobre la acumulación electrónica en dispositivos al modificar mínimamente el grosor de las distintas capas que lo conforman.
Durante la estancia internacional en los grupos del Dr.\ Piers Barnes y la Prof.\ Jenny Nelson en el Imperial College London, se desarrolló un estudio para dar respuesta a la difícil interpretación de los datos obtenidos con la espectroscopía de impedancia en celdas solares de perovskita, debido a la influencia que tienen los iones móviles sobre ellos.
Finalmente, se expone todo el software libre desarrollado durante la tesis para la adquisición de datos, su procesamiento, y la modelización con \textsl{drift\hyp{}diffusion} en celdas solares de tipo perovskita.


{
	\vfill
	\centering \rule{5cm}{1pt}\\
	\vfill
}

\selectlanguage{catalan}
\EnableQuotes
\noindent

\paragraph{Catalan}
% original meu
%Aquesta tesis inclou el treball fet en ICIQ sobre fabricació, caracterització, i modelització de cel·les solars de perovskita híbrida.
%Agafades des de la recerca en altres tipus de cel·les solars, les eines de anàlisi, les metodologies i, més important, la seva interpretació han sigut analitzades i adaptades a aquest nou tipus de dispositiu.
%Llavors, aquestes tècniques han sigut utilitzades per analitzar i entendre la influència de quatre diferents i nous transportadors de forats electrònics sobre el voltatge de cel·les de perovskita.
%Un altre estudi ha investigat la acumulació dels electrons en les cel·les per mig de petits canvis en el gruix de cada capa i analitzant les mostres per mig de les mateixes tècniques.
%Des de la meva estada internacional en els grups de Dr. Piers Barnes i Prof. Jenny Nelson en Imperial College London un altre estudi ha sigut complert sobre la complexa interpretació dels resultats de espectroscòpia de impedància en presència de ions mobles en les cel·les de perovskita.
%També està exposat tot el programari lliure que ha sigut desenvolupat per la adquisició i processament de dades i per la modelització deriva-difusió de cel·les solars de perovskita.

% Corregit per Santi Gene Marimon i Maria Mendez Malaga
Aquesta tesis inclou el treball realitzat a l'ICIQ sobre la fabricació, caracterització, i modelització de cel·les solars de perovskita híbrida.
Originades des de la recerca en altres tipus de cel·les solars; les eines d'anàlisi, les metodologies i, el més important, la seva interpretació han sigut analitzades i adaptades a aquest nou tipus de dispositiu.
Així, aquestes tècniques han estat utilitzades per analitzar i entendre la influència de quatre transportadors de forats electrònics diferents sobre el voltatge de cel·les de perovskita.
Un altre estudi ha investigat l'acumulació dels electrons en les cel·les mitjançant petits canvis en el gruix de cada capa i analitzant les mostres amb les mateixes tècniques.
A partir de la meva estada internacional als grups del Dr.\ Piers Barnes i la Prof.\ Jenny Nelson a l'Imperial College London, s'ha dut a terme un altre estudi sobre la complexa interpretació dels resultats d'espectroscòpia d'impedància en presència de ions mòbils en les cel·les de perovskita.
També està exposat tot el programari lliure que ha estat desenvolupat per a l'adquisició i processament de dades, i per a la modelització del tipus deriva-difusió de cel·les solars de perovskita.

\vfill

\selectlanguage{british}
\EnableQuotes